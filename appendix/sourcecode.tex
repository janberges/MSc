% !TEX root = ../thesis.tex

\def\path{ebmb}

\chapter{Source code}
\label{source code}

On the following pages the complete source code of the \name{Eliashberg} solvers
developed as part of this thesis is exposed. This may be considered out of place
in a written work as the present one, but for the sake of completeness and a
verifiability of the presented results it is done nonetheless.

The software consists of three programs, all written in conformance with a
recent standard of the programming language \emph{Fortran}, which serve
different purposes: \code{ebmb} provides electronic self-energies on both the
real and imaginary frequency axis, \code{critical} determines critical parameter
sets by varying a parameter of choice while holding the others constant and
\code{tc} may be used to find critical temperatures for all electronic bands
separately.

The programs may be run via the command line, where all parameters are given as
arguments (except for densities of states which must be provided in text files)
and the results are either formatted and prompted to standard output or written
to disk using their internal representation, the latter being lossless but also
platform-dependent. Alternatively, an interface for the popular high-level
language \emph{Python} may be used, which is presented first.

\section{Python interface}
\label{Python interface}

The following Python module provides wrapper functions for the different
\name{Eliashberg} solvers. Parameters are directly passed to the calling
functions and the results returned as (dictionaries of) \emph{NumPy} arrays. In
addition, there are two functions to generate electronic densities of states,
either from an analytic expression in the special case of a square tight-binding
lattice or by sampling an arbitrary dispersion relation, the domain of which may
be divided into different subdomains in order to obtain separate densities of
states for each of them.

\lstinputlisting[language=py]{\path/ebmb.py}

\section{Universal modules}

This section combines Fortran modules which are either used globally, i.e. in
nearly all subsequent subroutines, or applicable universally, i.e. not only in
the context of solving the \name{Eliashberg} equations.

\subsection{global.f90}

The following module provides information which must be globally accessible.
This includes the desired accuracy of floating-point numbers, mathematical and
physical constants as well as container data types which are used to pass around
parameters and results without the need for endless argument lists at functions
calls or, more troublesome, global variables. Additionally, a operator is
defined which compares floating point numbers for approximate equality. It
decides, for example, whether numerical self-consistency has been reached or an
iteration has to continue. In all calculations performed, the default negligible
float difference $10^{-15}$ has been chosen.

\lstinputlisting[language=f90]{\path/universal/global.f90}

\subsection{eigenvalues.f90}

When testing for superconductivity via the linearized \name{Eliashberg}
equations, the quantity of interest is the greatest eigenvalue of the kernel
given e.g. in Eq.~\ref{CDOS linearized gap equation}, which is a non-symmetric
matrix with real eigenvalues. It is determined either using the corresponding
routine from the well-established linear-algebra package \emph{LAPACK}
\cite{LAPACK99} or via the power method. The latter has the advantage that an
eigenvalue of a prior calculation can be used as an initial guess if the
parameter set has only changed slightly. However, it requires that the
eigenvalue searched for notably exceeds the rest of the spectrum in magnitude.
If an interfering eigenvalue is of opposite sign, the situation is still
unproblematic since the spectrum can be shifted by adding or subtracting a lower
or upper bound to the diagonal of the matrix, respectively.%
%
\footnote{A suitable eigenvalue bound is given by a function presented in the
next section.}
%
Otherwise, the iteration will oscillate rather than converge and the method must
be abandoned.

\lstinputlisting[language=f90]{\path/universal/eigenvalues.f90}

\subsection{tools.f90}

In the following, five succinct functions or subroutines are presented which
each serve a very specific purpose and may be used in a variety of occasions.
%
\begin{itemize}
    \item \code{argument(n)}
        returns the value of the \code n-th command-line argument as a string of
        the corresponding length.
    %
    \item \code{bound(matrix)}
        returns a bound for the magnitude of the eigenvalues of the given
        \code{matrix} in terms of the minimum of the maximum row and column sums
        \cite[Eqs.~1.1, 1.2]{Wolkowicz80}.
    %
    \item \code{differential(x, dx)}
        calculates a list of weights or \q{differentials} \code{dx} from a list
        of sample points \code x to be used for numerical integration with the
        trapezoidal rule.
    %
    \item \code{interval(x, a, b, lower, upper)}
        discretizes the interval from \code a to \code b (returned as \code x),
        where \code{lower} and \code{upper} decide which bounds are to be
        included.
    %
    \item \code{matches(str, char)}
        counts the occurrences of a character in a string.
\end{itemize}

\lstinputlisting[language=f90]{\path/universal/tools.f90}

\subsection{formatting.f90}

In Fortran, the formatting of floating point numbers is controlled via so-called
\emph{edit descriptors}. Since the desired format shall be left to the user,
such edit descriptors must be generated at run time. The following module makes
this issue a little more comfortable.

\lstinputlisting[language=f90]{\path/universal/formatting.f90}

\section{\name{Eliashberg} solvers}

In this section four modules are presented which constitute the physical core of
the software, namely the solvers for the multi-band \name{Eliashberg} equations.
They are subdivided on the one hand into solvers for the self-energy on the
imaginary axis or for the maximum eigenvalue of the kernel of the linearized
equations and on the other hand according to whether the density of states is
taken into account or not.

\subsection{self\_energy.f90}
\label{self_energy.f90}

This lengthy module calculates the self-energy taking the density of states into
account, i.e. solves Eqs.~\ref{local multi-band Eliashberg equations}, in the
course of which not only particle numbers or chemical potentials according to
Section~\ref{chemical potential} have to be calculated but also the rescaled
\name{Coulomb} pseudo-potential defined in Eq.~\ref{rescaled Coulomb
pseudo-potential}. Repeatedly, energy integrals are performed numerically using
the trapezoidal rule. Since all necessary quantities are already present, the
calculation of the kernel in Eq.~\ref{linearized gap equation} is also performed
at the end of the process.

\lstinputlisting[language=f90]{\path/eliashberg/self_energy.f90}

\subsection{self\_energy\_cdos.f90}

Just as the preceding one, this module calculates the self-energy, but within
the CDOS approximation defined in Eq.~\ref{CDOS Eliashberg equations}.
Accordingly, the \name{Coulomb} pseudo-potential is rescaled following
Eq.~\ref{CDOS rescaled Coulomb pseudo-potential}.

\lstinputlisting[language=f90]{\path/eliashberg/self_energy_cdos.f90}

\subsection{eigenvalue.f90}

This module determines the maximum eigenvalue of the \name{Eliashberg} kernel
which takes the density of states into account. Since the kernel, defined in
Eq.~\ref{linearized gap equation}, is returned by the self-energy solver
presented in Section~\ref{self_energy.f90}, all that has to be done is call the
eigenvalue solvers and optionally cache the eigenvector to be reused as initial
guess if the number of \name{Matsubara} frequencies remains constant.

\lstinputlisting[language=f90]{\path/eliashberg/eigenvalue.f90}

\subsection{eigenvalue\_cods.f90}

Again, a procedure analogous to the preceding one is reimplemented for the CDOS
approximation, Eq.~\ref{CDOS linearized gap equation}. In this case, however, an
exact analytic expression for the renormalization is known which saves the
expenses of calling the self-energy solver. For convenience, not only the
respective eigenvector but also the memory allocation of all non-scalar
quantities is saved where possible, i.e. if the number of \name{Matsubara}
frequencies is not altered.

\lstinputlisting[language=f90]{\path/eliashberg/eigenvalue_cdos.f90}

\section{Continuation to the real axis}
\label{real_axis}

The following two modules are concerned with the analytic continuation of the
imaginary-axis results to the real axis by means of \name{Padé} approximants.
The first provides the approximant, the second applies it to the results.

\subsection{pade.f90}

This is the actual implementation of the algorithm given in Section~\ref{Pade
approximants}. Since there is no interest in the intermediate approximants, the
backward-recurrence method is used. For each set of values on the imaginary
axis, the coefficients are calculated only once; thereafter arbitrary real-axis
values can be requested until the module is reinitialized with a new data set.

\lstinputlisting[language=f90]{\path/real_axis/pade.f90}

\subsection{real\_axis.f90}

This subroutine applies the above algorithm to the previously calculated
self-energy. It may calculate both the measurable gap defined in
Eq.~\ref{measurable gap} via a fixed-point iteration and the real-axis
self-energy on an equidistantly discretized interval.

\lstinputlisting[language=f90]{\path/real_axis/real_axis.f90}

\section{I/O}

Subsequently, the input and output routines are presented. As stated at the
beginning of the chapter, the input is always via the command-line, whereas the
output is directed towards either the standard output or disk.

\subsection{load.f90}

This module is responsible for loading the command-line arguments, initializing
the coupling matrices and, possibly, reading the density of states from a given
file.

\lstinputlisting[language=f90]{\path/io/load.f90}

\subsection{store.f90}

If the results shall be processed further with the help of other programs, it is
desirable that no information is lost during the transfer process. Hence, the
unformatted output of the data in their internal representation is provided by
the following routine. It creates binary files which are structured by textual
statements which define the identifier, dimension and type of the following
data. This file format is understood by the Python interface module described in
Section~\ref{Python interface}.

\lstinputlisting[language=f90]{\path/io/store.f90}

\subsection{tell.f90}

If the results shall be displayed in a human-readable fashion, the following
module comes into play. It formats the results considering a possibly defined
edit descriptor and prints them to the standard output. If the storage on disk
is preferred, the output can be redirected via the command line like this:
%
\begin{quote}
    \verb|$ ebmb T=10 lambda=1 muStar=0.1 > self-energy.txt|
\end{quote}

\lstinputlisting[language=f90]{\path/io/tell.f90}

\section{Programs}

At this point all modules and their corresponding subroutines and functions have
been presented. Yet to be discussed are the actual executable programs by which
they are used, namely \code{ebmb}, \code{critical} and \code{tc}.

\subsection{ebmb.f90}

The workflow of \code{ebmb} is simple: load the parameters, call the desired
self-energy solver, optionally continue the results to the real axis and output
them via the desired channels.

\lstinputlisting[language=f90]{\path/programs/ebmb.f90}

\subsubsection{Example of application}

Below, the usage of \code{ebmb} on the command line is exemplified:
%
\begin{verbatim}
$ ebmb T=10 lambda=1 muStar=0.1 resolution=300
imaginary-axis solution [66]:

        omega/eV               Z        Delta/eV
________________________________________________
  0.002707214062  1.967226661541  0.002590550215
  0.008121642186  1.927225217112  0.002403912680
  0.013536070311  1.861850034908  0.002090211042
   . . .           . . .           . . .
  0.284257476526  1.072524734721 -0.000838949545
  0.289671904651  1.064021558735 -0.000840387592
  0.295086332775  1.054374032052 -0.000841668334

constant Coulomb contribution (eV):

 -0.000868488361

real-axis solution:

        omega/eV           Re[Z]           Im[Z]    Re[Delta]/eV    Im[Delta]/eV
________________________________________________________________________________
  0.000000000000  1.972750967345  0.000000000000  0.002615530353  0.000000000000
  0.001003344482  1.973490139549  0.000055160511  0.002619009158 -0.000000029469
  0.002006688963  1.975724185412  0.000092322544  0.002629488374 -0.000000045184
   . . .           . . .           . . .           . . .           . . .
  0.297993311037  1.002267219390  0.105412692473 -0.000896723088 -0.000007474029
  0.298996655518  1.002244792387  0.105059048188 -0.000896528744 -0.000007407687
  0.300000000000  1.002222640329  0.104707774935 -0.000896336488 -0.000007342091
\end{verbatim}

\subsection{critical.f90}

\code{critical} does not only call the appropriate \name{Eliashberg} eigenvalue
solver but also performs individual tasks, namely the identification of the
variable parameter and its optimization via the bisection method.

\lstinputlisting[language=f90]{\path/programs/critical.f90}

\subsubsection{Examples of application}

By default, the critical temperature is determined:
%
\begin{quote}
    \verb|$ critical lambda=2,0.1,0.1,1| \\
    \verb|47.261724151225|
\end{quote}
%
In a second step one could determine an effective scalar coupling constant:
%
\begin{quote}
    \verb|$ critical T=47.261724151225 lambda=-1| \\
    \verb|1.897134874960|
\end{quote}

\subsection{tc.f90}

Finally, a program is presented which determines the critical temperatures for
all bands separately via the bisection method. Since this requires the
calculation of the full self-energy at each step of the iteration, the use of
\code{critical} should be preferred, if possible.

\lstinputlisting[language=f90]{\path/programs/tc.f90}

\section{User manual}

On the following pages, the user manual for the software package is displayed.

\foreach \page in {1, ..., 5} {
    \begin{figure}
        \includegraphics[page=\page, width=\textwidth, trim=35mm 35mm 35mm 35mm]
            {\path/manual/ebmb.pdf}
    \end{figure}
    }
