%!TEX root = ../thesis.tex

\chapter{\name{Fourier} analysis}

Whenever there is translational symmetry, no matter if with respect to space or
time, it is instructive to explore the space of the corresponding conserved
quantity, momentum or energy. This applies especially to solid state physics,
where one often has spatial periodicity in addition to conservation of energy.

The mathematical tools to switch between such related representations are given
by the \name{Fourier} analysis. Despite its prominence the different
transformations for the possible combinations of discrete and continuous domains
shall be briefly presented, together with the underlying orthogonality
relations.

\section{Discrete \textsc{Fourier} transform}

The \emph{discrete \textsc{Fourier} transform} and its inverse read
%
\begin{align*}
    \hat y_\nu &= \frac 1 {\sqrt N}
    \smashoperator{\sum_{n = n_0}^{n_0 + N - 1}}
    \E^{2 \pi \I n \nu / N} y_n = \hat y_{\nu + N},
    \\
    \quad y_n &= \frac 1 {\sqrt N}
    \smashoperator{\sum_{\nu = \nu_0}^{\nu_0 + N - 1}}
    \E^{-2 \pi \I n \nu / N} \hat y_\nu = y_{n + N},
\end{align*}
%
where $\hat y_\nu$ and $y_n \in \mathds C$ are defined for all $\nu$ and $n \in
\mathds Z$ and recur with period $N \in \mathds N^+$ making the lower bounds
$\nu_0$ and $n_0 \in \mathds Z$ arbitrary. Invertibility is ensured by the
orthogonality relation
%
\begin{equation*}
    \smashoperator{\sum_{n = n_0}^{n_0 + N - 1}} \E^{2 \pi \I n \nu / N} =
    \sum_{n \in \mathds Z} \delta_{\nu, n N},
\end{equation*}
%
which is proven recognizing the partial sum of a geometrical series:
%
\begin{equation*}
    \sum_{n = 0}^{N - 1} \E^{2 \pi \I n \nu / N} =
    \begin{cases}
        \sum_{n = 0}^{N - 1} 1 = N
            & \text{for $\frac \nu N \in \mathds Z$,} \\
        \frac{\E^{2 \pi \I \nu} - 1}{\E^{2 \pi \I n \nu / N} - 1} = 0
            & \text{otherwise.}
    \end{cases}
\end{equation*}

\section{\textsc{Fourier} series}

The \emph{\textsc{Fourier} series} and its coefficients read
%
\begin{align*}
    y(t) &= \sum_{n \in \mathds Z} \E^{2 \pi \I n t / T} y_n = y(t + T), \\
    y_n &= \frac 1 T \int \D t \, \E^{-2 \pi \I n t / T} y(t),
\end{align*}
%
where $y(t) \in \mathds C$ is defined for all $t \in \mathds R$ and periodic
with period $T \in \mathds R^+$ making the lower limit $t_0 \in \mathds R$
arbitrary and $y_n \in \mathds C$ is defined for all $n \in \mathds Z$. There
are two orthogonality relations,
%
\begin{align*}
    \int \from{t_0} \till{t_0 + T} \D t \, \E^{2 \pi \I n t T}
    &= T \delta_{n, 0},
    \\
    \sum_{n \in \mathds Z} \E^{2 \pi \I n t / T}
    &= T \sum_{n \in \mathds Z} \delta(t - n T).
\end{align*}
%
The second equation is just mentioned for completeness since the left-hand side
is the \name{Fourier} series of the right-hand side. The first equation is
proven by simply carrying out the integral:
%
\begin{equation*}
    \int \from 0 \till T \D t \, \E^{2 \pi \I n t / T} =
    \begin{cases}
        \int_0^ T \D t \, 1 = T
            & \text{for $n = 0$,} \\
        \frac{\E^{2 \pi \I n t / T}}{2 \pi \I n / T} \big|_{t = 0}^T = 0
            & \text{otherwise.}
    \end{cases}
\end{equation*}

\section{\textsc{Fourier} transform}

The \emph{\textsc{Fourier} transform} and its inverse read
%
\begin{align*}
    \hat y(f) &= \int \from{-\infty} \till \infty
    \D t \, \E^{2 \pi \I f t} y(t),
    \\
    y(t) &= \int \from{-\infty} \till \infty
    \D f \, \E^{-2 \pi \I f t} \hat y(f),
\end{align*}
%
where $\hat y(f)$ and $y(t) \in \mathds C$ are defined for all $f$ and $t \in
\mathds R$. The orthogonality relation is
%
\begin{equation*}
    \int \from{-\infty} \till \infty \D t \, \E^{2 \pi \I f t} = \delta(f).
\end{equation*}
%
With $\omega = 2 \pi f$ and $\eta \in \mathds R$ introduced to generate
convergence it is proven in three steps:
%
\begin{subequations}
    \begin{align}
        \label{int+ dt exp(i omega t)}
        \int \from 0 \till \infty \D t \, \E^{\I \omega t}
        &= \lim_{\eta \rightarrow 0}
            \int \from 0 \till \infty \D t \, \E^{\I \omega t} \E^{-\eta t}
        = \lim_{\eta \rightarrow 0}
            \frac{\E^{\I \omega t} \E^{-\eta t}}{\I \omega - \eta}
            \Big|_{\omega = 0}^\infty
        = \lim_{\eta \rightarrow 0} \frac 1 {\eta - \I \omega},
        = \frac \I {\omega + \I 0^+}
        \\
        \label{int- dt exp(i omega t)}
        \int \from{-\infty} \till 0 \D t \, \E^{\I \omega t}
        &= \lim_{\eta \rightarrow 0}
            \int \from{-\infty} \till 0 \D t \, \E^{\I \omega t} \E^{\eta t}
        = \lim_{\eta \rightarrow 0}
            \frac{\E^{\I \omega t} \E^{\eta t}}{\I \omega - \eta}
            \Big|_{\omega = -\infty}^0
        = \lim_{\eta \rightarrow 0} \frac 1 {\eta + \I \omega}
        = \frac{-\I}{\omega - \I 0^+},
        \\
        \label{int dt exp(i omega t)}
        \int \from{-\infty} \till \infty \D t \, \E^{\I \omega t}
        &= \lim_{\eta \rightarrow 0} \left[
            \frac 1 {\eta - \I \omega} + \frac 1 {\eta + \I \omega} \right]
        = \lim_{\eta \rightarrow 0} \frac{2 \eta}{\eta^2 + \omega^2}
        = 2 \pi \delta(\omega).
    \end{align}
\end{subequations}
