% !TEX root = ../thesis.tex

% GENERAL

\usepackage[margin=3cm, bindingoffset=1cm]{geometry}

\usepackage[utf8]{inputenc}

\usepackage[ngerman, english]{babel}

\usepackage[math]{iwona}
\usepackage[T1]{fontenc}

\usepackage[scaled=0.92]{inconsolata}

\usepackage{microtype, slantsc}

\usepackage[font=small, labelfont=bf]{caption, subcaption}

% MATH

\usepackage{mathtools, upgreek}
\usepackage[sans]{dsfont}

% GRAPHICS

\usepackage{tikz}

\usetikzlibrary{external}
\tikzexternalize[prefix=tikzout/, only named, export=true]

\usetikzlibrary{decorations.markings}

\pgfdeclaredecoration{waves}{start}{
    \state{start}[width=0pt, next state=wave, persistent precomputation={
        \pgfmathsetlengthmacro\pgfdecorationsegmentlength{
             \pgfdecoratedpathlength / int(2
                * \pgfdecoratedpathlength
                / \pgfdecorationsegmentlength
                )
            }
        }] \relax
    \state{wave}[width=\pgfdecorationsegmentlength, persistent precomputation={
        \pgfmathsetlengthmacro\pgfdecorationsegmentamplitude{
            -\pgfdecorationsegmentamplitude
            }
        }]{
        \pgfpathsine{\pgfpoint
            {\pgfdecorationsegmentlength/2}
            {\pgfdecorationsegmentamplitude/2}
            }
        \pgfpathcosine{\pgfpoint
            {\pgfdecorationsegmentlength/2}
            {\pgfdecorationsegmentamplitude/-2}
            }
        }
    \state{final}{\pgfmoveto\pgfpointdecoratedpathlast}
    }

\def\xshift{0.525pt + 0.9625\pgflinewidth}

\tikzset{forward/.style={
    thick,
    decoration={
        markings,
        mark=at position 1/2 with {\arrow[xshift=\xshift]>},
        },
    preaction={decorate},
    }}

\tikzset{backward/.style={
    thick,
    decoration={
        markings,
        mark=at position 1/2 with {\arrow[xshift=\xshift]<},
        },
    preaction={decorate},
    }}

\tikzset{inward/.style={
    thick,
    decoration={
        markings,
        mark=at position 1/3 with {\arrow[xshift=\xshift]>},
        mark=at position 2/3 with {\arrow[xshift=\xshift]<},
        },
    preaction={decorate},
    }}

\tikzset{outward/.style={
    thick,
    decoration={
        markings,
        mark=at position 1/3 with {\arrow[xshift=\xshift]<},
        mark=at position 2/3 with {\arrow[xshift=\xshift]>},
        },
    preaction={decorate},
    }}

\tikzset{electron/.style={forward}}

\tikzset{phonon/.style={
    thick,
    decoration={waves, amplitude=2mm/3, segment length=2mm},
    decorate,
    }}

\definecolor{blockcolor}{gray}{0.9}


% HEADERS AND FOOTERS

\usepackage{fancyhdr}

\pagestyle{fancy}

\fancyhf{}

\def\chaptermark#1{\markboth
    {\thechapter\quad\MakeUppercase{#1}}
    {\MakeUppercase{#1}\quad\thechapter}}

\def\sectionmark#1{\markright{
    \MakeUppercase{#1}\quad\thesection}}

\fancyhead[OL]\leftmark
\fancyhead[ER]\rightmark
\fancyhead[OR]\thepage
\fancyhead[EL]\thepage

\def\headrulewidth{0.4pt}
\def\footrulewidth{0pt}


% SOURCE CODE LISTINGS

\usepackage{listings}

\lstdefinelanguage{f90}{
	keywords = {
        abstract, allocatable, allocate, call, case, character, class, close,
        complex, contains, default, do, elemental, else, elsewhere, end,
        function, if, implicit, integer, intent, interface, logical, module,
        only, open, operator, optional, parameter, pass, pointer, print,
        private, procedure, program, public, read, real, save, select, stop,
        subroutine, then, type, use, where, while, write
        },
	comment = [l]{!},
	string = [d]',
	morestring = [d]",
	sensitive = false,
    }

\lstdefinelanguage{py}{
    keywords = {
        and, as, assert, break, class, continue, def, del, elif, else, except,
        exec, finally, for, from, global, if, import, in, is, lambda, not, or,
        pass, print, raise, return, try, while, with, yield
        },
    comment = [l]{\#},
    string = [b]',
    morestring = [b]",
    sensitive = true,
    }

\definecolor     {basiccolor}{rgb}{0.1, 0.3, 0.5}
\definecolor   {keywordcolor}{rgb}{0.0, 0.4, 0.8}
\definecolor{identifiercolor}{rgb}{0.2, 0.2, 0.2}
\definecolor   {commentcolor}{rgb}{0.5, 0.5, 0.5}
\definecolor    {numbercolor}{rgb}{0.5, 0.5, 0.5}

\lstset{
	basicstyle = \color{basiccolor} \ttfamily \small,
	keywordstyle = \color{keywordcolor} \bfseries,
	identifierstyle = \color{identifiercolor},
	commentstyle = \color{commentcolor},
	numberstyle = \color{numbercolor} \tiny,
    %
	showstringspaces = false,
	%
	numbers = left,
    %
	numbersep = 3mm,
	aboveskip = 4mm,
	belowskip = 4mm,
    }

% BIBLIOGRAPHY

\usepackage[backend=bibtex8, firstinits=true, sorting=none]{biblatex}

\bibliography{references}

\let\mkbibnamelast\textsc
\def\labelnamepunct{\addcolon\space}
\def\multicitedelim{\addsemicolon\space}

\DeclareFieldFormat[article]{volume}{\mkbibbold{#1}}

\DeclareCiteCommand\barecite{}{\usebibmacro{citeindex}\usebibmacro{cite}}{}{}

\DeclareBibliographyDriver{incollection}{%
    \usebibmacro{bibindex}%
    \usebibmacro{begentry}%
    \usebibmacro{author/translator+others}%
    \setunit\labelnamepunct\newblock
    \usebibmacro{title}%
    \newunit\newblock
    \usebibmacro{in:}%
    Ref.~\barecite{\thefield{booktitle}}%
    \setunit\bibpagespunct
    \usebibmacro{chapter+pages}%
    \newunit\newblock
    \usebibmacro{doi+eprint+url}%
    \setunit\bibpagerefpunct\newblock
    \usebibmacro{pageref}%
    \usebibmacro{finentry}%
    }

% HYPERLINKS

\definecolor{linkcolor}{rgb}{0.5, 0, 0}

\usepackage[colorlinks, allcolors=linkcolor]{hyperref} % ocgcolorlinks

\pdfstringdefDisableCommands{\let\name\relax}
