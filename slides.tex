% !TEX TS-options = -shell-escape

\documentclass[10pt]{beamer}

\mode<presentation>

\usetheme[width=2cm]{Goettingen}

% !TEX root = ../thesis.tex

\makeatletter

% GENERAL

\@ifclassloaded{beamer}{}
    {\usepackage[margin=3cm, bindingoffset=1cm]{geometry}}

\usepackage[utf8]{inputenc}

\usepackage[ngerman, english]{babel}

\usepackage[math]{iwona}
\usepackage[T1]{fontenc}

\usepackage[scaled=0.92]{inconsolata}

\usepackage{microtype, slantsc}

\@ifclassloaded{beamer}{}
    {\usepackage[font=small, labelfont=bf]{caption}}

\@ifclassloaded{beamer}{}
    {\usepackage[font=normalsize, labelfont=bf]{subcaption}}

% MATH

\usepackage{mathtools, upgreek}
\usepackage[sans]{dsfont}

% GRAPHICS

\usepackage{tikz}

\usetikzlibrary{external}
\tikzexternalize[prefix=tikzout/, only named, export=true]

\usetikzlibrary{decorations.markings}

\pgfdeclaredecoration{waves}{start}{
    \state{start}[width=0pt, next state=wave, persistent precomputation={
        \pgfmathsetlengthmacro\pgfdecorationsegmentlength{
             \pgfdecoratedpathlength / int(2
                * \pgfdecoratedpathlength
                / \pgfdecorationsegmentlength
                )
            }
        }] \relax
    \state{wave}[width=\pgfdecorationsegmentlength, persistent precomputation={
        \pgfmathsetlengthmacro\pgfdecorationsegmentamplitude{
            -\pgfdecorationsegmentamplitude
            }
        }]{
        \pgfpathsine{\pgfpoint
            {\pgfdecorationsegmentlength/2}
            {\pgfdecorationsegmentamplitude/2}
            }
        \pgfpathcosine{\pgfpoint
            {\pgfdecorationsegmentlength/2}
            {\pgfdecorationsegmentamplitude/-2}
            }
        }
    \state{final}{\pgfmoveto\pgfpointdecoratedpathlast}
    }

\def\xshift{0.525pt + 0.9625\pgflinewidth}

\tikzset{forward/.style={
    thick,
    decoration={
        markings,
        mark=at position 1/2 with {\arrow[xshift=\xshift]>},
        },
    preaction={decorate},
    }}

\tikzset{backward/.style={
    thick,
    decoration={
        markings,
        mark=at position 1/2 with {\arrow[xshift=\xshift]<},
        },
    preaction={decorate},
    }}

\tikzset{inward/.style={
    thick,
    decoration={
        markings,
        mark=at position 1/3 with {\arrow[xshift=\xshift]>},
        mark=at position 2/3 with {\arrow[xshift=\xshift]<},
        },
    preaction={decorate},
    }}

\tikzset{outward/.style={
    thick,
    decoration={
        markings,
        mark=at position 1/3 with {\arrow[xshift=\xshift]<},
        mark=at position 2/3 with {\arrow[xshift=\xshift]>},
        },
    preaction={decorate},
    }}

\tikzset{electron/.style={forward}}

\tikzset{phonon/.style={
    thick,
    decoration={waves, amplitude=2mm/3, segment length=2mm},
    decorate,
    }}

\definecolor{blockcolor}{gray}{0.9}


% HEADERS AND FOOTERS

\usepackage{fancyhdr}

\@ifclassloaded{beamer}{}
    {\pagestyle{fancy}}

\fancyhf{}

\def\chaptermark#1{\markboth
    {\thechapter\quad\MakeUppercase{#1}}
    {\MakeUppercase{#1}\quad\thechapter}}

\def\sectionmark#1{\markright{
    \MakeUppercase{#1}\quad\thesection}}

\fancyhead[OL]\leftmark
\fancyhead[ER]\rightmark
\fancyhead[OR]\thepage
\fancyhead[EL]\thepage

\def\headrulewidth{0.4pt}
\def\footrulewidth{0pt}


% SOURCE CODE LISTINGS

\usepackage{listings}

\lstdefinelanguage{f90}{
    keywords = {
        abstract, allocatable, allocate, call, case, character, class, close,
        complex, contains, default, do, elemental, else, elsewhere, end,
        function, if, implicit, integer, intent, interface, logical, module,
        only, open, operator, optional, parameter, pass, pointer, print,
        private, procedure, program, public, read, real, save, select, stop,
        subroutine, then, type, use, where, while, write
        },
    comment = [l]{!},
    string = [d]',
    morestring = [d]",
    sensitive = false,
    }

\lstdefinelanguage{py}{
    keywords = {
        and, as, assert, break, class, continue, def, del, elif, else, except,
        exec, finally, for, from, global, if, import, in, is, lambda, not, or,
        pass, print, raise, return, try, while, with, yield
        },
    comment = [l]{\#},
    string = [b]',
    morestring = [b]",
    sensitive = true,
    }

\definecolor     {basiccolor}{rgb}{0.1, 0.3, 0.5}
\definecolor   {keywordcolor}{rgb}{0.0, 0.4, 0.8}
\definecolor{identifiercolor}{rgb}{0.2, 0.2, 0.2}
\definecolor   {commentcolor}{rgb}{0.5, 0.5, 0.5}
\definecolor    {numbercolor}{rgb}{0.5, 0.5, 0.5}

\lstset{
    basicstyle = \color{basiccolor} \ttfamily \small,
    keywordstyle = \color{keywordcolor} \bfseries,
    identifierstyle = \color{identifiercolor},
    commentstyle = \color{commentcolor},
    numberstyle = \color{numbercolor} \tiny,
    %
    showstringspaces = false,
    %
    numbers = left,
    %
    numbersep = 3mm,
    aboveskip = 4mm,
    belowskip = 4mm,
    }

% BIBLIOGRAPHY

\usepackage[backend=bibtex8, firstinits=true, sorting=none]{biblatex}

\bibliography{references}

\let\mkbibnamelast\textsc
\def\labelnamepunct{\addcolon\space}
\def\multicitedelim{\addsemicolon\space}

\DeclareFieldFormat[article]{volume}{\mkbibbold{#1}}

\DeclareCiteCommand\barecite{}{\usebibmacro{citeindex}\usebibmacro{cite}}{}{}

\DeclareBibliographyDriver{incollection}{%
    \usebibmacro{bibindex}%
    \usebibmacro{begentry}%
    \usebibmacro{author/translator+others}%
    \setunit\labelnamepunct\newblock
    \usebibmacro{title}%
    \newunit\newblock
    \usebibmacro{in:}%
    \printtext{Ref.~\barecite{\thefield{booktitle}}}%
    \setunit\bibpagespunct
    \usebibmacro{chapter+pages}%
    \newunit\newblock
    \usebibmacro{doi+eprint+url}%
    \setunit\bibpagerefpunct\newblock
    \usebibmacro{pageref}%
    \usebibmacro{finentry}%
    }

% HYPERLINKS

\definecolor{linkcolor}{rgb}{0.5, 0, 0}

\@ifclassloaded{beamer}{}
    {\usepackage[pdfa, colorlinks, allcolors=linkcolor]{hyperref}} % ocgcolorlinks

\pdfstringdefDisableCommands{\let\name\relax}

\makeatother

% !TEX root = ../thesis.tex

% FLOATS

\setcounter{topnumber}{1}
\setcounter{bottomnumber}{1}

\def\textfraction{0.1}
\def\topfraction{0.9}
\def\bottomfraction{0.9}
\def\floatpagefraction{0.9}

% PENALTIES

\clubpenalty=10000
\widowpenalty=10000
\displaywidowpenalty=10000
\binoppenalty=\maxdimen
\relpenalty=\maxdimen

% !TEX root = ../thesis.tex


% TEXT

\def\centered#1#2{\hspace{0pt}\llap{#1}\hspace{1pc}\rlap{#2}}

\let\name\textsc

\def\q#1{\textit{`#1'}}
\def\qq#1{\textit{``#1''}}


% MATH

\let\op\mathrm
\let\unit\mathrm
\let\vec\boldsymbol

\let\epsilon\varepsilon
\let\Delta\varDelta
\let\Gamma\varGamma
\let\Lambda\varLambda
\let\Omega\varOmega
\let\Sigma\varSigma
\let\Theta\varTheta
\let\Pi\varPi
\let\Xi\varXi
\let\upphi\upvarphi

\let\up\uparrow
\let\down\downarrow

\def\D{\mathrm d}
\def\I{\mathrm i}
\def\E{\mathrm e}

\let\Re\relax
\let\Im\relax

\DeclareMathOperator\Re{Re}
\DeclareMathOperator\Im{Im}
\DeclareMathOperator\Tr{Tr}
\DeclareMathOperator\Cov{Cov}
\DeclareMathOperator\sgn{sgn}

\def\super#1{\sp{\text{#1}}}
\def\sub#1{\sb{\text{#1}}}
\def\from#1{\sb{\mathrlap{#1}}}
\def\till#1{\sp{\mathrlap{#1}}}

\def\hyper#1#2{\overbrace{#1}^{\displaystyle\mathclap{{}#2}}}
\def\hypo#1#2{\underbrace{#1}_{\displaystyle\mathclap{{}#2}}}

\def\bm$#1${$\boldsymbol{#1}$}


% INCLUDE SYMBOLS FOR FEYNMAN DIAGRAMS

\def\feyn#1{\includegraphics[width=5mm,valign=c]{fig/#1.pdf}}


% QUANTUM MECHANICS

\def\bra#1{\langle#1|}
\def\ket#1{|#1\rangle}
\def\bracket#1#2{\langle#1|#2\rangle}
\def\av#1{\langle#1\rangle}


\usetikzlibrary{shadows.blur}

\tikzset{remembered/.style={
    remember picture, inner sep=0, outer sep=0, baseline, anchor=base}}

\tikzset{infobox/.style={shape=rectangle,
    rounded corners, draw, fill=white, opacity=0.9, blur shadow}}

\defbeamertemplate{footline}{info}[1]{%
    \hfill\smash{\color{titleblue}{\footnotesize #1}}\hfill%
    \hspace{2cm}\llap{\color{white}{\scriptsize%
        \insertframenumber\,/\,\inserttotalframenumber\hspace*{1mm}}}%
    \vspace{1mm}%
    }

\setbeamertemplate{footline}[info]{}
\setbeamertemplate{caption}[numbered]
\setbeamertemplate{navigation symbols}{}

\definecolor{titleblue}{rgb}{0.2, 0.2, 0.7}

\usefonttheme{professionalfonts}

\setbeamerfont        {title}{series=\bfseries}
\setbeamerfont     {subtitle}{series=\normalfont}
\setbeamerfont   {frametitle}{series=\bfseries}
\setbeamerfont{framesubtitle}{series=\normalfont, size=\normalsize}

\setbeamercolor     {title in sidebar}{fg=white}
\setbeamercolor    {author in sidebar}{fg=white}
\setbeamercolor   {section in sidebar}{fg=white, bg=titleblue!75}
\setbeamercolor{subsection in sidebar}{fg=white, bg=titleblue!75}

\setbeamercolor   {section in sidebar shaded}{fg=titleblue!25}
\setbeamercolor{subsection in sidebar shaded}{fg=titleblue!25}

\setbeamerfont     {title in sidebar}{size=\fontsize{6.0pt}{7.2pt}\selectfont,
                                    series=\bfseries}
\setbeamerfont    {author in sidebar}{size=\fontsize{5.5pt}{6.6pt}\selectfont}
\setbeamerfont   {section in sidebar}{size=\fontsize{5.5pt}{6.6pt}\selectfont,
                                    series=\bfseries}
\setbeamerfont{subsection in sidebar}{size=\fontsize{5.5pt}{6.6pt}\selectfont}

\setbeamerfont{section in toc}{series=\bfseries}

\setbeamertemplate{sidebar canvas right}
    {\hspace{-21mm}\input{results/sidebar.sl}}

\title[%
    \hspace{-5mm} On the scope of \\
    \hspace{-5mm} \name{McMillan}'s formula%
    ]
    {On the scope of \name{McMillan}'s formula}

\subtitle{\medskip \bfseries \small Presentation of master's thesis}

\AtBeginDocument{
    \author[\hspace{-5mm} Jan Berges]{%
        \small
        \begin{tabular}{r l}
            \it by
                & Jan Berges \\[2mm]
            \it Supervisors
                & Prof. Dr. Tim Wehling \\
                & Prof. Dr. Gerd Czycholl
        \end{tabular}%
        }
    }

\institute{\footnotesize%
    Institut für Theoretische Physik \\
    \emph{Electronic Structure and Correlated Nanosystems}%
    }

\titlegraphic{%
    \includegraphics[height=0.8cm]{figures/uni.pdf}\hfill%
    \includegraphics[height=1.6cm]{figures/itp.pdf}\vspace{-15mm}%
    }

\date{\small \today}

%\includeonlyframes{McMillan}

\begin{document}
	\begin{frame}
		\titlepage
	\end{frame}

	\begin{frame}{Contents}
		\tableofcontents
	\end{frame}

	\section{Motivation}

	\begin{frame}{Motivation}
	\end{frame}

    \section{Many-body physics}

    \subsection[El.-ph. interaction]{Electron-phonon interaction}

	\begin{frame}[label=interaction]{Electron-phonon interaction}
        \begin{figure}
            \small
            % !TEX root = ../thesis.tex
%
\tikzsetnextfilename{hartree}
%
\begin{tikzpicture}
    \coordinate (i) at (0, 0);
    \coordinate (2) at (2, 0);
    \coordinate (1) at (2, 4/3);
    \coordinate (o) at (4, 0);

    \draw [phonon] (2) -- node [right] {$0$} (1);

    \draw [electron] (i) -- node [below] {$\vec k \sigma$} (2);
    \draw [electron] (2) -- node [below] {$\vec k \sigma$} (o);

    \draw [electron] (1) arc (-90:270:5mm)
        node [yshift=5mm] {$\vec k' \sigma'$};

    \foreach \point in {1, 2} \fill (\point) circle (1pt);

    \node [below] at (i) {$0$};
    \node [above left] at (2) {$\tau''$};
    \node [below left] at (1) {$\tau'$};
    \node [below] at (o) {$\tau$};

    \useasboundingbox ([yshift=1mm] current bounding box.north east);
\end{tikzpicture}%

            % !TEX root = ../thesis.tex
%
\tikzsetnextfilename{fock}
%
\begin{tikzpicture}[bend angle=45]
    \coordinate (i) at (0, 0);
    \coordinate (2) at (4/3, 4/3);
    \coordinate (t) at (6/3, 6/3);
    \coordinate (1) at (8/3, 4/3);
    \coordinate (o) at (4, 0);

    \draw [phonon] (2) -- node [below] {$\vec q$} (1);

    \draw [electron] (i) to [bend right]
        node [above left] {$\vec k \sigma$} (2);
    \draw [electron] (2) to [bend left] (t)
        node [above] {$\vec k - \vec q \sigma$} to [bend left] (1);
    \draw [electron] (1) to [bend right]
        node [above right] {$\vec k \sigma$} (o);

    \foreach \point in {1, 2} \fill (\point) circle (1pt);

    \node [below] at (i) {$0$};
    \node [left] at (2) {$\tau''$};
    \node [right] at (1) {$\tau'$};
    \node [below] at (o) {$\tau$};
\end{tikzpicture}%

            % !TEX root = ../thesis.tex
%
\tikzsetnextfilename{glasses}
%
\begin{tikzpicture}
    \coordinate (i) at (0, 0);
    \coordinate (2) at (4/3, 4/3);
    \coordinate (1) at (8/3, 4/3);
    \coordinate (o) at (4, 0);

    \draw [phonon] (2) -- node [below] {$0$} (1);

    \draw [electron] (i) -- node [below] {$\vec k \sigma$} (o);

    \draw [electron] (2) arc (0:360:5mm)
        node [xshift=-5mm] {$\vec k'' \sigma''$};
    \draw [electron] (1) arc (180:540:0.5)
        node [xshift=5mm] {$\vec k' \sigma'$};

    \foreach \point in {1, 2} \fill (\point) circle (1pt);

    \node [below] at (i) {$0$};
    \node [above right] at (2) {$\tau''$};
    \node [above left] at (1) {$\tau'$};
    \node [below] at (o) {$\tau$};
\end{tikzpicture}%

            \input{diagrams/porthole.tex}
            \caption{Electron renormalization processes with one phonon}
        \end{figure}
	\end{frame}

    \subsection{\name{Hedin}'s equations}

    \setbeamertemplate{footline}[info]{%
        L. \name{Hedin}, Phys. Rev. \textbf{139}, A796 (1965)}

	\begin{frame}[label=Hedin]{\name{Hedin}'s equations}
        \begin{gather*}
            %!TEX root = ../thesis.tex
%
\tikzsetnextfilename{electron}
%
\begin{tikzpicture}[baseline]
    \pgfmathsetmacro\r{sqrt(sqrt(3)/pi)/2}

    \draw [electron, double] (0, 0) -- (1, 0);

    \node at (1.5, 0) {$=$};

    \draw [electron] (2, 0) -- (3, 0);

    \node at (3.5, 0) {$+$};

    \coordinate (i) at (5-\r, 0);
    \coordinate (o) at (5+\r, 0);

    \draw [electron] (4, 0) -- (i);
    \draw [electron, double] (o) -- (6, 0);

    \draw [fill=black!10!white] (5, 0) circle [radius=\r] node {$\Sigma$};

    \foreach \point in {i, o} \fill (\point) circle (1.5pt);
\end{tikzpicture}
 \\
            \only<1>{%!TEX root = ../thesis.tex
%
\begin{tikzpicture}[baseline]
    \pgfmathsetmacro\r{sqrt(sqrt(3)/pi)/2}

    \coordinate (i) at (1-2*\r, 0);

    \draw [fill=black!10!white] (1-\r, 0) circle [radius=\r] node {$\Sigma$};

    \coordinate (o) at (1, 0);

    \node at (2, 0) {$=$};

    \coordinate (0) at (3, 0.0);
    \coordinate (1) at (3, 0.5);

    \node at (4, 0) {$+$};

    \coordinate (I) at (5.0, 0);
    \coordinate (X) at (5.5, {sqrt(3)/2});
    \coordinate (R) at (6.0, 0);
    \coordinate (O) at (7.0, 0);

    \begin{scope}[gray]
        \foreach \point in {i, 0, I}
            \draw [electron] (\point) +(180:5mm) -- (\point);
    
        \foreach \point in {o, 0, O}
            \draw [electron] (\point) -- +(0:5mm);
    \end{scope}

    \draw [phonon] (0) -- (1);
    \draw [electron, double] (1) arc (-90:270:2.5mm);

    \draw [phonon, double] (X) -- (O);
    \draw [electron, double] (R) -- (O);

    \draw [fill=black!10!white] (X) -- (I) -- (R) -- (X);

    \foreach \point in {i, o, 0, 1, I, X, R, O} \fill (\point) circle (1.5pt);

    \node [below=3mm] at (X) {$\Gamma$};
\end{tikzpicture}
}%
            \only<2>{\input{diagrams/self-energy-gw.tex}} \\
            %!TEX root = ../thesis.tex
%
\tikzsetnextfilename{phonon}
%
\begin{tikzpicture}[baseline]
    \pgfmathsetmacro\r{sqrt(sqrt(3)/pi)/2}

    \draw [phonon, double] (0, -0.5) -- (0, 0.5);

    \node at (0.5, 0) {$=$};

    \draw [phonon] (1, -0.5) -- (1, 0.5);

    \node at (1.5, 0) {$+$};

    \coordinate (T) at (2+\r, +\r);
    \coordinate (B) at (2+\r, -\r);

    \draw [phonon, double] (T) -- +(+90:1-\r);
    \draw [phonon] (B) -- +(-90:1-\r);

    \draw [fill=blockcolor] (2+\r, 0) circle [radius=\r] node {$\Pi$};

    \foreach \point in {T, B} \fill (\point) circle (1.5pt);
\end{tikzpicture}%
 \hspace{14mm}
            %!TEX root = ../thesis.tex
%
\begin{tikzpicture}[baseline]
    \pgfmathsetmacro\r{sqrt(sqrt(3)/pi)/2}
    %
    \coordinate (t) at (1-\r, +\r);
    \coordinate (b) at (1-\r, -\r);
    %
    \draw [fill=black!10!white] (1-\r, 0) circle [radius=\r] node {$\Pi$};
    %
    \node at (1.5, 0) {$=$};
    %
    \coordinate (I) at (2.0, 0);
    \coordinate (T) at (2.5, +{sqrt(3)/2});
    \coordinate (B) at (2.5, -{sqrt(3)/2});
    \coordinate (O) at (3.0, 0);
    %
    \foreach \point in {t, T}
        \draw [phonon, gray] (\point) -- +(+90:5mm);
    %
    \foreach \point in {b, B}
        \draw [phonon, gray] (\point) -- +(-90:5mm);
    %
    \draw [electron, double] (B) -- (I);
    \draw [electron, double] (O) -- (B);
    %
    \draw [fill=black!10!white] (T) -- (I) -- (O) -- (T);
    %
    \foreach \point in {t, b, I, T, B, O} \fill (\point) circle (1.5pt);
    %
    \node [below=3mm] at (T) {$\Gamma$};
\end{tikzpicture}%
 \\[-7mm]
            \only<1>{%!TEX root = ../thesis.tex
%
\begin{tikzpicture}[baseline]
    \coordinate (i) at (0.0, -{sqrt(3)/4});
    \coordinate (x) at (0.5, +{sqrt(3)/4});
    \coordinate (o) at (1.0, -{sqrt(3)/4});
    %
    \node at (1.5, 0) {$=$};
    %
    \coordinate (0) at (2.5, 0);
    %
    \node at (3.5, 0) {$+$};
    %
    \coordinate (I) at (4.00, -{sqrt(3)/2});
    \coordinate (l) at (4.25, -{sqrt(3)/4});
    \coordinate (L) at (4.50, 0);
    \coordinate (X) at (5.00, +{sqrt(3)/2});
    \coordinate (R) at (5.50, 0);=
    \coordinate (r) at (5.75, -{sqrt(3)/4});
    \coordinate (O) at (6.00, -{sqrt(3)/2});
    %
    \begin{scope}[gray]
        \foreach \point in {i, 0, I}
            \draw [backward] (\point) -- +(225:5mm);
        %
        \foreach \point in {x, 0, X}
            \draw [phonon] (\point) -- +(90:5mm);
        %
        \foreach \point in {o, 0, O}
            \draw [forward] (\point) -- +(315:5mm);
    \end{scope}
    %
    \draw [electron, double] (l) -- (L);
    \draw [electron, double] (R) -- (r);
    %
    \draw [fill=blockcolor]
        (x) -- (i) -- (o) -- (x)
        (X) -- (L) -- (R) -- (X)
        (I) -- (O) -- (r) -- (l) -- (I);
    %
    \foreach \point in {i, x, o, 0, I, l, L, X, R, r, O}
        \fill (\point) circle (1.5pt);
    %
    \foreach \point in {x, X}
        \node [below=3mm] at (\point) {$\Gamma$};
    %
    \node at (5, -{3*sqrt(3)/8}) {$\delta \Sigma / \delta G$};
\end{tikzpicture}%
}%
            \only<2>{% !TEX root = ../slides.tex
%
\tikzsetnextfilename{vertex-gw}
%
\begin{tikzpicture}[baseline]
    \coordinate (i) at (0.0, -{sqrt(3)/4});
    \coordinate (x) at (0.5, +{sqrt(3)/4});
    \coordinate (o) at (1.0, -{sqrt(3)/4});

    \node at (1.5, 0) {$\approx$};

    \coordinate (0) at (2.5, 0);

    \coordinate (I) at (4.00, -{sqrt(3)/2});
    \coordinate (X) at (5.00, +{sqrt(3)/2});
    \coordinate (O) at (6.00, -{sqrt(3)/2});

    \begin{scope}[gray]
        \foreach \point in {i, 0}
            \draw [backward] (\point) -- +(225:5mm);

        \foreach \point in {x, 0}
            \draw [phonon] (\point) -- +(90:5mm);

        \foreach \point in {o, 0}
            \draw [forward] (\point) -- +(315:5mm);
    \end{scope}

    \begin{scope}[opacity=0]
        \foreach \point in {I}
            \draw [backward] (\point) -- +(225:5mm);

        \foreach \point in {X}
            \draw [phonon] (\point) -- +(90:5mm);

        \foreach \point in {O}
            \draw [forward] (\point) -- +(315:5mm);
    \end{scope}

    \draw [fill=blockcolor] (x) -- (i) -- (o) -- (x);

    \foreach \point in {i, x, o, 0}
        \fill (\point) circle (1.5pt);

    \node [below=3mm] at (x) {$\Gamma$};
\end{tikzpicture}%
%
            \mathllap{\text{(GW\,/\,\name{Migdal})}}}
        \end{gather*}
	\end{frame}

    \subsection[Superconductivity?]{What about superconductivity?}

    \setbeamertemplate{footline}[info]{}

	\begin{frame}[label=superconductivity]{What about superconductivity?}
        \begin{itemize}
            \item \name{Dyson} equation
            %
            \vspace{-3mm}
            \begin{align*}
                \only<1>{% !TEX root = ../slides.tex
%
\tikzsetnextfilename{dyson-1}
%
\begin{tikzpicture}[scale=0.7]
    \useasboundingbox (0, -0.7) rectangle (11, 1.4);

    \foreach \r in {(2, 0.7), (4, 0.7)}
        \draw [backward] \r -- +(1, 0);

    \foreach \r in {(0, 0.7), (6, 0.7)}
        \draw [backward, double] \r -- +(1, 0);

    \draw [backward, double] (5, 0.7) arc (180:0:0.5);

    \draw [phonon, double] (5, 0.7) -- +(1, 0);

    \foreach \x in {5, 6}
        \fill (\x, 0.7) circle (0.075);

    \foreach \r in {(1.5, 0.7)} \node at \r {$=$};

    \node at (3.5, 0.7) {$+$};

    \useasboundingbox ([yshift=1mm] current bounding box.north east);
\end{tikzpicture}%
}%
                \only<2>{% !TEX root = ../slides.tex
%
\tikzsetnextfilename{dyson-2}
%
\begin{tikzpicture}[scale=0.7]
    \useasboundingbox (0, -0.7) rectangle (11, 1.4);

    \foreach \r in {(4, 0), (8, 0), (2, 0.7), (4, 0.7), (8, 0.7)}
        \draw [backward] \r -- +(1, 0);

    \foreach \r in {(0, 0.7), (6, 0.7)}
        \draw [backward, double] \r -- +(1, 0);

    \foreach \r in {(0, 0.0), (6, 0.0)}
        \draw [outward, double] \r -- +(1, 0);

    \draw [ inward, double] (10, 0.7) -- +(1, 0);
    \draw [forward, double] (10, 0.0) -- +(1, 0);

    \draw [backward, double] (5, 0) arc (180:360:0.5);
    \draw [ outward, double] (9, 0) arc (180:360:0.5);
    \draw [ outward, double] (9, 0.7) arc (180:0:0.5);
    \draw [backward, double] (5, 0.7) arc (180:0:0.5);

    \foreach \x in {5, 9} {
        \foreach \y in {0.0, 0.7} {
            \draw [phonon, double] (\x, \y) -- +(1, 0);
            }
        }

    \foreach \x in {5, 6, 9, 10}
        \foreach \y in {0, 0.7}
            \fill (\x, \y) circle (0.075);

    \foreach \r in {(1.5, 0.7), (1.5, 0)} \node at \r {$=$};

    \foreach \r in {(3.5, 0.7), (7.5, 0.7), (7.5, 0)}
        \node at \r {$+$};

    \useasboundingbox ([yshift=1mm] current bounding box.north east);
\end{tikzpicture}%
}%
                \only<3>{\input{diagrams/dyson-3.tex}}
            \end{align*}
            %
            \item \name{Green} functions
            %
            \begin{itemize}
                \item electronic
                %
                \only<-2>{
                    \begin{align*}
                        \tikz [baseline=-2.5pt]
                            \draw [backward, double] (0, 0) -- (0.7, 0);
                        \quad &= \quad
                        \mathrlap{ -\av{ \op T \,
                            \op c_{\vec k \up}(\tau) \,
                            \op c_{\vec k \up}^+ } }
                        \hspace{6cm}
                        \\
                        \tikz [baseline=-2.5pt]
                            \draw [backward] (0, 0) -- (0.7, 0);
                        \quad &= \quad
                        -\av{ \op T \,
                            \op c_{\vec k \up}(\tau) \,
                            \op c_{\vec k \up}^+ }_0
                    \end{align*}
                    }
                %
                \only<3>{
                    \begin{align*}
                        \tikz [baseline=-2.5pt]
                            \draw [forward, double] (0, 0) -- (0.7, 0);
                        \quad &= \quad
                        \mathrlap{ -\av{ \op T \,
                            \op c_{-\vec k \down}^+(\tau) \,
                            \op c_{-\vec k \down} } }
                        \hspace{6cm}
                        \\
                        \tikz [baseline=-2.5pt]
                            \draw [forward] (0, 0) -- (0.7, 0);
                        \quad &= \quad
                        -\av{ \op T \,
                            \op c_{-\vec k \down}^+(\tau) \,
                            \op c_{-\vec k \down} }_0
                    \end{align*}
                    }
                %
                \item phononic
                %
                \begin{align*}
                    \tikz [baseline=-2.5pt]
                        \draw [phonon, double] (0, 0) -- (0.7, 0);
                    \quad &= \quad
                    \mathrlap{ -\av{ \op T \,
                        \upphi_{\vec q}(\tau) \,
                        \upphi_{\vec q}^+ } }
                    \hspace{6cm}
                    \\
                    \tikz [baseline=-2.5pt]
                        \draw [phonon] (0, 0) -- (0.7, 0);
                    \quad &= \quad
                    -\av{ \op T \,
                        \upphi_{\vec q}(\tau) \,
                        \upphi_{\vec q}^+ }_0
                    \quad \text{with} \quad
                    \upphi_{\vec q} = \op b_{\vec q} + \op b_{-\vec q}^+
                \end{align*}
                %
                \item<2-> anomalous
                %
                \only<-2>{
                    \begin{align*}
                        \tikz [baseline=-2.5pt]
                            \draw [outward, double] (0, 0) -- (0.7, 0);
                        \quad &= \quad
                        \mathrlap{ -\av{ \op T \,
                            \op c_{\vec k \up}(\tau) \,%
                            \vphantom{\op c_{\vec k \up}^+}%
                            \op c_{-\vec k \down} } }
                        \hspace{6cm}
                        \\
                        \tikz [baseline=-2.5pt]
                            \draw [outward] (0, 0) -- (0.7, 0);
                        \quad &= \quad
                        -\av{ \op T \,
                            \op c_{\vec k \up}(\tau) \,%
                            \vphantom{\op c_{\vec k \up}^+}%
                            \op c_{-\vec k \down} }_0
                        \quad = \quad 0
                    \end{align*}
                    }
                %
                \only<3>{
                    \begin{align*}
                        \tikz [baseline=-2.5pt]
                            \draw [inward, double] (0, 0) -- (0.7, 0);
                        \quad &= \quad
                        \mathrlap{ -\av{ \op T \,
                            \op c_{-\vec k \down}^+(\tau) \,
                            \op c_{\vec k \up}^+ } }
                        \hspace{6cm}
                        \\
                        \tikz [baseline=-2.5pt]
                            \draw [inward] (0, 0) -- (0.7, 0);
                        \quad &= \quad
                        -\av{ \op T \,
                            \op c_{-\vec k \down}^+(\tau) \,
                            \op c_{\vec k \up}^+ }_0
                        \quad = \quad 0
                    \end{align*}
                    }
            \end{itemize}
        \end{itemize}
	\end{frame}

    \section{\name{Eliashberg} theory}

    \subsection{\name{Nambu} formalism}

    \setbeamertemplate{footline}[info]{%
        Y. \name{Nambu}, Phys. Rev. \textbf{117}, 648 (1960)}

    \begin{frame}[label=Nambu]{\name{Nambu} formalism}
        \begin{itemize}
            \item \name{Dyson} matrix equation
            %
            \begin{gather*}
                \vec G = \vec G_0 + \vec G_0 \vec \Sigma \vec G
                \quad \text{or} \quad
                \vec G^{-1} = \vec G_0^{-1} - \vec \Sigma \\
                % !TEX root = ../slides.tex
%
\tikzsetnextfilename{nambu-dyson-matrices}
%
\begin{tikzpicture}[yscale=-1]
    \foreach \r in {(0.8, 0.25), (3.8, 0.25), (6.8, 0.25)}
        \node at \r {$\Bigg[$};

    \foreach \r in {(2.5, 0.25), (5.5, 0.25), (8.5, 0.25)}
        \node at \r {$\Bigg]$};

    \draw [backward] (4.0, 0.00) -- +(0.5, 0);
    \draw [ forward] (4.8, 0.50) -- +(0.5, 0);

    \foreach \r in {(5.05, 0), (4.25, 0.5)}
        \node at \r {$0$};

    \draw [backward, double] (1.0, 0.0) -- +(0.5, 0);
    \draw [ outward, double] (1.8, 0.0) -- +(0.5, 0);
    \draw [  inward, double] (1.0, 0.5) -- +(0.5, 0);
    \draw [ forward, double] (1.8, 0.5) -- +(0.5, 0);

    \draw [backward, double] (7.0, 0.0) arc (180:360:2.5mm);
    \draw [ outward, double] (7.8, 0.0) arc (180:360:2.5mm);
    \draw [  inward, double] (7.0, 0.5) arc (180:360:2.5mm);
    \draw [ forward, double] (7.8, 0.5) arc (180:360:2.5mm);

    \foreach \x in {7, 7.8}
        \foreach \y in {0, 0.5}
            \draw [phonon, double] (\x, \y) -- +(0.5, 0);

    \foreach \x in {7, 7.5, 7.8, 8.3}
        \foreach \y in {0, 0.5}
            \fill (\x, \y) circle (1.5pt);

    \node at (0.25, 0.25) {$\vec G$};
    \node at (3.20, 0.25) {$\vec G_0$};
    \node at (6.25, 0.25) {$\vec \Sigma$};

    \foreach \x in {0.55, 3.55, 6.55}
        \node at (\x, 0.25) {$=$};

    %\node at (4.50, 0.25) {$+$};
\end{tikzpicture}%

            \end{gather*}
            %
            \item Linear combination of \name{Pauli} matrices
            %
            \begin{align*}
                \vec \Sigma_{\vec k}(\I \omega_n)
                &= \I \omega_n [1 -
                \tikz [remembered] \node (Z) {$Z_{\vec k}$};
                (\I \omega_n)] \vec \sigma_0 +
                \tikz [remembered] \node (phi) {$\phi_{\vec k}$};
                (\I \omega_n) \vec \sigma_1 +
                \tikz [remembered] \node (chi) {$\chi_{\vec k}$};
                (\I \omega_n) \vec \sigma_3
                \\
                \uncover<7->{
                    &= -\frac {g^2 T} N \sum_{\vec q m}
                    \vec \sigma_3 \,
                    \tikz [remembered] \node (B) {$\vec G_{\vec q}$}; \,
                    (\I \omega_m) \vec \sigma_3 \, \hypo
                        {D_{\vec k - \vec q}(\I \omega_n - \I \omega_m)}
                        {\tikz \draw [phonon, double] (0, 0) -- (0.7, 0);}
                    }
                \\
                \uncover<5->{
                    \vec G\alt<5>{^0}{^{\vphantom 0}}_{\vec k}(\I \omega_n) &=
                    \tikz [remembered, inner ysep=1mm] \node (A) {$
                        \displaystyle -\frac{
                            \I \omega_n
                            \uncover<6->{Z_{\vec k}(\I \omega_n)}
                            \vec \sigma_0
                            \uncover<6->{+ \phi_{\vec k}(\I \omega_n)
                            \vec \sigma_1}
                            + [\epsilon_{\vec k} - \mu
                            \uncover<6->{ + \chi_{\vec k}(\I \omega_n)}]
                            \vec \sigma_3
                            }{
                            [\omega_n
                            \uncover<6->{Z_{\vec k}(\I \omega_n)}]^2
                            \uncover<6->{+ \phi^2_{\vec k}(\I \omega_n)}
                            + [\epsilon_{\vec k} - \mu
                            \uncover<6->{+ \chi_{\vec k}(\I \omega_n)}]^2
                            }
                        $};
                    }
            \end{align*}
            \uncover<2-6>{
                \begin{tikzpicture}[remember picture, overlay, ultra thick, red,
                    inner sep=0, outer sep=0, <-, shorten <=1mm]
                    \only<2->{\draw (Z) to[out=240, in=90] +(-1cm, -7mm)
                        node [below] {renormalization};}
                    \only<3->{\draw (phi) to[out=270, in=90] +(0, -7mm)
                        node [below] {order parameter};}
                    \only<4->{\draw (chi) to[out=300, in=90] +(1cm, -6.5mm)
                        node [below] {energy shift};}
                \end{tikzpicture}
                }
            \uncover<8>{
                \begin{tikzpicture}[remember picture, overlay, ultra thick, red]
                    \draw [->, shorten >=1mm] (A) to[out=90, in=300] (B);
                    \draw [rounded corners]
                        (A.south west) rectangle (A.north east);
                    \node [below=7mm] at (A) {$\scriptstyle \blacktriangleright$
                        coefficient comparison};
                \end{tikzpicture}
                }
        \end{itemize}
    \end{frame}

    \subsection[\name{Eliashberg} eqs.]{\name{Eliashberg} equations}

    \setbeamertemplate{footline}[info]{%
        G. M. \name{Eliashberg}, Soviet Phys. JETP \textbf{11}, 696 (1960)}

    \def\vspacing{\vphantom{\int\limits_Z^Z}}

    \begin{frame}[label=Eliashberg1]{\name{Eliashberg} equations}
        \begin{center} \bf
            Anisotropic equations
        \end{center}
        %
        \begin{align*}
            Z_{\vec k}(\I \omega_n) &= 1 \alert<4> - \frac T {\alert<4> N}
            \frac 1 {\omega_n} \sum_{\vec q m} \frac
                {\omega_m Z_{\vec q}(\I \omega_m)}
                {\Theta_{\vec q}(m)}
            \alert<4>{g^2 D_{\vec k - \vec q}(\I \omega_n - \I \omega_m)}
            \vspacing
            \\
            \phi_{\vec k}(\I \omega_n) &= \alert<4> -\frac T {\alert<4> N}
            \sum_{\vec q m} \frac
                {\phi_{\vec q}(\I \omega_m)}
                {\Theta_{\vec q}(m)}
            \big[ \alert<4>{g^2 D_{\vec k - \vec q}(\I \omega_n - \I \omega_m)
            \only<2->{+ \alert<2>{U_{\vec k - \vec q}}}} \big]
            \vspacing
            \\
            \chi_{\vec k}(\I \omega_n) &= \frac T {\alert<4> N}
            \sum_{\vec q m} \frac
                {\epsilon_{\vec q} - \mu + \chi_{\vec q}(\I \omega_m)}
                {\Theta_{\vec q}(m)}
            \alert<4>{g^2 D_{\vec k - \vec q}(\I \omega_n - \I \omega_m)}
            \vspacing
            \\
            \hspace{2cm} & \hspace{8cm}
            \\
            \Theta_{\vec k}(n) &= [\omega_n Z_{\vec k}(\I \omega_n)]^2
            + [\epsilon_{\vec k} - \mu + \chi_{\vec k}(\I \omega_n)]^2
            + \phi^2_{\vec k}(\I \omega_n)
            \vspacing
        \end{align*}
        %
        \uncover<4>{
            \begin{tikzpicture}[remember picture, overlay]
                \node [infobox, xshift=-25mm, yshift=-23mm]
                    at (current page.center)
                {$\lambda_{\vec q}(n) = -n(\mu_0) \, g^2 D_{\vec q}(\I \nu_n)$};
                \node [infobox, xshift=15mm, yshift=-23mm]
                    at (current page.center)
                {${\mu \sub C}_{\vec q} = n(\mu_0) \, U_{\vec q}$};
            \end{tikzpicture}
            }
    \end{frame}

    \addtocounter{framenumber}{-1}

    \begin{frame}[label=Eliashberg2]{\name{Eliashberg} equations}
        \begin{center} \bf
            Anisotropic equations
        \end{center}
        %
        \begin{align*}
            Z_{\alert<3>{\vec k}}(\I \omega_n)
            &= 1 \alert<1>+ \frac T {\alert<1>{N(\mu_0)}} \frac 1 {\omega_n}
            \sum_{\vec q m} \frac
                {\omega_m Z_{\alert<3>{\vec q}}(\I \omega_m)}
                {\Theta_{\vec q}(m)}
            \alert<1>{\lambda_{\alert<3>{\vec k - \vec q}}(n - m)}
            \vspacing
            \\
            \phi_{\alert<3>{\vec k}}(\I \omega_n)
            &= \frac T {\alert<1>{N(\mu_0)}}
            \sum_{\vec q m} \frac
                {\phi_{\alert<3>{\vec q}}(\I \omega_m)}
                {\Theta_{\vec q}(m)}
            \big[
                \alert<1>{\lambda_{\alert<3>{\vec k - \vec q}}(n - m)
                - {\mu \sub C}_{\alert<3>{\vec k - \vec q}}}
            \big]
            \vspacing
            \\
            \chi_{\alert<3>{\vec k}}(\I \omega_n)
            &= \alert<1> -\frac T {\alert<1>{N(\mu_0)}}
            \sum_{\vec q m} \frac
                { \epsilon_{\vec q} - \mu
                + \chi_{\alert<3>{\vec q}}(\I \omega_m) }
                {\Theta_{\vec q}(m)}
            \alert<1>{\lambda_{\alert<3>{\vec k - \vec q}}(n - m)}
            \vspacing
            \\
            \hspace{2cm} & \hspace{8cm}
            \\
            \Theta_{\vec k}(n)
            &= [\omega_n Z_{\alert<3>{\vec k}}(\I \omega_n)]^2
            + [\epsilon_{\vec k} - \mu
            + \chi_{\alert<3>{\vec k}}(\I \omega_n)]^2
            + \phi^2_{\alert<3>{\vec k}}(\I \omega_n)
            \vspacing
        \end{align*}
        %
        \uncover<1>{
            \begin{tikzpicture}[remember picture, overlay]
                \node [infobox, xshift=-25mm, yshift=-23mm]
                    at (current page.center)
                {$\lambda_{\vec q}(n) = -n(\mu_0) \, g^2 D_{\vec q}(\I \nu_n)$};
                \node [infobox, xshift=15mm, yshift=-23mm]
                    at (current page.center)
                {${\mu \sub C}_{\vec q} = n(\mu_0) \, U_{\vec q}$};
            \end{tikzpicture}
            }
    \end{frame}

    \begin{frame}[label=Eliashberg3]{\name{Eliashberg} equations}
        \begin{center} \bf
            Local approximation
        \end{center}
        %
        \begin{align*}
            Z(\I \omega_n) &= 1 + \frac T {N(\mu_0)} \frac 1 {\omega_n}
            \alert<2>{\sum_{\vec q \color{black} m}} \frac
                {\omega_m Z\vphantom{_{\vec q}}(\I \omega_m)}
                {\Theta_{\alert<2>{\vec q}}(m)}
            \lambda(n - m)
            \vspacing
            \\
            \phi(\I \omega_n) &= \frac T {N(\mu_0)}
            \alert<2>{\sum_{\vec q \color{black} m}} \frac
                {\phi\vphantom{_{\vec q}}(\I \omega_m)}
                {\Theta_{\alert<2>{\vec q}}(m)}
            \big[ \lambda(n - m) - \mu \sub C \big]
            \vspacing
            \\
            \chi(\I \omega_n) &= -\frac T {N(\mu_0)}
            \alert<2>{\sum_{\vec q \color{black} m}} \frac
                { \epsilon_{\alert<2>{\vec q}} - \mu
                + \chi\vphantom{_{\vec q}}(\I \omega_m) }
                {\Theta_{\alert<2>{\vec q}}(m)}
            \lambda(n - m)
            \vspacing
            \\
            \hspace{2cm} & \hspace{8cm}
            \\
            \Theta_{\alert<2>{\vec k}}(n) &= [\omega_n Z(\I \omega_n)]^2
            + [\epsilon_{\alert<2>{\vec k}} - \mu + \chi(\I \omega_n)]^2
            + \phi^2(\I \omega_n)
            \vspacing
        \end{align*}
        %
        \tikz[remember picture, overlay] \node at (0, 0) {};
    \end{frame}

    \addtocounter{framenumber}{-1}

    \newsavebox\CDOS
    \savebox\CDOS{\small\input{results/cdos-1.sl}}

    \begin{frame}[label=Eliashberg4]{\name{Eliashberg} equations}
        \only<4>{\addtocounter{framenumber}{1}}
        %
        \begin{center} \bf
            \alt<-3>{Local}{CDOS} approximation
        \end{center}
        %
        \begin{align*}
            Z(\I \omega_n) &= 1 + \frac T {\omega_n} \sum_{\vphantom{\vec q} m}
            \alert<1,9>{\int \from{-\infty} \till \infty \D \epsilon} \,
            \only<1-7>{\alert<7>{\frac
                {\alert<1>{N(\alt<1-3>{\alert<3> \epsilon}{\alert<4>{\mu_0}})}}
                {N(\mu_0)}
                }}
            \frac
                {\omega_m Z\vphantom{_{\vec q}}(\I \omega_m)}
                {\alert<9>{\Theta_{\alert<1> \epsilon}(m)}}
            \lambda(n - m)
            \vspacing
            \\
            \phi(\I \omega_n) &= T \sum_{\vphantom{\vec q} m}
            \alert<1,9>{\int \from{-\infty} \till \infty \D \epsilon} \,
            \only<1-7>{\alert<7>{\frac
                {\alert<1>{N(\alt<1-3>{\alert<3> \epsilon}{\alert<4>{\mu_0}})}}
                {N(\mu_0)}
                }}
            \frac
                {\phi\vphantom{_{\vec q}}(\I \omega_m)}
                {\alert<9>{\Theta_{\alert<1> \epsilon}(m)}}
            \big[ \lambda(n - m) - \mu \sub C \big]
            \vspacing
            \\
            \chi(\I \omega_n) &= -T \sum_{\vphantom{\vec q} m}
            \alert<1,9>{\int \from{-\infty} \till \infty \D \epsilon} \,
            \only<1-7>{\alert<7>{\frac
                {\alert<1>{N(\alt<1-3>{\alert<3> \epsilon}{\alert<4>{\mu_0}})}}
                {N(\mu_0)}
                }}
            \frac
                {\epsilon - \mu + \chi\vphantom{_{\vec q}}(\I \omega_m)}
                {\alert<9>{\Theta_{\alert<1> \epsilon}(m)}}
            \lambda(n - m)
            \vspacing
            \\
            \hspace{2cm} & \hspace{8cm}
            \\
            \alert<9>{\Theta_{\alert<1> \epsilon}(n)}
            & \alert<9>{ = [\omega_n Z(\I \omega_n)]^2
            + [\epsilon - \mu + \chi(\I \omega_n)]^2
            + \phi^2(\I \omega_n) }
            \vspacing
        \end{align*}
        %
        \uncover<5>{
            \begin{tikzpicture}[remember picture, overlay]
                \node [infobox] at (current page.center) {\usebox\CDOS};
            \end{tikzpicture}
            }
    \end{frame}

    \addtocounter{framenumber}{-1}

    \begin{frame}[label=Eliashberg5]{\name{Eliashberg} equations}
        \only<7,9>{\addtocounter{framenumber}{1}}
        %
        \begin{center} \bf
            \only<-6>{CDOS approximation}%
            \only<7-8>{Linearized equations}%
            \only<9->{Multi-band equations}
        \end{center}
        %
        \begin{align*}
            \alt<-8> Z {Z_{\alert<9> i}} (\I \omega_n) &=
            1 + \frac
                {\alert<1> \pi \alert<6-7>{\alt<-6> T {T \sub c}}}
                {\omega_n}
            \sum_{m \only<9->{\alert<9>{\smash[b] j}}}
            \alt<-6>{\frac
                {\omega_m \only<-3>{\alert<3>{Z(\I \omega_m)}}}
                {\alert<1>{\sqrt{
                    \alt<4->{\omega_m^2}{[\omega_m \alert<3>{Z(\I \omega_m)}]^2}
                    + \alert<3-4,6>{\alt<4-> \Delta \phi^2(\I \omega_m)}
                    }}}}{\alert<7>{\sgn(\omega_m)} \,}
            \alt<-8> \lambda {\lambda_{\alert<9>{i j}}} (n - m)
            \vspacing
            \\
            \alert<3-4>{
                \only<-3> \phi
                \only<4-8> \Delta
                \only<9-10>{\Delta_{\alert<9> i}}
                (\I \omega_n)
                }
            &=
            \only<-3>{\alert<1> \pi T}
            \only<4->{\frac
                {\pi \alert<6-7>{\alt<-6> T {T \sub c}}}
                {\alert<4>{Z(\I \omega_n)}}
                }
            \sum_{m \only<9->{\alert<9>{\smash[b] j}}}
            \frac
                {\alert<3-4>{
                    \only<-3> \phi
                    \only<4-8> \Delta
                    \only<9-10>{\Delta_{\alert<9>{\smash[b] j}}}
                    (\I \omega_m)
                    }}
                {\alert<1>{\alt<-6>{\sqrt{
                    \alt<4->{\omega_m^2}{[\omega_m \alert<3>{Z(\I \omega_m)}]^2}
                    + \alert<3-4,6>{\alt<4-> \Delta \phi^2(\I \omega_m)}
                    }}{\alert<7>{|\omega_m|}}}}
            [ \alt<-8> \lambda {\lambda_{\alert<9>{i j}}} (n - m)
            - \mu \sub C \only<9->{_{\alert<9>{i j}}} ]
            \vspacing
            \\
            \only<1>{\chi(\I \omega_n)} & \only<1> =
            \only<1>{\alert<1> 0}
            \vspacing
            \\
            \hspace{2cm} & \hspace{8cm}
            \\
            & \vspacing
        \end{align*}
        %
        \uncover<3-4>{
            \tikz[remember picture, overlay]
                \node [infobox, xshift=-12mm, yshift=-15mm]
                    at (current page.center)
                { $\displaystyle \Delta(\I \omega_n)
                = \frac{\phi(\I \omega_n)}{Z(\I \omega_n)}$ };
            }
    \end{frame}

    \subsection[\name{Coulomb} potential]
        {Rescaling the \name{Coulomb} interaction}

    \setbeamertemplate{footline}[info]{%
        e.g. P. B. \name{Allen} and B. \name{Mitrović},
        Solid state physics \textbf{37} (1982)}

    \newsavebox\RDOS
    \savebox\RDOS{\small\input{results/rdos.sl}}
    \savebox\CDOS{\small\input{results/cdos.sl}}

    \begin{frame}[label=Coulomb]{Rescaling the \name{Coulomb} interaction}
        \begin{itemize}
            \item Introduction of cutoff frequency $\omega_N$
            %
            \begin{align*}
                \sum_m \enspace &\rightarrow \enspace
                \sum_m^{|\omega_m| < \omega_N}
                \\
                \intertext{
                    \item Compensation by rescaled \name{Coulomb}
                    pseudo-potential $\mu^*(N)$
                    }
                %
                \frac 1 {\mu \sub C} \enspace &\rightarrow \enspace
                \begin{aligned}[t]
                    \frac 1 {\mu^*(N)} &\approx \frac 1 {\mu \sub C}
                    + \frac 1 \pi \int \from{-\infty} \till \infty \D \epsilon
                    \frac{n(\epsilon)}{n(\mu_0)}
                    \frac
                        {\arctan \frac{\epsilon - \mu}{\omega_N}}
                        {\epsilon - \mu}
                    \\[5mm]
                    \uncover<2->{
                        &\approx \frac 1
                            {\tikz [remembered] \node (muC) {$\mu \sub C$};}
                        + \ln \frac
                            {\tikz [remembered] \node (EF) {$E \sub F$};}
                            {\omega_N}
                        \quad \text{for} \quad
                        \omega_N \ll E \sub F
                        }
                \end{aligned}
            \end{align*}
        \end{itemize}
        %
        \begin{tikzpicture}[remember picture, overlay]
            \uncover<2->{
                \node [infobox, xshift=-4cm, yshift=-28mm]
                    at (current page.center)
                    {\only<-3>{\usebox\RDOS}\only<4>{\usebox\CDOS}};
                }
            \uncover<3->{
                \draw [red]
                    (muC.south west) -- (muC.north east)
                    ( EF.south west) -- ( EF.north east);
                \node [inner sep=0, outer sep=0, below right=1.5mm, red]
                    (muStar) at (muC) {$\mu^*$};
                \node [inner sep=0, outer sep=0, above left=1.5mm, red]
                    (omegaE) at (EF) {$\omega \sub E$};
                \node [infobox, xshift=14mm, yshift=14mm]
                    at (current page.center) {
                        \name{McMillan:}
                        $\displaystyle \frac 1 {\mu^*}
                        = \frac 1 {\mu \sub C}
                        + \ln \frac{E \sub F}{\omega \sub E}$
                        };
                }
            \uncover<4>{
                \draw [titleblue]
                    (muStar.south west) -- (muStar.north east)
                    (omegaE.south west) -- (omegaE.north east);
                \node [inner sep=0, outer sep=0, below right=1.5mm, titleblue]
                    (muStarM) at (muStar) {$\mu^*(M)$};
                \node [inner sep=0, outer sep=0, above left=1.5mm, titleblue]
                    (omegaM) at (omegaE) {$\omega_M$};
                \node [titleblue, below right=3mm] at (muStarM)
                    { $\scriptstyle \blacktriangleright$
                      natural cutoff at $E \sub F$ };
                }
        \end{tikzpicture}
    \end{frame}

    \subsection[Effective scalars]{Effective scalar coupling strengths}

    \setbeamertemplate{footline}[info]{}

    \begin{frame}[label=scalars]{Effective scalar coupling strengths}
        \framesubtitle{
            Scalar $\lambda$ which yields the same $T \sub c$ as matrix $\vec
            \lambda = [\lambda_{i j}]$?}

        Approximate analytical solutions for $\mu^* = 0$
        %
        \begin{enumerate}
            \item Renormalization $Z(\I \omega_n) = 1$
            %
            \begin{center}
                \begin{tikzpicture}
                    \node [infobox] at (current page.center)
                        {$\lambda$ is greatest eigenvalue of $\vec \lambda$};
                \end{tikzpicture}
            \end{center}
            %
            \begin{itemize}
                \item Two bands:
                %
                \begin{equation*}
                    \lambda
                    = \frac 1 2 \Big[ \lambda_{11} + \lambda_{22} + \sqrt{
                        (\lambda_{1 1} - \lambda_{2 2})^2
                        + 4 \lambda_{1 2} \lambda_{2 1}
                        }
                    \Big]
                \end{equation*}
            \end{itemize}
            %
            \item $N = 1$ \name{Matsubara} frequencies
            %
            \begin{center}
                \begin{tikzpicture}
                    \node [infobox] at (current page.center)
                        { $\lambda$ is greatest eigenvalue of $[2 \lambda_{i j}
                        - \delta_{i j} \sum_k \lambda_{i k}]$ };
                \end{tikzpicture}
            \end{center}
            %
            \begin{itemize}
                \item Two bands:
                %
                \begin{equation*}
                    \lambda = \frac 1 2 \Big[
                          \lambda_{1 1} - \lambda_{1 2}
                        - \lambda_{2 1} + \lambda_{2 2}
                        + \sqrt{
                            ( \lambda_{1 1} - \lambda_{1 2}
                            + \lambda_{2 1} - \lambda_{2 2} )^2
                            + 16 \lambda_{1 2} \lambda_{2 1}
                            }
                    \Big]
                \end{equation*}
            \end{itemize}
        \end{enumerate}
    \end{frame}

    \section{Single-band case}

    \subsection{Self-energy}

    \setbeamertemplate{footline}[info]{%
        $\omega \sub E = 20 \, \unit{meV}$, $\lambda = 1$, $\mu^* = 0.1$,
        $\omega_N = 15 \, \omega \sub E$, $n = 0.5$}

    \begin{frame}[label=self-energy]{Self-energy \hspace{22mm} \clap{
            \color{black} \normalsize
            \only<1-2>{Order parameter}%
            \only<3>{Renormalization}%
            \only<4>{Energy shift}%
            }}

        \begin{figure}
            \small
            \only<1-2>{%
                \input{results/self-energy-delta-im-slides.sl}\vspace{-5mm}%
                \uncover<2>{\input{results/self-energy-delta-re-slides.sl}}%
                }%
            \only<3>{%
                \input{results/self-energy-z-im-slides.sl}\vspace{-5mm}%
                \input{results/self-energy-z-re-slides.sl}%
                }%
            \only<4>{%
                \input{results/self-energy-chi-im-slides.sl}\vspace{-5mm}%
                \input{results/self-energy-chi-re-slides.sl}%
                }%
            \vspace{-3cm}%
            \uncover<1>{
                \begin{equation*}
                    \textbf{\name{Padé} approximant} \qquad
                    \Delta(\omega) = \dfrac{
                       c_0 \, \phantom{(\omega - \I \omega_0)}}
                    {1 + \dfrac{c_1 \, (\omega - \I \omega_0)}
                    {1 + \dfrac{c_2 \, (\omega - \I \omega_1)} \dots}}
                \end{equation*}
                }%
        \end{figure}
    \end{frame}

    \subsection{Energy dependence}

    \setbeamertemplate{footline}[info]{}

    \begin{frame}[label=lattice]{Square lattice}
        \begin{figure}
            \small
            \begin{minipage}[c][4cm][c]{12cm/3}
                % !TEX root = ../thesis.tex
%
\tikzsetnextfilename{square-lattice}
%
\begin{tikzpicture}[baseline, shorten >=2mm, shorten <=2mm]
    \foreach \point in {(0, 0), (-1, 0), (1, 0), (0, -1), (0, 1)} {
        \shade [ball color=gray] \point circle (1.5mm);
        }
    \draw [orange] (-0.5, -0.5) rectangle (0.5, 0.5);
    \foreach \point in {(-1, 0), (1, 0), (0, -1), (0, 1)} {
        \draw [<-] \point to [bend left]  (0, 0);
        \draw [->] \point to [bend right] (0, 0);
        }
\end{tikzpicture}

            \end{minipage}%
            \begin{minipage}[c][4cm][c]{16cm/3}
                \begin{align*}
                    \op H = -t \sum_{\vec R}
                         [  \ket{\vec R + \vec t_1}
                         &+ \ket{\vec R - \vec t_1} \\{}
                          + \ket{\vec R + \vec t_2}
                         &+ \ket{\vec R - \vec t_2} ]
                    \bra{\vec R}
                \end{align*}
            \end{minipage}\\
            \input{results/square-lattice-dispersion.sl}%
            \input{results/square-lattice-dos.sl}
        \end{figure}
    \end{frame}

    \setbeamertemplate{footline}[info]{%
        $\omega \sub E = 20 \, \unit{meV}$, $\lambda = 1$, $\mu^* = 0.1$,
        $\omega_N = 15 \, \omega \sub E$}

    \begin{frame}[label=energy]{Energy dependence}
        \begin{figure}
            \small
            \centering
            \medmuskip=0mu
            \input{results/energy-dos-slides.sl}
            \input{results/energy-shift-slides.sl}
            \input{results/energy-tc-slides.sl}
        \end{figure}
    \end{frame}

    \subsection{\name{McMillan}'s formula}

    \setbeamertemplate{footline}[info]{%
        W. L. \name{McMillan}, Phys. Rev. \textbf{167}, 331 (1968)}

    \newsavebox\McMillanA
    \newsavebox\McMillanB

    \savebox\McMillanA
        {\hspace*{2mm}\small\input{results/mcmillan-1.sl}\hspace*{2mm}}

    \savebox\McMillanB
        {\small\input{results/mcmillan-2.sl}\hspace*{4mm}}

    \begin{frame}[label=McMillan]{\name{McMillan}'s formula}
        \begin{enumerate}
            \item Linearized CDOS \name{Eliashberg} equations on real axis
            \item Trial function for energy gap
            %
            \begin{equation*}
                \Delta(\omega) = \begin{cases}
                    \Delta_0 & \text{for $|\omega| < \omega_0$} \\
                    \Delta_\infty & \text{otherwise}
                \end{cases}
            \end{equation*}
            %
            \item Analytic expression for $T \sub c$
            %
            \begin{equation*}
                T \sub c = \mathrlap{
                    \alt<1>{\omega_0}{
                        \frac
                            {\alt<4>{\omega \sub E}{\av \omega}}
                            {\only<2> A \only<3>{1.20} \only<4->{0.94}}
                        }
                    \exp \bigg[{ -\frac
                        { \only<2> B \only<3>{1.04} \only<4->{1.11} \,
                          \only<2->(1 + \lambda \only<2->) }
                        { \lambda - \only<2> C \only<3>{0.62} \only<4->{0.74}
                          \only<2->\, \lambda \mu^*
                          \only<1>{\av \omega / \omega_0} - \mu^* }
                    }\bigg]
                }
                \hspace{4cm}
            \end{equation*}
            %
            \item<2-> Relaxation by introduction of fit parameters
            \item<3-> Linear regression for\dots
            %
            \begin{itemize}
                \setlength{\itemindent}{15mm}
                \item[originally:] phonon DOS of niobium
                \item[in this work:]<4-> \name{Einstein} spectrum
            \end{itemize}
        \end{enumerate}
        %
        \uncover<5-6>{
            \begin{tikzpicture}[remember picture, overlay]
                \node [infobox] at (current page.center)
                {\only<5>{\usebox\McMillanA}\only<6>{\usebox\McMillanB}};
            \end{tikzpicture}
            }
    \end{frame}

    \setbeamertemplate{footline}[info]{}

	\section{Conclusion}

	\begin{frame}{Conclusion}
	\end{frame}
\end{document}
