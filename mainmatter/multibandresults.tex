% !TEX root = ../thesis.tex

\chapter{Multi-band results}

\section{Temperature dependence of order parameter}

\begin{figure}
   \small
   \input{results/order_parameter/connect.sl}
   \input{results/order_parameter/connect_back.sl}
   \input{results/order_parameter/connect_forth.sl}
   \caption{Slowly switching on inter-band coupling\dots \cite{NicolCarbotte05}}
\end{figure}

\section{Effective electron-phonon coupling}

\subsection{cDOS approximation}

\begin{itemize}
    \item Mathematically, it is proved that the maximum eigenvalue of the
          coupling matrix is the effective coupling constant. So just show some
          graphs \dots
\end{itemize}

\begin{figure}
   \small
   \input{results/2-band/inter_inter.sl}
   \input{results/2-band/intra_intra.sl}
   \input{results/2-band/inter_intra.sl}
   \input{results/2-band/intra_inter.sl}
   \input{results/2-band/inter_askew.sl}
   \input{results/2-band/intra_askew.sl}
\end{figure}

\begin{figure}
   \small
   \input{results/2-band/lamda_inf.sl}
   \input{results/2-band/lamda_inf_hyperbola.sl}
\end{figure}

\begin{figure}
   \small
   \begin{subfigure}{7cm}
      \input{results/2-band/tc_cutoff-independent.sl}
      \caption{cutoff-independent}
   \end{subfigure}%
   \begin{subfigure}{7cm}
      \input{results/2-band/tc_non-renormalized.sl}
      \caption{non-renormalized}
   \end{subfigure}%
\end{figure}

\subsection{Subdomain-resolved density of states}

\begin{itemize}
    \item Show differences between maximum eigenvalue and effective coupling
          constant for different densities of states
\end{itemize}
