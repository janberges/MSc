% !TEX root = ../thesis.tex

\chapter{Multi-band results}

\section{Temperature dependence of order parameter}

\begin{figure}
   \small
   \input{results/order_parameter/connect.sl}
   \input{results/order_parameter/connect_back.sl}
   \input{results/order_parameter/connect_forth.sl}
   \captionsetup{singlelinecheck=off}
   \caption[Incrementally switching on inter-band coupling]{
       Incrementally switching on inter-band coupling. In analogy to Fig.~11 of
       Ref.~\barecite{NicolCarbotte05}, the temperature dependence of the
       leading \name{Matsubara} gaps of a two-band system with a phonon
       frequency $\omega \sub E = 20 \, \unit{meV}$ and no explicit
       \name{Coulomb} interaction is shown for different electron-phonon
       coupling matrices
       %
       \begin{equation*}
           \vec \lambda \sub{top} =
           \begin{bmatrix}
               1 & x \\
               x & 2
           \end{bmatrix},
           \quad
           \vec \lambda \sub{left} =
           \begin{bmatrix}
               1 & 0 \\
               x & 2
           \end{bmatrix}
           \quad \text{and} \quad
           \vec \lambda \sub{right} =
           \begin{bmatrix}
               1 & x \\
               0 & 2
           \end{bmatrix}.
       \end{equation*}
       }
\end{figure}

\section{Effective electron-phonon coupling}

\subsection{cDOS approximation}

\begin{itemize}
    \item Mathematically, it is proved that the maximum eigenvalue of the
          coupling matrix is the effective coupling constant. So just show some
          graphs \dots
\end{itemize}

\begin{figure}
   \small
   \input{results/2-band/inter_inter.sl}
   \input{results/2-band/intra_intra.sl}
   \input{results/2-band/inter_intra.sl}
   \input{results/2-band/intra_inter.sl}
   \input{results/2-band/inter_askew.sl}
   \input{results/2-band/intra_askew.sl}
   \caption{
       Hyperbolas of constant $T \sub c$. For the parameters $\omega \sub E = 20
       \, \unit{meV}$, $\omega_N = 25 \, \omega \sub E$ and $\mu^* = 0$,
       hyperbolas are shown, along which either intra- or inter-band
       electron-phonon coupling strengths in a two-band system may be jointly
       varied without changing the critical temperature. They intersect the
       bisector or the odd quadrants at $\lambda \sub{d.} = \lambda_{1 1} =
       \lambda_{2 2}$ and $\lambda \sub{od.} = \lambda_{1 2} = \lambda_{2 1}$,
       respectively. Note that the matrix $\vec \lambda = \lambda \sub{d.} \vec
       \sigma_0 + \lambda \sub{od.} \vec \sigma_1$ and the scalar $\lambda =
       \lambda \sub{d.} + \lambda \sub{od.}$ yield the same critical
       temperature.}
\end{figure}

\begin{figure}
   \small
   \input{results/2-band/lamda_inf.sl}
   \input{results/2-band/lamda_inf_hyperbola.sl}
   \caption{
       Position of the asymptotes $\lambda_\infty$ of the constant-$T \sub c$
       hyperbola for inter-band coupling only, i.e. $\lambda_{1 1} = \lambda_{2
       2} = 0$, as a function of the coupling strengh $\lambda \sub{od.}$. Not
       only exact results of the cDOS \name{Eliashberg} theory but also the
       behavior within the approximations of a cutoff-independent mapping onto
       effective scalar coupling constants (Section~\ref{cutoff-independent})
       and a renormalization of unity (Section~\ref{non-renormalized}) are
       presented. Phonon and cutoff frequency are chosen to be $\omega \sub E =
       20 \, \unit{meV}$ and $\omega_N = 15 \, \omega \sub E$, respectively.}
\end{figure}

\begin{figure}
   \small
   \begin{subfigure}{7cm}
      \input{results/2-band/tc_cutoff-independent.sl}
      \caption{cutoff-independent}
   \end{subfigure}%
   \begin{subfigure}{7cm}
      \input{results/2-band/tc_non-renormalized.sl}
      \caption{non-renormalized}
   \end{subfigure}%
   \caption{
       Visualization of the quality of the introduced approximate mappings onto
       effective scalar coupling strengths. For 500 two-band electron-phonon
       coupling matrices, the elements of which are random samples from a
       uniform distribution over $[0, 1)$, and their corresponding scalar
       couplings strengths, critical temperatures are calculated and plotted
       against each other. Again, $\omega \sub E = 20 \, \unit{meV}$ and
       $\omega_N = 15 \, \omega \sub E$.}
\end{figure}

\subsection{Subdomain-resolved density of states}

\begin{itemize}
    \item Show differences between maximum eigenvalue and effective coupling
          constant for different densities of states
\end{itemize}
