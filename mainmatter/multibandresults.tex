% !TEX root = ../thesis.tex

\chapter{Multi-band results}

In the previous chapter it has been confirmed that \name{McMillan}'s equation
can very well be used to predict critical temperatures of local
\name{Eliashberg} theory, if only one electronic band is considered. The
question remains, whether it may also be applied if the coupling strengths are
matrices rather than scalars, which is equivalent to the problem of finding
single-band systems which resemble multi-band systems with respect to the
critical temperature -- self-energies etc. will certainly differ. Having three
degrees of freedom at hand, namely the phonon frequency $\omega \sub E$, the
electron-phonon coupling $\lambda$ and the \name{Coulomb} pseudo-potential
$\mu^*$, there are of course innumerable possibilities to accomplish this.
However, since an unphysical redistribution of influences shall be avoided, the
further shall not be modified and, if the latter vanishes, one is left with
determining a single parameter $\lambda$ without ambiguities. This is what the
present chapter is dedicated to, for the special case of a two-band system and
within the CDOS approximation.

To begin with, the temperature dependence of the order parameters of the
different bands is investigated in analogy to Section~\ref{temperature
dependence of the order parameter}. Next, it presented how intra- and inter-band
coupling strengths can be varied independently without altering the critical
temperature. Based on this, the approximate mappings onto scalar coupling
constants proposed in Section~\ref{effective coupling constants} are validated
using the example of inter-band coupling alone. Finally, the same approximations
are tested with respect to the critical temperature.

\section{Temperature dependence of order parameter}

\begin{figure}
    \small
    \input{results/order_parameter/connect.sl}
    \input{results/order_parameter/connect_back.sl}
    \input{results/order_parameter/connect_forth.sl}
    \captionsetup{singlelinecheck=off}
    \caption[Incrementally switching on inter-band coupling]{
        Incrementally switching on inter-band coupling. In analogy to Fig.~11 of
        Ref.~\barecite{NicolCarbotte05}, the temperature dependence of the
        leading \name{Matsubara} gaps of a two-band system with a phonon
        frequency $\omega \sub E = 20 \, \unit{meV}$ and no explicit
        \name{Coulomb} interaction is shown for different electron-phonon
        coupling matrices
        %
        \begin{equation*}
            \vec \lambda \sub{top} =
            \begin{bmatrix}
                1 & x \\
                x & 2
            \end{bmatrix},
            \quad
            \vec \lambda \sub{left} =
            \begin{bmatrix}
                1 & 0 \\
                x & 2
            \end{bmatrix}
            \quad \text{and} \quad
            \vec \lambda \sub{right} =
            \begin{bmatrix}
                1 & x \\
                0 & 2
            \end{bmatrix}.
        \end{equation*}
        }
    \label{incrementally switching on inter-band coupling}
\end{figure}
%
In the multi-band formalism a different self-energy, and thus a different energy
gap, may be associated with each band. In Fig.~\ref{incrementally switching on
inter-band coupling} the two gaps of the considered two-band systems are shown
as functions of temperature, i.e. in their role as order parameters, for
different strengths of the inter-band coupling. The latter is chosen equally
spaced on a logarithmic scale, which leads to a higher resolution on the side of
small couplings. In the upper panel both inter-band coupling strengths are
varied simultaneously, in the lower panels either of them is zero in order to
point out its specific influence. Actually the latter situations are unphysical,
since they may only be realized if either of the band densities of states, which
also enter the diagonal coupling strengths, is zero or infinite (see
Section~\ref{multi-band equations}). The diagonal elements, however, are chosen
to be finite constants which differ by a factor of two.

In the case of intra-band coupling only, one finds two curves which exactly
resemble those for the single-band systems corresponding to the separate bands
-- because that is what they are. Each band has its own critical temperature,
the upper one is associated with the system as a whole. In the background of
Fig.~\ref{incrementally switching on inter-band coupling} the number of
iterations needed to read convergence is displayed for this special case. It is
found to be enhanced at both critical temperatures.

As soon as the influence of the band with the higher critical temperature on the
other band is switched on -- it is important to note that this coupling is not
bidirectional --, the end of the lower curve extends towards the end of the
upper one resulting in a common critical temperature, which retains the exact
value determined by the greater intra-band coupling strength (right panel). For
this to happen, the strength of the switched-on coupling is irrelevant and only
influences the magnitude along the newly formed tail.

In the case of a reversed influence, the upper critical temperature comes closer
to the lower one, which in turn remains constant (left panel). At the same time,
the lower curve becomes apparent in the shape of the upper one at the
corresponding temperatures, resulting in sharp bend at the lower critical
temperature.

When both effects are combined, i.e. when the inter-band couplings are switched
on simultaneously, there is a common temperature which depends on the coupling
strength (upper panel). Because of the extension of the lower one, both curves
are differentiable except at the critical temperature.

Notably, the critical temperature is not always enhanced if a component of the
electron-phonon coupling increases. The possibility of an inhibiting influence
is disregarded when taking the maximum eigenvalue of the coupling matrix as
effective scalar coupling strength, as described in
Section~\ref{non-renormalized}, since this monotonically increases as a function
of any element and so does the critical temperature.

\section{Effective electron-phonon coupling}

\subsection{CDOS approximation}

\begin{itemize}
    \item Mathematically, it is proved that the maximum eigenvalue of the
          coupling matrix is the effective coupling constant. So just show some
          graphs \dots
\end{itemize}

\begin{figure}
    \small
    \input{results/2-band/inter_inter.sl}
    \input{results/2-band/intra_intra.sl}
    \input{results/2-band/inter_intra.sl}
    \input{results/2-band/intra_inter.sl}
    \input{results/2-band/inter_askew.sl}
    \input{results/2-band/intra_askew.sl}
    \caption[Hyperbolas of constant $T \sub c$]{
        Hyperbolas of constant $T \sub c$. For the parameters $\omega \sub E = 20
        \, \unit{meV}$, $\omega_N = 25 \, \omega \sub E$ and $\mu^* = 0$,
        hyperbolas are shown, along which either intra- or inter-band
        electron-phonon coupling strengths in a two-band system may be jointly
        varied without changing the critical temperature. They intersect the
        bisector or the odd quadrants at $\lambda \sub{d.} = \lambda_{1 1} =
        \lambda_{2 2}$ and $\lambda \sub{od.} = \lambda_{1 2} = \lambda_{2 1}$,
        respectively. Note that the matrix $\vec \lambda = \lambda \sub{d.} \vec
        \sigma_0 + \lambda \sub{od.} \vec \sigma_1$ and the scalar $\lambda =
        \lambda \sub{d.} + \lambda \sub{od.}$ yield the same critical
        temperature.}
\end{figure}

\begin{figure}
    \small
    \input{results/2-band/lamda_inf.sl}
    \input{results/2-band/lamda_inf_hyperbola.sl}
    \caption[Asymptotes of inter-band hyperbola of constant $T \sub c$]{
        Position of the asymptotes $\lambda_\infty$ of the constant-$T \sub c$
        hyperbola for inter-band coupling only, i.e. $\lambda_{1 1} = \lambda_{2
        2} = 0$, as a function of the coupling strengh $\lambda \sub{od.}$. Not
        only exact results of the CDOS \name{Eliashberg} theory but also the
        behavior within the approximations of a cutoff-independent mapping onto
        effective scalar coupling constants (Section~\ref{cutoff-independent})
        and a renormalization of unity (Section~\ref{non-renormalized}) are
        presented. Phonon and cutoff frequency are chosen to be $\omega \sub E =
        20 \, \unit{meV}$ and $\omega_N = 15 \, \omega \sub E$, respectively.}
\end{figure}

\begin{figure}
    \small
    \begin{subfigure}{7cm}
        \input{results/2-band/tc_cutoff-independent.sl}
        \caption{cutoff-independent}
    \end{subfigure}%
    \begin{subfigure}{7cm}
        \input{results/2-band/tc_non-renormalized.sl}
        \caption{non-renormalized}
    \end{subfigure}%
    \caption[Quality of the effective scalar coupling strengths]{
        Visualization of the quality of the introduced approximate mappings onto
        effective scalar coupling strengths. For 500 two-band electron-phonon
        coupling matrices, the elements of which are random samples from a
        uniform distribution over $[0, 1)$, and their corresponding scalar
        couplings strengths, critical temperatures are calculated and plotted
        against each other. Again, $\omega \sub E = 20 \, \unit{meV}$ and
        $\omega_N = 15 \, \omega \sub E$.}
\end{figure}

\subsection{Subdomain-resolved density of states}

\begin{itemize}
    \item Show differences between maximum eigenvalue and effective coupling
          constant for different densities of states
\end{itemize}
