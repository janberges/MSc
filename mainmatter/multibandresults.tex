% !TEX root = ../thesis.tex

\chapter{Multi-band results}

In the previous chapter it has been confirmed that \name{McMillan}'s equation
can very well be used to predict critical temperatures of local
\name{Eliashberg} theory, if only one electronic band is considered. The
question remains, whether it may also be applied if the coupling strengths are
matrices rather than scalars, which is equivalent to the problem of finding
single-band systems which resemble multi-band systems with respect to the
critical temperature -- self-energies etc. will certainly differ. Having three
degrees of freedom at hand, namely the phonon frequency $\omega \sub E$, the
electron-phonon coupling $\lambda$ and the \name{Coulomb} pseudo-potential
$\mu^*$, there are of course innumerable possibilities to accomplish this.
However, since an unphysical redistribution of influences shall be avoided, the
further shall not be modified and, if the latter vanishes, one is left with
determining a single parameter $\lambda$ without ambiguities. This is what the
present chapter is dedicated to, for the special case of a two-band system and
within the CDOS approximation.

To begin with, the temperature dependence of the order parameters of the
different bands is investigated in analogy to Section~\ref{temperature
dependence of the order parameter}. Next, it presented how intra- and inter-band
coupling strengths can be varied independently without altering the critical
temperature. Based on this, the approximate mappings onto scalar coupling
constants proposed in Section~\ref{effective coupling constants} are validated
using the example of inter-band coupling alone. Finally, the same approximations
are tested with respect to the critical temperature.

\section{Temperature dependence of order parameter}

\begin{figure}
    \small
    \input{results/order_parameter/connect.sl}
    \input{results/order_parameter/connect_back.sl}
    \input{results/order_parameter/connect_forth.sl}
    \captionsetup{singlelinecheck=off}
    \caption[Incrementally switching on inter-band coupling]{
        Incrementally switching on inter-band coupling. In analogy to Fig.~11 of
        Ref.~\barecite{NicolCarbotte05}, the temperature dependence of the
        leading \name{Matsubara} gaps of a two-band system with a phonon
        frequency $\omega \sub E = 20 \, \unit{meV}$ and no explicit
        \name{Coulomb} interaction is shown for different electron-phonon
        coupling matrices
        %
        \begin{equation*}
            \vec \lambda \sub{top} =
            \begin{bmatrix}
                1 & x \\
                x & 2
            \end{bmatrix},
            \quad
            \vec \lambda \sub{left} =
            \begin{bmatrix}
                1 & 0 \\
                x & 2
            \end{bmatrix}
            \quad \text{and} \quad
            \vec \lambda \sub{right} =
            \begin{bmatrix}
                1 & x \\
                0 & 2
            \end{bmatrix}.
        \end{equation*}
        }
    \label{incrementally switching on inter-band coupling}
\end{figure}
%
In the multi-band formalism a different self-energy, and thus a different energy
gap, may be associated with each band. In Fig.~\ref{incrementally switching on
inter-band coupling} the two gaps of the considered two-band systems are shown
as functions of temperature, i.e. in their role as order parameters, for
different strengths of the inter-band coupling. The latter is chosen equally
spaced on a logarithmic scale, which leads to a higher resolution on the side of
small couplings. In the upper panel both inter-band coupling strengths are
varied simultaneously, in the lower panels either of them is zero in order to
point out its specific influence. Actually the latter situations are unphysical,
since they may only be realized if either of the band densities of states, which
also enter the diagonal coupling strengths, is zero or infinite (see
Section~\ref{multi-band equations}). The diagonal elements, however, are chosen
to be finite constants which differ by a factor of two.

In the case of intra-band coupling only, one finds two curves which exactly
resemble those for the single-band systems corresponding to the separate bands
-- because that is what they are. Each band has its own critical temperature,
the upper one is associated with the system as a whole. In the background of
Fig.~\ref{incrementally switching on inter-band coupling} the number of
iterations needed to read convergence is displayed for this special case. It is
found to be enhanced at both critical temperatures.

As soon as the influence of the band with the higher critical temperature on the
other band is switched on -- it is important to note that this coupling is not
bidirectional --, the end of the lower curve extends towards the end of the
upper one resulting in a common critical temperature, which retains the exact
value determined by the greater intra-band coupling strength (right panel). For
this to happen, the strength of the switched-on coupling is irrelevant and only
influences the magnitude along the newly formed tail.

In the case of a reversed influence, the upper critical temperature comes closer
to the lower one, which in turn remains constant (left panel). At the same time,
the lower curve becomes apparent in the shape of the upper one at the
corresponding temperatures, resulting in sharp bend at the lower critical
temperature.

When both effects are combined, i.e. when the inter-band couplings are switched
on simultaneously, there is a common temperature which depends on the coupling
strength (upper panel). Because of the extension of the lower one, both curves
are differentiable except at the critical temperature.

Notably, the critical temperature is not always enhanced if a component of the
electron-phonon coupling increases. The possibility of an inhibiting influence
is disregarded when taking the maximum eigenvalue of the coupling matrix as
effective scalar coupling strength, as described in
Section~\ref{non-renormalized}, since this monotonically increases as a function
of any element and so does the critical temperature.

\section{Effective electron-phonon coupling}

Now the dependence of the critical temperature on the electron-phonon coupling
matrix shall be analyzed in more detail. Only few exacts results will be stated
but rather the approximations proposed in Section~\ref{effective coupling
constants} tested with respect to their accuracy.

\subsection{Hyperbolas of constant $T \sub c$}

\begin{figure}
    \small
    \input{results/2-band/inter_inter.sl}
    \input{results/2-band/intra_intra.sl}
    \input{results/2-band/inter_intra.sl}
    \input{results/2-band/intra_inter.sl}
    \input{results/2-band/inter_askew.sl}
    \input{results/2-band/intra_askew.sl}
    \caption[Hyperbolas of constant $T \sub c$]{
        Hyperbolas of constant $T \sub c$. For the parameters $\omega \sub E = 20
        \, \unit{meV}$, $\omega_N = 25 \, \omega \sub E$ and $\mu^* = 0$,
        hyperbolas are shown, along which either intra- or inter-band
        electron-phonon coupling strengths in a two-band system may be jointly
        varied without changing the critical temperature. They intersect the
        bisector or the odd quadrants at $\lambda \sub{d.} = \lambda_{1 1} =
        \lambda_{2 2}$ and $\lambda \sub{od.} = \lambda_{1 2} = \lambda_{2 1}$,
        respectively. Note that the matrix $\vec \lambda = \lambda \sub{d.} \vec
        \sigma_0 + \lambda \sub{od.} \vec \sigma_1$ and the scalar $\lambda =
        \lambda \sub{d.} + \lambda \sub{od.}$ yield the same critical
        temperature.}
    \label{hyperbolas of constant Tc}
\end{figure}
%
To divide the problem into manageable parts the following question is proposed:
How must either the intra- or the inter-band coupling strengths change
\emph{simultaneously}, with the respective other pair of elements held constant,
so that the corresponding critical temperature is conserved?

The outcome for some representative parameter sets is displayed in
Fig.~\ref{hyperbolas of constant Tc}. The numerical results, illustrated as
scatter plots, immediately suggest the following dependency: The matching intra-
and inter-band coupling strengths lie on convex and concave sections of
hyperbolas, respectively. This assumption is confirmed by comparison with a
guess for the analytic dependence, which will be developed subsequently.

Let $x$ and $y$ represent the variable and dependent elements of the
electron-phonon coupling matrix, respectively. A possible equation defining a
hyperbola through $x = y = \lambda$ reads
%
\begin{equation*}
    y = \frac {\lambda^2} x.
\end{equation*}
%
However, the asymptotes would coincide with the $x$- and $y$-axes which is
definitely not always the case in Fig.~\ref{hyperbolas of constant Tc}. The
hyperbola is thus compressed by the factors $\alpha$ and $\beta$ in $x$- and
$y$-direction, respectively, with $x = y = \lambda$ as the fixed point. This
yields
%
\begin{equation*}
    y = \frac
        {\frac{\lambda^2} {\alpha (x - \lambda) + \lambda} - \lambda}
        \beta
    + \lambda
    = \frac 1 \beta \frac
        {\lambda^2 - \lambda [\alpha (x - \lambda) + \lambda]}
        {\alpha (x - \lambda) + \lambda}
    + \lambda
    = \lambda - \frac \alpha \beta \frac
        {\lambda (x - \lambda)}
        {\alpha (x - \lambda) + \lambda}
    = \lambda - \frac \alpha \beta \frac
        \lambda
        {\alpha + \frac \lambda {x - \lambda}}.
\end{equation*}
%
The new asymptotes are at $x = x_\infty = \lambda (1 - \alpha^{-1})$ and $y =
y_\infty = \lambda (1 - \beta^{-1})$. Thus
%
\begin{equation*}
    y = \lambda - \frac{\lambda - y_\infty}{\lambda - x_\infty} \frac
        \lambda
        {\frac \lambda {\lambda - x_\infty} + \frac \lambda {x - \lambda}}
    = \lambda - \frac{\lambda - y_\infty}{x - x_\infty} (x - \lambda).
\end{equation*}
%
This formula describes the desired relation of the intra-band coupling strengths
for $x_\infty$ and $y_\infty$ less than or equal to $\lambda$. In contrast,
$x_\infty$ and $y_\infty$ greater than or equal to $\lambda$ are required for
the description of intra-band coupling. The asymptotes may also be expressed in
terms of the $x$- and $y$-intercepts $x_0$ and $y_0$:
%
\begin{align*}
    x_0 &= \lambda \Big[ 1 - \frac{\lambda - x_\infty}{y_\infty} \Big], &
    x_\infty &= \frac{\lambda^2 x_0}{\lambda (x_0 + y_0) - x_0 y_0}, \\
    y_0 &= \lambda \Big[ 1 - \frac{\lambda - y_\infty}{x_\infty} \Big], &
    y_\infty &= \frac{\lambda^2 y_0}{\lambda (x_0 + y_0) - x_0 y_0}.
\end{align*}
%
For each parameter set the asymptotes are determined numerically. For the
inter-band coupling, $x_\infty$ and $y_\infty$ are calculated directly by
assigning a very large value, $10^{15}$ say, to either of $\lambda_{1 2}$ or
$\lambda_{2 1}$ and solving for the other element. In the intra-band case the
asymptotes are excluded from the domain of possible values. Hence, the
intercepts are calculated, where either $\lambda_{1 1}$ or $\lambda_{2 2}$ is
nullified. The position of the asymptotes is then concluded by means of the
above equations.

The determined hyperbolas are plotted together with the numerical data points,
which reveals a very good agreement. The choice of the correct asymptotes
depends not only on $\lambda$ but also on the two matrix elements which are held
constant. If an analytic expression of these dependencies were known, a closed
set of equations would be obtained which could be solved for the scalar
complement of the whole coupling matrix. This task is not accomplished within
the present work, but in the following section some predictions of the
approximate mappings regarding this matter are presented.

\subsection{Asymptotic behavior for inter-band coupling}

\begin{figure}
    \small
    \input{results/2-band/lamda_inf.sl}
    \input{results/2-band/lamda_inf_hyperbola.sl}
    \caption[Asymptotes of inter-band hyperbola of constant $T \sub c$]{
        Position of the asymptotes $\lambda_\infty$ of the constant-$T \sub c$
        hyperbola for inter-band coupling only, i.e. $\lambda_{1 1} = \lambda_{2
        2} = 0$, as a function of the coupling strengh $\lambda$. Not only exact
        results of the CDOS \name{Eliashberg} theory but also the behavior
        within the approximations of a cutoff-independent mapping onto effective
        scalar coupling constants (Section~\ref{cutoff-independent}) and a
        renormalization of unity (Section~\ref{non-renormalized}) are presented.
        Phonon and cutoff frequency are chosen to be $\omega \sub E = 20 \,
        \unit{meV}$ and $\omega_N = 15 \, \omega \sub E$, respectively.}
    \label{inter-band hyperbola of constant Tc}
\end{figure}
%
In this section the approximate mappings onto effective scalar coupling
strengths which were introduced in Section~\ref{effective coupling constants}
are visualized for the special case of inter-band coupling alone, i.e.
$\lambda_{1 1} = \lambda_{2 2} = 0$. Because of the following symmetry of the
off-diagonal elements, $\lambda_{1 2}$ as a function of $\lambda_{2 1}$ is an
involution, i.e. its own inverse. Subsequently, both asymptotes are have the
same distance to their corresponding axes.

The latter is determined both numerically and analytically according to
Eqs.~\ref{non-renormalized} and \ref{cutoff-independent}. In the case studied,
the latter reduce to
%
\begin{equation*}
    \lambda = \sqrt{\lambda_{1 2} \lambda_{2 1}}
    \quad \text{and} \quad
    \lambda = -\frac 1 2 \Big[ \lambda_{1 2} + \lambda_{2 1} + \sqrt{
        (\lambda_{2 1} + \lambda_{1 2})^2
        + 12 \lambda_{1 2} \lambda_{2 1}
        }
    \Big].
\end{equation*}
%
Solving for $\lambda_{2 1}$ and taking the limit $\lambda_{1 2} \rightarrow
\infty$ yields the corresponding asymptotes:
%
\begin{equation*}
    \lambda_{2 1} = \frac{\lambda^2}{\lambda_{1 2}} \rightarrow 0
    \quad \text{and} \quad
    \lambda_{2 1} = \frac{\lambda + \lambda_{1 2}}{3 \lambda_{1 2} - \lambda}
    \lambda \rightarrow \frac \lambda 3.
\end{equation*}
%
The results are presented in Fig.~\ref{inter-band hyperbola of constant Tc}. For
all values of $\lambda$, the approximation of a cutoff-independent mapping
yields better results than the assumption of a renormalization of unity.
Especially for $\lambda \approx 1$, the agreement is satisfactory; for larger
$\lambda$ is worsens continuously.

\section{Critical temperatures for single-band approximations}

\begin{figure}
    \small
    \begin{subfigure}{7cm}
        \input{results/2-band/tc_cutoff-independent.sl}
        \caption{cutoff-independent}
    \end{subfigure}%
    \begin{subfigure}{7cm}
        \input{results/2-band/tc_non-renormalized.sl}
        \caption{non-renormalized}
    \end{subfigure}%
    \caption[Quality of the effective scalar coupling strengths]{
        Visualization of the quality of the introduced approximate mappings onto
        effective scalar coupling strengths. For 500 two-band electron-phonon
        coupling matrices, the elements of which are random samples from a
        uniform distribution over $[0, 1)$, and their corresponding scalar
        couplings strengths, critical temperatures are calculated and plotted
        against each other. Again, $\omega \sub E = 20 \, \unit{meV}$ and
        $\omega_N = 15 \, \omega \sub E$.}
\end{figure}
