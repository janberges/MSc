% !TEX root = ../thesis.tex

\chapter{Many-body physics}
\label{many-body physics}

Superconductivity results from the interaction of a huge number of particles and
quasiparticles, conventionally electrons and phonons, and depends strongly on
temperature. A prominent approach to such problems is the \name{Green}-function
method of statistical physics, which emerged in the middle of the 20th century
as a side product of quantum electrodynamics.%
%
\footnote{In 1949 \name{Dyson} publishes an attempt to unify \qq{The radiation
theories of Tomonaga, Schwinger, and Feynman} \cite{Dyson49} together with an
early \name{Feynman} diagram. Six years later \name{Matsubara} applies the new
methods to the calculation of the grand-canonical partition function
\cite{Matsubara55}.}
%
Since the equations which make up \name{Eliashberg} theory, namely
Eqs.~\ref{Eliashberg equations} on page~\pageref{Eliashberg equations}, are only
meaningful within this framework, a review of the relevant aspects is given in
this chapter.%
%
\footnote{Physical concepts that are very well established nowadays will be
stated without reference to their specific origins. They are covered in most
textbooks on this subject such as the one by \name{Mahan} \cite{Mahan00}.}

Initially, the possible ways of handling time dependence in quantum mechanics
are presented, focussing on the interaction picture. On this basis \name{Green}
functions are defined, including their imaginary-axis formulation. Next, not
only to exemplify what has been stated so far but also because it will provide
the fundamental building block for what follows, the special case of
non-interacting particles is discussed. Subsequently, the perturbation series
and \name{Wick}'s theorem for non-zero temperatures are derived, which prepare
the ground for \name{Feynman}'s diagrammatical perturbation theory. For some
model interactions the most important diagrams are deduced explicitly. Finally,
the self-energy and its most common approximations are introduced.

In some places where no ambiguity arises, different quantities are represented
by the same symbol and only distinguished by the name of their formal argument.
This concerns mutual \name{Fourier} transforms as well as functions of real and
imaginary time.

\section{Dynamical pictures}

It is convenient to present time dependence in quantum mechanics on the basis of
expectation values. Let the \textsc{Hamilton} operator of the system be $\op H =
\op H_0 + \op V$ with $\op H_0$ diagonal and $\op V$ an interaction. At time
$t$, the expectation value of the observable $\op X$ in the state $\ket \psi$
reads
%
\begin{equation*}
    \bra \psi \, \E^{\I \op H t} \, \op X \, \E^{-\I \op H t} \, \ket \psi =
    \bra \psi \, \E^{\I \op H t} \, \E^{-\I \op H_0 t} \,
    \E^{\I \op H_0 t} \, \op X \, \E^{-\I \op H_0 t} \,
    \E^{\I \op H_0 t} \, \E^{-\I \op H t} \, \ket \psi.
\end{equation*}
%
So far, both observables and states are assumed to be independent of time. In
practice however, they are usually associated with the adjacent exponential
functions, which can be done in different ways, each of which corresponds to a
so-called \emph{dynamical picture}.

In the \emph{\name{Schrödinger} picture} the full time dependency is ascribed to
the states. Differentiation with respect to time yields the \textsc{Schrödinger}
equation. Formally,
%
\begin{equation*}
    \ket{\psi(t)} = \E^{-\I \op H t} \, \ket \psi
    \quad \Rightarrow \quad
    \I \frac \D {\D t} \ket{\psi(t)} = \op H \ket{\psi(t)}.
\end{equation*}

As opposed to this, in the \emph{\textsc{Heisenberg} picture} only the
observables depend on time. In this case differentiation using the product rule
yields \textsc{Heisenberg}'s equation of motion. Thus
%
\begin{equation*}
    \op X(t) = \E^{\I \op H t} \, \op X \, \E^{-\I \op H t}
    \quad \Rightarrow \quad
    \I \frac \D {\D t} \op X(t) = [\op X(t), \op H].
\end{equation*}

A useful compromise is provided by the the \emph{\textsc{Dirac} picture}, also
known as \emph{interaction picture}, where the time dependence is shared among
observables and states. The former evolve according to the unperturbed part of
the \textsc{Hamilton} operator while the latter are governed by the interaction.

To be able to distinguish between the different pictures, time arguments will be
enclosed in square brackets $[\dots]$. Formally, both a \name{Heisenberg}- and a
\name{Schrödinger}-like equation of motion emerge for observables and states,
respectively:
%
\begin{subequations}
    \begin{align}
        \label{Dirac operator}
        \op X[t] = \E^{\I \op H_0 t} \, \op X \, \E^{-\I \op H_0 t}
        \quad &\Rightarrow \quad
        \I \frac \D {\D t} \op X[t] = [\op X[t], \op H_0],
        \\
        \label{Dirac state}
        \ket{\psi[t]} = \E^{\I \op H_0 t} \, \E^{-\I \op H t} \, \ket \psi
        \quad &\Rightarrow \quad
        \I \frac \D {\D t} \ket{\psi[t]} = \op V[t] \ket{\psi[t]}.
    \end{align}
\end{subequations}

\subsection{\name{Dyson} series}

According to Eq.~\ref{Dirac state}, the unitary \emph{time-evolution operator}
$\op S(t, t_0)$ for arbitrary initial times $t_0$,%
%
\footnote{\name{Dirac} introduces this operator in §\,44 \qq{The perturbation
considered as causing transitions} of his textbook on quantum mechanics
\cite{Dirac58}.}
%
precisely the one defined by $\ket{\psi[t]} = \op S(t, t_0) \ket{\psi[t_0]}$,
has to satisfy the differential equation
%
\begin{equation*}
    \frac \D {\D t} \op S(t, t_0) = -\I \op V[t] \, \op S(t, t_0)
    \quad \text{with} \quad
    \op S(t_0, t_0) = 1.
\end{equation*}
%
Integration followed by a fixed-point iteration, which is expected to converge,
yields
%
\begin{equation*}
    \op S(t, t_0) = 1 - \I \int \from{t_0} \till t \D t' \,
    \op V[t] \, \op S(t, t_0) =
    \sum_{n = 0}^\infty (-\I)^n
    \int \from{t_0} \till t \D t_1
    \int \from{t_0} \till{t_1} \D t_2 \dots
    \int \from{t_0} \till{t_{n - 1}} \D t_n \,
    \op V[t_1] \dots \op V[t_n].
\end{equation*}

With $S_n$ being the group of all $n!$ permutations of $1 \dots n$ one can just
as well write
%
\begin{equation*}
    \op S(t, t_0) = \sum_{n = 0}^\infty \frac{(-\I)^n}{n!} \sum_{P \in S_n}
    \int \from{t_0} \till t \D t_{P(1)}
    \int \from{t_0} \till{t_{P(1)}} \D t_{P(2)} \dots
    \int \from{t_0} \till{t_{P(n - 1)}} \D t_{P(n)} \,
    \op V[t_{P(1)}] \dots \op V[t_{P(n)}]
\end{equation*}
%
since a permutation of the dummy variables does not alter the value of the
integral. For $t > t_0$, which will be assumed from now on, the domain of
integration is always an $n$-simplex defined by $t_0 < t_{P(n)} < \dots <
t_{P(1)} < t$. Noting that the $n$-simplices for all permutations add up to the
$n$-dimensional hypercube defined by $t_0 < t_i < t$ for all $i \in \{ 0 \dots n
\}$,%
%
\footnote{The idea of speaking of \q{simplices} and \q{hypercubes} in this
context is taken from lecture notes by V. \name{Kaplunovsky}.}
%
one finds most beneficial formulation of the \emph{\name{Dyson} series}%
%
\footnote{See Eq.~32 in Ref.~\barecite{Dyson49} for an analogous formula.}
%
\begin{equation} \label{Dyson series}
    \op S(t, t_0) = \sum_{n = 0}^\infty \frac{(-\I)^n}{n!}
    \int \from{t_0} \till t \D t_1 \dots
    \int \from{t_0} \till t \D t_n \,
    \op T \, \op V[t_1] \dots \op V[t_n],
\end{equation}
%
where $\op T$ is a \emph{time-ordering operator}, which sorts the factors of a
product of operators chronologically with the result that their actual time
arguments ascend from right to left.

If the time dependence of the perturbation operator does not involve any
non-commuting quantities, the time-ordering operator is no longer needed and
Eq.~\ref{Dyson series} can be written as
%
\begin{equation*}
    \op S(t, t_0) = \exp \left[
        -\I \int \from{t_0} \till t \D t' \, \op V[t'] \right].
\end{equation*}
%
But even if the operators do not commute it is common practice to use a symbolic
exponential function with the time-ordering operator placed in front.

\section{\name{Green} functions}
\label{Green functions}

Gathering complete information about an interacting many-body system is a hard,
if not impossible task. However, much can be learned by studying the behavior of
single test particles within this system.

To that end \emph{correlation functions} like $\av{\op A(t) \, \op B(t')}$ are
considered, where $\op A$ and $\op B$ are either fermionic or bosonic ladder
operators. They describe how the presence or absence of a certain particle at
one time correlates with an analogous occurrence at another time.

To take an equilibrium temperature $\beta^{-1}$ into account, $\av \dots$ shall
denote an ensemble average. For brevity, a canonical ensemble with the partition
function $Z = \Tr \E^{-\beta \op H}$ is assumed; associating the \name{Hamilton}
operator $\op H$ with $\op H - \mu \op N$, where $\mu$ is the chemical potential
and $\op N$ the particle-number operator, yields the grand canonical ensemble.
%
\begin{equation} \label{C(t)}
    \begin{split}
        \av{\op A(t) \, \op B(t')} &= \frac 1 Z \Tr
        \big[
            \E^{-\beta \op H} \,
            \E^{\I \op H t} \, \op A \, \E^{-\I \op H t} \,
            \E^{\I \op H t'} \, \op B \, \E^{-\I \op H t'}
        \big]
        \\
        &= \frac 1 Z \Tr
        \big[
            \E^{-\beta \op H} \,
            \E^{\I \op H (t - t')} \, \op A \, \E^{-\I \op H (t - t')} \, \op B
        \big]
        = \av{\op A(t - t') \, \op B} \equiv C(t - t')
    \end{split}
\end{equation}
%
depends on time differences only, since the trace of a product is invariant
under cyclic permutations of the factors. With the energy eigenstates $\ket n$
and -values $E_n$,
%
\begin{equation} \label{C(t) in energy basis}
    C(t) = \frac 1 Z \sum_{n m} \bra n
    \E^{-\beta \op H} \, \E^{\I \op H t} \, \op A \, \E^{-\I \op H t}
    \ket m \bra m \op B \ket n =
    \frac 1 Z \sum_{n m} \E^{-\beta E_n}
    \underbrace{\bra n \op A \ket m}_{A_{n m}}
    \underbrace{\bra m \op B \ket n}_{B_{m n}}
    \E^{\I (E_n - E_m) t}.
\end{equation}
%
A \name{Fourier} transform using Eq.~\ref{int dt exp(i omega t)} reveals a
weighted excitation spectrum
%
\begin{equation} \label{C(omega)}
    C(\omega) = \frac 1 {2 \pi} \int \from{-\infty} \till \infty \D t \,
    C(t) \, \E^{\I \omega t} =
    \frac 1 Z \sum_{n m} \E^{-\beta E_n} \, A_{n m} B_{m n} \,
    \delta[\omega - (E_m - E_n)].
\end{equation}

At temperature $\beta^{-1}$, the probability that a many-particle state $\ket n$
with energy $E_n$ is realized reads $\E^{-\beta E_n} / Z$ and the average
occupation number of a single-particle state with energy $\epsilon$ is given by
$f(\epsilon) = [\E^{\beta \epsilon} \pm 1]^{-1}$ as derived in Section~\ref{free
particles}. From now on upper and lower signs hold for fermions and bosons,
respectively. One can easily verify that
%
\begin{equation*}
    \E^{-\beta E_n} =
    \big[ 1 \mp f(E_m - E_n) \big]
    \big[ \E^{-\beta E_n} \pm \E^{-\beta E_m} \big]
\end{equation*}
%
and write Eq.~\ref{C(omega)} as $C(\omega) = [1 \mp f(\omega)] \, A(\omega)$
with the \emph{spectral function}
%
\begin{equation} \label{A(omega)}
    A(\omega) = \frac 1 Z \sum_{n m} \big[
        \E^{-\beta E_n} \pm \E^{-\beta E_m}
    \big] \, A_{n m} B_{m n} \, \delta[\omega - (E_m - E_n)].
\end{equation}
%
It is straightforward to reverse the steps that led from Eq.~\ref{C(t)} to
Eq.~\ref{C(omega)} and apply them to Eq.~\ref{A(omega)}. The \name{Fourier}
transform of the spectral function turns out to be the expectation value of the
(anti-) commutator of the respective operators:
%
\begin{equation} \label{A(t)}
    A(t) = \int \from{-\infty} \till \infty \D \omega \,
    A(\omega) \, \E^{-\I \omega t} =
    \av{[\op A(t), \op B]_\pm}.
\end{equation}

Causality forbids the present to affect the past. This may be implemented into
the theory by nullifying the correlation for negative time differences. Together
with a factor $-\I$, which is solely introduced for convenience, this yields the
\emph{retarded \name{Green} function}
%
\begin{equation} \label{G_ret(t)}
    G \sub{ret.} (t) = -\I \Theta(t) \, A(t) =
    -\I \Theta(t) \, \av{[\op A(t), \op B]_\pm}.
\end{equation}
%
With the help of Eqs.~\ref{A(t)} and \ref{int+ dt exp(i omega t)}, yet another
\name{Fourier} transform back to the energy domain, by convention without a
factor of $(2 \pi)^{-1}$, yields
%
\begin{equation} \label{G_ret(omega)}
    G \sub{ret.} (\omega) = \int \from{-\infty} \till \infty \D t \,
    G \sub{ret.} (t) \, \E^{\I \omega t} = -\I
    \int \from{-\infty} \till \infty \D \omega' \, A(\omega')
    \int \from 0 \till \infty \D t \, \E^{\I (\omega - \omega') t} =
    \int \from{-\infty} \till \infty \D \omega' \,
    \frac{A(\omega')}{\omega - \omega' + \I 0^+}.
\end{equation}
%
The \name{Sokhotski–Plemelj} theorem leads to a useful expression for the
spectral function:
%
\begin{equation} \label{spectral representation}
	\Im \frac 1 {\omega + \I 0^+} = -\pi \delta(\omega),
    \quad \Rightarrow \quad
    A(\omega) = -\frac 1 \pi \Im G \sub{ret.} (\omega).
\end{equation}

\subsection{Imaginary-time formalism}

The retarded \name{Green} function as defined in Eq.~\ref{G_ret(t)} contains two
different types of exponential functions: the ensemble weight which decays with
increasing energy, and the ones oscillating with time. It would be convenient to
be able to treat both in one go which can be accomplished by assuming the time
to be purely imaginary. Introducing a real parameter $\tau = \I t$ and writing
%
\begin{equation*}
    \op X(\tau) = \E^{\op H \tau} \, \op X \, \E^{-\op H \tau}
\end{equation*}
%
as in Eq.~2.6 of Ref.~\barecite{Matsubara55}, the theory as been formally freed
from the periodic terms.

The \emph{\name{Matsubara-Green} function} is preliminary defined as
%
\begin{equation*}
    G(\tau) = -\av{\op T \, \op A(\tau) \, \op B},
\end{equation*}
%
where $\op T$ acts as in Eq.~\ref{Dyson series}, except that is sorts with
respect to the parameter $\tau$ and induces a change of sign each time two
fermion operator are transposed.

Using the cyclic property of the trace and introducing unity, one finds the
property
%
\begin{align*}
    \av{\op A(\tau) \, \op B} &= \frac 1 Z \Tr \big[ \E^{-\beta \op H} \,
    \E^{\op H \tau} \, \op A \, \E^{-\op H \tau} \,
    \E^{\beta \op H} \, \E^{-\beta \op H} \, \op B \big]
    \\
    &= \frac 1 Z \Tr \big[ \E^{-\beta \op H} \, \op B \,
    \E^{\op H (\tau - \beta)} \, \op A \, \E^{-\op H (\tau - \beta)} \big]
    = \av{\op B \, \op A(\tau - \beta)}
\end{align*}
%
and as a consequence for $0 < \tau < \beta$
%
\begin{equation} \label{(anti-) periodicity}
    G(\tau)
    = -\av{\op B \, \op A(\tau - \beta)}
    = \pm \av{\op T \, \op A(\tau - \beta) \, \op B}
    = \mp G(\tau - \beta).
\end{equation}

Being only interested in the interval $(-\beta, \beta)$ one can just as well
consider a modified \textsc{Green} function which continues periodically beyond
this domain, namely
%
\begin{equation*}
    \widetilde G(\tau) = \frac 1 \beta \sum_{n \in \mathds Z}
        \E^{-\I \pi n \tau / \beta} G_n
    \quad \text{with} \quad
    G_n = \frac 1 2 \int \from{-\beta} \till \beta \D \tau \,
        \E^{\I \pi n \tau / \beta} G(\tau).
\end{equation*}
%
Because of the (anti-) periodicity found in Eq.~\ref{(anti-) periodicity} one
has
%
\begin{equation*}
    \int \from{-\beta} \till 0 \D \tau \, \E^{\I \pi n \tau / \beta} \, G(\tau)
    = \mp \int \from{-\beta} \till 0 \D \tau \,
        \E^{\I \pi n \tau / \beta} G(\tau + \beta)
    = \mp \E^{-\I \pi n} \int \from 0 \till \beta \D \tau \,
        \E^{\I \pi n \tau / \beta} G(\tau),
\end{equation*}
%
which causes every other series coefficient to vanish, thus
%
\begin{equation*}
    G_n =
    \begin{cases}
        \int_0^\beta \D \tau \, \E^{\I \pi n \tau / \beta} \, G(\tau)
        & \text{for fermions (bosons) if $n$ is odd (even),} \\ 0
        & \text{otherwise.}
    \end{cases}
\end{equation*}

So one redefines the \emph{\name{Matsubara}-Green function} and its
imaginary-axis representation
%
\begin{equation} \label{Matsubara-Green function}
    G(\tau) = \frac 1 \beta \sum_{n \in \mathds Z}
        \E^{-\I \omega_n \tau} G(\I \omega_n)
    \quad \text{and} \quad
    G(\I \omega_n) = \int \from 0 \till \beta \D \tau \,
        \E^{\I \omega_n \tau} G(\tau)
    = -\int \from 0 \till \beta \D \tau \,
        \E^{\I \omega_n \tau} \av{\op A(\tau) \op B},
\end{equation}
%
with the \emph{\name{Matsubara} frequencies}
%
\begin{equation*}
    \omega_n =
    \begin{cases}
        \frac{(2 n + 1) \pi} \beta & \text{for fermions}, \\
        \frac{2 n \pi} \beta & \text{for bosons.}
    \end{cases}
\end{equation*}

The latter are denoted $\nu_n$ in the future. Using Eq.~\ref{C(t) in energy
basis} and comparing with Eqs.~\ref{G_ret(omega)} one finds that $G \sub{ret.}
(\omega)$ can be obtained from $G(\I \omega)$ through \emph{analytic
continuation} $\I \omega \rightarrow \omega + \I 0^+$, since
%
\begin{equation} \label{G(i omega_n)}
    \begin{split}
        G(\I \omega_n) &= -\frac 1 Z \sum_{n m}
        \E^{-\beta E_n} A_{n m} B_{m n}
        \int \from 0 \till \beta \D \tau \,
        \E^{\I \omega_n \tau} \E^{(E_n - E_m) \tau} \\
        &=
        \frac 1 Z \sum_{n m} A_{n m} B_{m n}
        \frac{\E^{-\beta E_n} \pm \E^{-E_m \beta}}{\I \omega_n - (E_m - E_n)}
        = \int \from{-\infty} \till \infty \D \omega
        \frac{A(\omega)}{\I \omega_n - \omega}.
    \end{split}
\end{equation}

\section{Free particles}
\label{free particles}

The theory presented in the preceding sections will now be exemplified by means
of the special case of free particles, which will turn out to be fundamental in
the following section. In this context \q{free} shall not be understood as
\q{free from any forces} but rather as \q{non-interacting} since independent
\name{Bloch} electrons in a fixed lattice are very well considered free, just as
every quasi-particle which diagonalizes a single-particle \name{Hamilton}
operator. In second quantization the latter is quadratic in the particle
operators and diagonalized it reads
%
\begin{equation*}
    \op H_0 = \sum_k \epsilon_k \op n_k
    \quad \text{with} \quad
    \op n_k = \op a_k^+ \op a_k.
\end{equation*}
%
where $\op a_k$ annihilates a fermion or boson with arbitrary, possibly
combined, quantum number $k$, while $\epsilon_k$ and the operator $\op n_k$
represent the corresponding energy and occupation number.

First, the time-dependence of the creation and annihilation operators is
presented. Using Eq.~\ref{Dirac operator} as well as the canonical (anti-)
commutation relations one finds
%
\begin{equation} \label{time-dependent ladder operators}
    \begin{aligned}
        \op a_k[t] &= \E^{-\I \epsilon_k t} \, \op a_k, &
        \op a_k^+[t] &= \E^{\I \epsilon_k t} \, \op a_k^+, \\
        \op a_k[\tau] &= \E^{-\epsilon_k \tau} \, \op a_k, &
        \op a_k^+[\tau] &= \E^{\epsilon_k \tau} \, \op a_k^+.
    \end{aligned}
\end{equation}
%
The \name{Dirac} picture is used to allow for a perturbation $\op V$ to be added
without having to update these equations. If $\op H_0$ is already the full
\name{Hamilton} operator, they are equally valid in the \name{Heisenberg}
picture.

Next, the \emph{\name{Fermi-Dirac}} and the \emph{\name{Bose-Einstein}
distribution} $f_\pm$ are derived, which give the average number of fermions and
bosons, respectively, with the same quantum number as a function of their
energy. Each eigenstate $\ket n$ of the free \name{Hamilton} operator $\op H_0$
is a \name{Fock} state, thus an (anti-) symmetric product of the occupied
single-particle states, and the corresponding energy $E_n = \sum_k \epsilon_k
n_k$, where $n_k$ are occupation numbers, is a sum of single-particle energies.
%
\begin{equation*}
    f(\epsilon_k) = \av{\op n_k}_0
    = \frac{\sum_n \bra n \op n_k \ket n \E^{-\beta E_n}}
        {\sum_n \E^{-\beta E_n}}
    = \frac{\sum_{n_k = 0}^N n_k \E^{-\beta \epsilon_k n_k}}
        {\sum_{n_k = 0}^N \E^{-\beta \epsilon_k n_k}},
\end{equation*}
%
where $N$ is the maximum number of particles allowed to occupy the same state
and $\av \dots_0$ denotes an average with respect to a diagonal \name{Hamilton}
operator. In the last step the fraction has been reduced through division by the
partial sum of the denominator for which $\bra n \op n_k \ket n = 0$.

For fermions $N = 1$ and thus
%
\begin{equation*}
    f_+(\epsilon_k) = \frac{\E^{-\beta \epsilon_k}}{1 + \E^{-\beta \epsilon_k}}
    = \frac 1 {\E^{\beta \epsilon_k} + 1}.
\end{equation*}

For bosons $N = \infty$ and, recognizing that $0 + (n_k - 1) + 1, 1 + (n_k - 2)
+ 1 \dots (n_k - 1) + 0 + 1$ are $n_k$ ways to express $n_k$ as well as a
geometric series, one finds
%
\begin{equation*}
    f_-(\epsilon_k)
    = \frac {\sum_{n = 0}^\infty \sum_{m = 0}^\infty
        \E^{-\beta \epsilon_k (n + m + 1)}}
        {\sum_{n = 0}^\infty \E^{-\beta \epsilon_k n}}
    = \sum_{m = 0}^\infty \E^{-\beta \epsilon_k (m + 1)}
    = \frac{\E^{-\beta \epsilon_k}}{1 - \E^{-\beta \epsilon_k}}
    = \frac 1 {\E^{\beta \epsilon_k} - 1}.
\end{equation*}

Finally, the \name{Green} functions on the imaginary-axis as defined in
Eq.~\ref{Matsubara-Green function} are determined.

For electrons, the operators $\op A$ and $\op B$ are simply $\op c_{\vec k
\sigma}$ and $\op c_{\vec k \sigma}^+$. One can formulate correlation functions
$\av{\op c_{\vec k \sigma}[\tau] \, \op c_{\vec k \sigma}^+}_0 =
\E^{-\epsilon_{\vec k} \tau} [1 - f_+(\epsilon_{\vec k})]$ and therewith
%
\begin{equation} \label{G_0(i omega_n)}
    G_{\vec k \sigma}^0(\I \omega_n)
    = -\int \from 0 \till \beta \D \tau \, \E^{\I \omega_n \tau}
    \av{\op c_{\vec k \sigma}[\tau] \, \op c_{\vec k \sigma}^+}_0
    = \frac
        {1 + \E^{-\beta \epsilon_{\vec k}}}
        {\I \omega_n - \epsilon_{\vec k}}
        [1 - f_+(\epsilon_{\vec k})]
    = \frac 1 {\I \omega_n - \epsilon_{\vec k}},
\end{equation}
%
which corresponds to a spectral function $A(\omega) = \delta(\omega -
\epsilon_{\vec k})$ as follows from Eq.~\ref{G(i omega_n)}.

For phonons one usually applies a different notation with symmetric operators
$\upphi_{\vec q} = \op b_{\vec q} + \op b_{-\vec q}^+$. Analogously, this yields
$\av{\upphi_{\vec q}[\tau] \, \upphi_{\vec q}^+}_0 = \E^{-\omega_{\vec q} \tau}
[1 + f_-(\omega_{\vec q})] + \E^{\omega_{-\vec q} \tau} f_-(\omega_{-\vec q})$
and thus
%
\begin{equation} \label{D_0(i omega_n)}
    D_{\vec q}^0(\I \nu_n)
    = -\int \from 0 \till \beta \D \tau \, \E^{\I \nu_n \tau}
    \av{\upphi_{\vec q}[\tau] \, \upphi_{\vec q}^+}_0
    = \frac 1 {\I \nu_n - \omega_{\vec q}}
    - \frac 1 {\I \nu_n + \omega_{-\vec q}}
    \overset{\omega_{\vec q} = \omega_{-\vec q}} =
    -\frac{2 \omega_{\vec q}}{\nu_n^2 + \omega_{\vec q}^2}
\end{equation}
%
as well as the spectral function $B(\omega) = \delta(\omega - \omega_{\vec q}) -
\delta(\omega + \omega_{-\vec q})$.

\section{Perturbation series}

It is yet to be clarified how the full \name{Green} function for arbitrary
interactions can be calculated. The basis idea is to build it iteratively from
what is already known, namely the \name{Green} functions of free particles.

The first step is to rewrite the \name{Green} function in a way that both the
ensemble average and the time dependence of the operators refer to the
unperturbed part $\op H_0$ of the \name{Hamilton} operator.

In the imaginary-time formalism, the time-evolution operator according to
Eq.~\ref{Dyson series} reads
%
\begin{equation} \label{Matsubara-Dyson series}
    \op S(t, t_0) = \sum_{n = 0}^\infty \frac{(-1)^n}{n!}
    \int \from{\tau_0} \till \tau \D \tau_1 \dots
    \int \from{\tau_0} \till \tau \D \tau_n \,
    \op T \, \op V[\tau_1] \dots \op V[\tau_n],
\end{equation}
%
For brevity, let $\op S(\tau) = \op S(\tau, 0) = \E^{\op H_0 \tau} \, \E^{-\op H
\tau}$. Inserting unities it can be shown that
%
\begin{align*}
    \av \dots &= \frac 1 Z \Tr \big[ \E^{-\beta \op H} \dots \big]
    = \frac 1 Z \Tr \big[
        \E^{-\beta \op H_0} \, \E^{\beta \op H_0} \, \E^{-\beta \op H} \dots
    \big]
    = \frac{\av{\op S(\beta) \dots}_0}{\av{\op S(\beta)}_0},
    \\
    \op A(\tau) &= \E^{\op H \tau} \, \op A \, \E^{-\op H \tau}
    = \E^{\op H \tau} \, \E^{-\op H_0 \tau} \,
    \E^{\op H_0 \tau} \, \op A \, \E^{-\op H_0 \tau} \,
    \E^{\op H_0 \tau} \, \E^{-\op H \tau}
    = \op S^{-1}(\tau) \, \op A[\tau] \, \op S(\tau).
\end{align*}

The \name{Matsubara-Green} function defined in Eq.~\ref{Matsubara-Green
function} may thus be written as
%
\begin{align*}
    G(\tau) &= -\av{\op T \, \op A(\tau) \, \op B(0)} =
    -\Theta(\tau) \av{\op A(\tau) \, \op B(0)}
    \pm \Theta(-\tau) \av{\op B(-\tau) \, \op A(0)}
    \\
    &= \frac 1 {\av{S(\beta)}_0} \left \{
    \begin{aligned}
        - & \langle \op S(\beta) &&
        \op S^{-1}(\tau) && \op A[\tau] && \op S(\tau) &&
        \op B[0] \rangle_0 && \text{for $0 < \tau < \beta$,} \\
        \pm & \langle \op S(\beta) &&
        \op S^{-1}(-\tau) && \op B[-\tau] && \op S(-\tau) &&
        \op A[0] \rangle_0 && \text{for $-\beta < \tau < 0$.}
    \end{aligned}
    \right .
\end{align*}
%
Since $\op S(\beta) \, \op S^{-1}(\pm \tau) = \op S(\beta, \pm \tau)$, in both
cases the expectation value is fully time-ordered from right to left.
Consequently, it makes no difference if one introduces a time-ordering operator
for the \name{Dirac} picture that acts \emph{after} the time-evolution operators
have been expanded up to the level of creation and annihilation operators. Then
the factors may be freely permuted, except that sign changes have to taken into
account. However, the latter does not apply to the time-evolution operators
since the interaction usually contains an even number of fermion operators.
Mutual inverses cancel each other and the final expression reads
%
\begin{equation} \label{perturbation series}
    G(\tau) = -\frac{\av{\op T \, \op A[\tau] \, \op S(\beta) \, \op B[0]}_0}
    {\av{\op S(\beta)}_0}.
\end{equation}

Using the above formula, the \name{Matsubara-Green} function may be calculated
to an arbitrary order of accuracy by simply truncating the Taylor series in
Eq.~\ref{Matsubara-Dyson series} after the corresponding number of terms.

As an example, a perturbation of the form $\op V = \sum_{k k'} v_{k k'} \, \op
a_{k'}^+ \op a_k$ is considered. As a single-particle operator it describes the
interaction of a particle with some scattering potential rather than with other
particles. The first terms of the enumerator in Eq.~\ref{perturbation series}
read
%
\begin{gather*}
    \av{\op T \, \op A[\tau] \, \op S(\beta) \, \op B[0]}_0 =
    \av{\op T \, \op A[\tau] \, \op B[0]}_0 -
    \int \from 0 \till \beta \D \tau_1
    \sum_{k, k'} v_{k k'}
    \av{
        \op T \, \op A[\tau] \,
        \op a_{k'}^+[\tau_1] \op a_k[\tau_1] \,
        \op B[0]
    }_0 \\ + \frac 1 2
    \int \from 0 \till \beta \D \tau_1
    \int \from 0 \till \beta \D \tau_2
    \sum_{\mathclap{k, k', q, q'}} v_{k k'} v_{q q'}
    \av{
        \op T \, \op A[\tau] \,
        \op a_{k'}^+[\tau_1] \, \op a_k[\tau_1] \,
        \op a_{q'}^+[\tau_2] \, \op a_q[\tau_2] \,
        \op B[0]
    }_0 + \dots
\end{gather*}
%
Continuing in this way, expectation values of products of a growing number of
creation and annihilation operators will emerge. Fortunately, these can be
factorized into free-particle \name{Green} functions with the help of
\name{Wick}'s theorem, which is presented in the following section.

This is the last step to be taken on the way to \name{Feynman}'s diagrammatical
perturbation theory, which will be outlined afterwards during the analysis of
some model interactions.

\subsection{\name{Wick}'s theorem}

For an even number of fermion or any number of boson operators $\op A_i$ with $i
\in \{ 1 \dots n \}$ one has

\begin{equation} \label{left to right}
    [\op A_1, \op A_2 \dots \op A_n]_\pm =
    \sum_{i = 2}^n (\mp 1)^i \,
    \op A_2 \dots \op A_{i - 1} \,
    [\op A_1, \op A_i]_\pm \,
    \op A_{i + 1} \dots \op A_n,
\end{equation}
%
where $i \dots j$ is an empty sequence for $i > j$. For example, the special
case of six operators reads
%
\begin{align*}
        [\op A, \op B \op C \op D \op E \op F]_\pm
    =   [\op A, \op B]_\pm \op C \op D \op E \op F
    \mp \op B [\op A, \op C]_\pm \op D \op E \op F
    +   \op B \op C [\op A, \op D]_\pm \op E \op F
    \mp \op B \op C \op D [\op A, \op E]_\pm \op F
    +   \op B \op C \op D \op E [\op A, \op F]_\pm.
\end{align*}

Taking the average of the left-hand side of Eq.~\ref{left to right} with respect
to $\op H_0$ yields
%
\begin{equation*}
    \av{[\op A_1, \op A_2 \dots \op A_n]_\pm}_0
    = \frac 1 Z \Tr \{
    \E^{-\beta \op H_0} [ \op A_1, \op A_2 \dots \op A_n ]_\pm \}
    = \frac 1 Z \Tr \{
    [\E^{-\beta \op H_0}, \op A_1]_\pm \, \op A_2 \dots \op A_n] \},
\end{equation*}
%
where the cyclic property of the trace has been used. With the help of
Eq.~\ref{time-dependent ladder operators} one finds
%
\begin{equation*}
    [\E^{-\beta \op H_0}, \op A_1]_\pm
    = \E^{-\beta \op H_0} \E^{\beta \op H_0} [\E^{-\beta \op H_0}, \op A_1]_\pm
    = \E^{-\beta \op H_0} \big[
        1 \pm \E^{\epsilon(A_1) \beta}
    \big] \op A_1
    = \frac{\E^{-\beta \op H_0} \op A_1}{\pm f_\pm[\epsilon(A_1)]},
\end{equation*}
%
where $\epsilon(\op A)$ is the energy change caused by applying $\op A$ to the
state of the unperturbed system.

The right-hand side of Eq.~\ref{left to right} is averaged as well. A
comparison,
%
\begin{equation*}
    \av{[\op A_1, \op A_2 \dots \op A_n]_\pm}_0
    = \frac{\av{\op A_1 \dots \op A_n}_0}{\pm f_\pm[\epsilon(A_1)]}
    = \sum_{i = 2}^n (\mp 1)^i [\op A_1, \op A_i]_\pm
    \av{\op A_2 \dots \op A_{i - 1} \, \op A_{i + 1} \dots \op A_n}_0,
\end{equation*}
%
considering $\av{\op A \op B}_0 = \pm f_\pm[\epsilon(\op A)] [\op A, \op B]_\pm$
yields \emph{\name{Wick}'s theorem} \cite{Wick50} for non-zero temperatures
\cite{Gaudin60}:
%
\begin{align*}
    \av{\op A_1 \dots \op A_n}_0 &=
    \sum_{i = 2}^n (\mp 1)^i \av{\op A_1 \op A_i}_0 \,
    \av{\op A_2 \dots \op A_{i - 1} \, \op A_{i + 1} \dots \op A_n}_0 \\
    &= \sum_{\mathclap{P \in P_n}} (\mp 1)^{T(P)}
    \av{\op A_{P(1)} \op A_{P(2)}}_0 \,
    \av{\op A_{P(3)} \op A_{P(4)}}_0 \dots
    \av{\op A_{P(n - 1)} \op A_{P(n)}}_0,
\end{align*}
%
where $T(P)$ is the number of transpositions the permutation $P$ consists of and
%
\begin{equation*}
    P_n = \{ P \in S_n \mid
        P(1) < P(3) < \dots < P(n - 1)
        \text{ and }
        P(i) < P(i + 1) \text{ for all $i \in \{1, 3 \dots n - 1\}$}
        \}
\end{equation*}
%
is the group of all possible pairings. Since within each pair of operators the
original order is preserved, the theorem holds for time-ordered expectation
values as well, thus
%
\begin{equation} \label{Wick theorem}
    \av{\op T \, \op A_1[\tau_1] \dots \op A_n[\tau_n]}_0 =
    \sum_{\mathclap{P \in P_n}} (\mp 1)^{T(P)}
    \av{\op T \, \op A_{P(1)}[\tau_{P(1)}] \,
    \op A_{P(2)}[\tau_{P(2)}]}_0 \dots
    \av{\op T \, \op A_{P(n - 1)}[\tau_{P(n - 1)}] \,
    \op A_{P(n)}[\tau_{P(n)}]}_0.
\end{equation}

\section{Model interactions}
\label{Model interactions}

\begin{subequations} \label{Green functions of interest}
    For a satisfactory description of superconductivity, not only the \q{normal}
    \name{Green} function
    %
    \begin{equation}
        G_{\vec k \sigma}(\tau) = -\av{\op T \,
        \op c_{\vec k \sigma}(\tau) \,
        \op c_{\vec k \sigma}^+(0)} \in \mathds R,
    \end{equation}
    %
    which describes the propagation of an electron, but also the \emph{anomalous
    \name{Gor'kov-Green} function}
    %
    \begin{equation}
        F_{\vec k \sigma}(\tau) = -\av{\op T \,
        \op c_{\vec k \sigma}(\tau) \,
        \op c_{-\vec k -\sigma}(0)} \in \mathds C,
    \end{equation}
    %
    which indicates the existence of \name{Cooper} pairs in analogy to the BCS
    gap in Eq.~\ref{BCS order parameter}, is needed. Solely for convenience, one
    introduces two additional functions
    %
    \begin{align}
        \smash{\widetilde G_{\vec k \sigma}}(\tau) &= -\av{\op T \,
        \op c_{-\vec k -\sigma}^+(\tau) \,
        \op c_{-\vec k -\sigma}(0)} = -G_{-\vec k -\sigma}(-\tau),
        \\
        \smash{\widetilde F_{\vec k \sigma}}(\tau) &= -\av{\op T \,
        \op c_{-\vec k -\sigma}^+(\tau) \,
        \op c_{\vec k \sigma}^+(0)} = [F_{\vec k \sigma}(\tau)]^*,
    \end{align}
    %
    as well as the phonon \name{Green} function
    %
    \begin{equation}
        D_{\vec q}(\tau) = -\av{\op T \,
        \op \upphi_{\vec q}(\tau) \,
        \op \upphi_{\vec q}(0)} = D_{-\vec q}(-\tau) \in \mathds R.
    \end{equation}
\end{subequations}

In the following, leading terms of the electronic \name{Green} functions are
calculated explicitly for some model interactions. First, the perturbation
series for the electron-phonon interaction as described by the
\name{Holstein-Hamilton} operator is analyzed up the second order. As will be
shown, the latter bears a close resemblance to the first order terms of the
homogeneous electron gas and the more simple \name{Hubbard} model of interacting
electrons.

For \name{Hamilton} operators which conserve the number of electrons, by which
e.g. free electrons but also all of the following models are described, the
\name{Gor'kov-Green} functions vanish. Nevertheless, in subsequent applications
of \name{Wick}'s theorem, unperturbed expectation values of \name{Cooper}-pair
creation and annihilation operators \emph{which originate from the interaction}
terms are kept, since in a further step they will be redefined with respect to
the interacting system, in which the strict conservation of particles is
dropped.

\subsection{\name{Holstein} model}
\label{Holstein model}

In real space, the interaction term of the \emph{\name{Holstein-Hamilton}
operator} reads \cite{vonderLindenBergerValasek95}
%
\begin{equation*}
    \op V = g \sum_{\vec R \sigma}
    \op c_{\vec R \sigma}^+ \op c_{\vec R \sigma}
    [\op b_{\vec R} + \op b_{\vec R}^+],
\end{equation*}
%
where $g$ is the local electron-phonon coupling strength and $\op c_{\vec R
\sigma}$ and $\op b_{\vec R}$ annihilate electronic and phononic excitations,
respectively, localized at the lattice sites $\vec R$.

A discrete \name{Fourier} transform of the creation and annihilation operators
with the help of Eq.~\ref{DFT orthogonality relation} yields the momentum-space
representation
%
\begin{equation*}
    \op V = \frac g {\sqrt N} \sum_{\vec k \vec k' \sigma \vec q}
    \frac 1 N \sum_{\vec R} \E^{\I (\vec k' - \vec k) \vec R}
    \op c_{\vec k' \sigma}^+ \op c_{\vec k \sigma} [
        \E^{-\I \vec q \vec R} \op b_{\vec q} +
        \E^{\I \vec q \vec R} \op b_{\vec q}^+
        ]
    = \frac g {\sqrt N} \sum_{\vec k \sigma \vec q}
    \op c_{\vec k + \vec q \sigma}^+ \op c_{\vec k \sigma}
    \upphi_{\vec q}.
\end{equation*}

\subsubsection{Normal \name{Green} function}

First, $G_{\vec k \sigma}(\tau)$ is evaluated. The zeroth term is just the
\name{Green} function of non-interacting electrons, the \name{Matsubara}
representation of which is given in Eq.~\ref{G_0(i omega_n)}. First-order terms
vanish because $\upphi_{\vec q}$ is does not conserve the particle number,
unlike the unperturbed \name{Hamilton} operator with respect to which the
averages are taken. Eventually, in second order one finds
%
\begin{align} \label{Holstein 2nd}
    G \super{2nd}_{\vec k \sigma}(\tau) &\propto
    \sum_{\vec k' \vec k'' \sigma \sigma' \vec q \vec q'}
    \int \from 0 \till \beta \D \tau' \int \from 0 \till \beta \D \tau'' \,
    \av \dots_0,
    \\ \notag
    \av \dots_0 &= \av{\op T \,
        \op c_{\vec k \sigma}[\tau] \,
        \op c_{\vec k' + \vec q \sigma'}^+[\tau'] \,
        \op c_{\vec k' \sigma'}[\tau'] \,
        \upphi_{\vec q}[\tau'] \,
        \op c_{\vec k'' + \vec q' \sigma''}^+[\tau''] \,
        \op c_{\vec k'' \sigma''}[\tau''] \,
        \upphi_{\vec q'}[\tau''] \,
        \op c_{\vec k \sigma}^+[0]}_0
    \\ \notag
    &= \underbrace{
            \av{\op T \,
                \op c_{\vec k \sigma}[\tau] \,
                \op c_{\vec k' + \vec q \sigma'}^+[\tau'] \,
                \op c_{\vec k' \sigma'}[\tau'] \,
                \op c_{\vec k'' + \vec q' \sigma''}^+[\tau''] \,
                \op c_{\vec k'' \sigma''}[\tau''] \,
                \op c_{\vec k \sigma}^+[0]}_0
            }_{\displaystyle \av{\text{el.}}_0}
        \underbrace{
            \av{\op T \,
                \upphi_{\vec q}[\tau'] \,
                \upphi_{\vec q'}[\tau'']}_0
            }_{\displaystyle -D^0_{\vec q}(\tau' - \tau'')},
\end{align}
%
where the \name{Green} function $D^0_{\vec q}(\tau)$ of free phonons can be
transformed into Eq.~\ref{D_0(i omega_n)}. Further application of
\textsc{Wick}'s theorem as stated in Eq.~\ref{Wick theorem} to the electronic
part $\av{\text{el.}}_0$ yields
%
\begingroup
    \def\minalignsep{0pt}
    \begin{align*}
        &-
        && \av{\op T \, \op c_{\vec k \sigma}[\tau]
        \, \op c_{\vec k'' + \vec q' \sigma''}^+[\tau'']}_0
        && \av{\op T \, \op c_{\vec k' \sigma'}[\tau']
        \, \op c_{\vec k' + \vec q \sigma'}^+[\tau']}_0
        && \av{\op T \, \op c_{\vec k'' \sigma''}[\tau'']
        \, \op c_{\vec k \sigma}^+[0]}_0
        && \delta_{\vec q}^0 \delta_{\vec q'}^0
        && \delta_{\sigma''}^\sigma
        && \delta_{\vec k''}^{\vec k} \\
        &-
        && \av{\op T \, \op c_{\vec k \sigma}[\tau]
        \, \op c_{\vec k' + \vec q \sigma'}^+[\tau']}_0
        && \av{\op T \, \op c_{\vec k'' \sigma''}[\tau'']
        \, \op c_{\vec k'' + \vec q' \sigma''}^+[\tau'']}_0
        && \av{\op T \, \op c_{\vec k' \sigma'}[\tau']
        \, \op c_{\vec k \sigma}^+[0]}_0
        && \delta_{\vec q}^0 \delta_{\vec q'}^0
        && \delta_{\sigma'}^\sigma
        && \delta_{\vec k'}^{\vec k} \\
        &+
        && \av{\op T \, \op c_{\vec k \sigma}[\tau]
        \, \op c_{\vec k' + \vec q \sigma'}^+[\tau']}_0
        && \av{\op T \, \op c_{\vec k' \sigma'}[\tau']
        \, \op c_{\vec k'' + \vec q' \sigma''}^+[\tau'']}_0
        && \av{\op T \, \op c_{\vec k'' \sigma''}[\tau'']
        \, \op c_{\vec k \sigma}^+[0]}_0
        && \delta_{\vec q'}^{-\vec q}
        && \delta_{\sigma'}^\sigma \delta_{\sigma''}^\sigma
        && \delta_{\vec k''}^{\vec k} \delta_{\vec k'}^{\vec k - \vec q} \\
        &+
        && \av{\op T \, \op c_{\vec k \sigma}[\tau]
        \, \op c_{\vec k'' + \vec q' \sigma''}^+[\tau'']}_0
        && \av{\op T \, \op c_{\vec k'' \sigma''}[\tau'']
        \, \op c_{\vec k' + \vec q \sigma'}^+[\tau']}_0
        && \av{\op T \, \op c_{\vec k' \sigma'}[\tau']
        \, \op c_{\vec k \sigma}^+[0]}_0
        && \delta_{\vec q'}^{-\vec q}
        && \delta_{\sigma'}^\sigma \delta_{\sigma''}^\sigma
        && \delta_{\vec k'}^{\vec k} \delta_{\vec k''}^{\vec k + \vec q} \\
        &+
        && \av{\op T \, \op c_{\vec k' \sigma'}[\tau']
        \, \op c_{\vec k' + \vec q \sigma'}^+[\tau']}_0
        && \av{\op T \, \op c_{\vec k'' \sigma''}[\tau'']
        \, \op c_{\vec k'' + \vec q' \sigma''}^+[\tau'']}_0
        && \av{\op T \, \op c_{\vec k \sigma}[\tau]
        \, \op c_{\vec k \sigma}^+[0]}_0
        && \delta_{\vec q}^0 \delta_{\vec q'}^0
        &&
        && \\
        &-
        && \av{\op T \, \op c_{\vec k' \sigma'}[\tau']
        \, \op c_{\vec k'' + \vec q' \sigma''}^+[\tau'']}_0
        && \av{\op T \, \op c_{\vec k'' \sigma''}[\tau'']
        \, \op c_{\vec k' + \vec q \sigma'}^+[\tau']}_0
        && \av{\op T \, \op c_{\vec k \sigma}[\tau]
        \, \op c_{\vec k \sigma}^+[0]}_0
        && \delta_{\vec q'}^{-\vec q}
        && \delta_{\sigma''}^{\sigma'}
        && \delta_{\vec k''}^{\vec k' + \vec q} \\
        &+
        && \av{\op T \, \op c_{\vec k' \sigma'}[\tau']
        \, \op c_{\vec k'' \sigma''}[\tau'']}_0
        && \av{\op T \, \op c_{\vec k'' + \vec q' \sigma''}^+[\tau'']
        \, \op c_{\vec k' + \vec q \sigma'}^+[\tau']}_0
        && \av{\op T \, \op c_{\vec k \sigma}[\tau]
        \, \op c_{\vec k \sigma}^+[0]}_0
        && \delta_{\vec q'}^{-\vec q}
        && \delta_{\sigma''}^{-\sigma'}
        && \delta_{\vec k''}^{-\vec k'}.
    \end{align*}
\endgroup

\begin{figure}
    \small
    \medmuskip=0mu
    \begin{subfigure}[b]{\linewidth/3}
        \centering
        %!TEX root = ../thesis.tex
%
\begin{tikzpicture}
    \coordinate (i) at (0, 0);
    \coordinate (2) at (2, 0);
    \coordinate (1) at (2, 4/3);
    \coordinate (o) at (4, 0);
    %
    \draw [phonon] (2) -- node [right] {$0$} (1);
    %
    \draw [electron] (i) -- node [below] {$\vec k \sigma$} (2);
    \draw [electron] (2) -- node [below] {$\vec k \sigma$} (o);
    %
    \draw [electron] (1) arc (-90:270:5mm)
        node [yshift=5mm] {$\vec k' \sigma'$};
    %
    \foreach \point in {1, 2} \fill (\point) circle (1pt);
    %
    \node [below] at (i) {$0$};
    \node [above left] at (2) {$\tau''$};
    \node [below left] at (1) {$\tau'$};
    \node [below] at (o) {$\tau$};
    %
    \useasboundingbox ([yshift=1mm] current bounding box.north east);
\end{tikzpicture}%

        \caption{``\name{Hartree}''}
        \label{Hartree}
    \end{subfigure}%
    \begin{subfigure}[b]{\linewidth/3}
        \centering
        %!TEX root = ../thesis.tex
%
\begin{tikzpicture}[bend angle=45]
    \coordinate (i) at (0, 0);
    \coordinate (2) at (4/3, 4/3);
    \coordinate (t) at (6/3, 6/3);
    \coordinate (1) at (8/3, 4/3);
    \coordinate (o) at (4, 0);
    %
    \draw [phonon] (2) -- node [below] {$\vec q$} (1);
    %
    \draw [electron] (i) to [bend right]
        node [above left] {$\vec k \sigma$} (2);
    \draw [electron] (2) to [bend left] (t)
        node [above] {$\vec k - \vec q \sigma$} to [bend left] (1);
    \draw [electron] (1) to [bend right]
        node [above right] {$\vec k \sigma$} (o);
    %
    \foreach \point in {1, 2} \fill (\point) circle (1pt);
    %
    \node [below] at (i) {$0$};
    \node [left] at (2) {$\tau''$};
    \node [right] at (1) {$\tau'$};
    \node [below] at (o) {$\tau$};
\end{tikzpicture}%

        \caption{``\name{Fock}''}
        \label{Fock}
    \end{subfigure}%
    \begin{subfigure}[b]{\linewidth/3}
        \centering
        %!TEX root = ../thesis.tex
%
\tikzsetnextfilename{fock-anomalous}
%
\begin{tikzpicture}[bend angle=45]
    \coordinate (i) at (0, 0);
    \coordinate (2) at (4/3, 4/3);
    \coordinate (t) at (6/3, 6/3);
    \coordinate (1) at (8/3, 4/3);
    \coordinate (o) at (4, 0);

    \draw [phonon] (2) -- node [below] {$\vec q$} (1);

    \draw [electron] (2) to [bend left]
        node [above left] {$-\vec k -\sigma$} (i);
    \draw [electron] (t) to [bend right]
        node [above left] {$\vec q - \vec k -\sigma$} (2);
    \draw [electron] (t) to [bend left]
        node [above right] {$\vec k - \vec q \sigma$} (1);
    \draw [electron] (1) to [bend right]
        node [above right] {$\vec k \sigma$} (o);

    \foreach \point in {1, 2} \fill (\point) circle (1pt);

    \node [below] at (i) {$0$};
    \node [left] at (2) {$\tau''$};
    \node [right] at (1) {$\tau'$};
    \node [below] at (o) {$\tau$};
\end{tikzpicture}%
        \caption{``\name{Fock}'', anomalous}
        \label{Fock, anomalous}
    \end{subfigure}\\[1cm]
    \begin{subfigure}[b]{\linewidth/3}
        \centering
        % !TEX root = ../thesis.tex
%
\tikzsetnextfilename{glasses}
%
\begin{tikzpicture}
    \coordinate (i) at (0, 0);
    \coordinate (2) at (4/3, 4/3);
    \coordinate (1) at (8/3, 4/3);
    \coordinate (o) at (4, 0);

    \draw [phonon] (2) -- node [below] {$0$} (1);

    \draw [electron] (i) -- node [below] {$\vec k \sigma$} (o);

    \draw [electron] (2) arc (0:360:5mm)
        node [xshift=-5mm] {$\vec k'' \sigma''$};
    \draw [electron] (1) arc (180:540:0.5)
        node [xshift=5mm] {$\vec k' \sigma'$};

    \foreach \point in {1, 2} \fill (\point) circle (1pt);

    \node [below] at (i) {$0$};
    \node [above right] at (2) {$\tau''$};
    \node [above left] at (1) {$\tau'$};
    \node [below] at (o) {$\tau$};
\end{tikzpicture}%

        \caption{``glasses''}
        \label{glasses}
    \end{subfigure}%
    \begin{subfigure}[b]{\linewidth/3}
        \centering
        %!TEX root = ../thesis.tex
%
\begin{tikzpicture}[bend angle=45]
    \coordinate (i) at (0, 0);
    \coordinate (2) at (4/3, 4/3);
    \coordinate (t) at (6/3, 6/3);
    \coordinate (b) at (6/3, 2/3);
    \coordinate (1) at (8/3, 4/3);
    \coordinate (o) at (4, 0);

    \draw [phonon] (2) -- node [below] {$\vec q$} (1);

    \draw [electron] (i) -- node [below] {$\vec k \sigma$} (o);

    \draw [electron] (2) to [bend right] (b)
        node [below] {$\vec k' \sigma'$} to [bend right] (1);
    \draw [electron] (1) to [bend right] (t)
        node [above] {$\vec k' + \vec q \sigma''$} to [bend right] (2);

    \foreach \point in {1, 2} \fill (\point) circle (1pt);

    \node [below] at (i) {$0$};
    \node [left] at (2) {$\tau''$};
    \node [right] at (1) {$\tau'$};
    \node [below] at (o) {$\tau$};
\end{tikzpicture}%

        \caption{``porthole''}
        \label{porthole}
    \end{subfigure}%
    \begin{subfigure}[b]{\linewidth/3}
        \centering
        %!TEX root = ../thesis.tex
%
\tikzsetnextfilename{porthole-anomalous}
%
\begin{tikzpicture}[bend angle=45]
    \coordinate (i) at (0, 0);
    \coordinate (2) at (4/3, 4/3);
    \coordinate (t) at (6/3, 6/3);
    \coordinate (b) at (6/3, 2/3);
    \coordinate (1) at (8/3, 4/3);
    \coordinate (o) at (4, 0);

    \draw [electron] (i) -- node [below] {$\vec k \sigma$} (o);

    \draw [phonon] (2) -- node [below] {$\vec q$} (1);

    \draw [electron] (t) to [bend left]
        node [above right] {$\vec k' \sigma'$} (1);
    \draw [electron] (t) to [bend right]
        node [above left] {$-\vec k' -\sigma'$} (2);

    \draw [electron] (1) to [bend left]
        node [below right] {$\vec k' + \vec q \sigma'$} (b);
    \draw [electron] (2) to [bend right]
        node [below left] {$-\vec k' -\vec q -\sigma'$} (b);

    \foreach \point in {1, 2} \fill (\point) circle (1pt);

    \node [below] at (i) {$0$};
    \node [left] at (2) {$\tau''$};
    \node [right] at (1) {$\tau'$};
    \node [below] at (o) {$\tau$};
\end{tikzpicture}%

        \caption{``porthole'', anomalous}
        \label{porthole, anomalous}
    \end{subfigure}
    \caption[\name{Feynman} diagrams: 2nd-order electron-phonon interaction]
        {Second-order processes that occur in the \textsc{Feynman-Dyson}
         perturbation series of $G_{\vec k \sigma}$ and $F_{\vec k \sigma}$ in
         the \textsc{Holstein} model. The quantum numbers $\vec k'$, $\vec k''$,
         $\sigma'$, $\sigma''$ and $\vec q$ of internal lines are summed, time
         parameters $\tau'$ and $\tau''$ of internal vertices integrated over.
         Solid and wavy lines represent electrons and phonons, respectively. The
         convention is followed that arrows point from creation towards
         annihilation. Thus out- or inward double arrows stand for \name{Cooper}
         pairs of unspecified origin or destiny.}
    \label{Feynman diagrams}
\end{figure}

Substituting the corresponding non-interacting \textsc{Matsubara-Green}
functions and performing the summation to eliminate the \name{Kronecker} deltas,
one obtains
%
\begin{align*}
    \sum_{\smash[b]{\begin{smallmatrix*}[l]
        \vec k' & \vec k'' \\
        \sigma' & \sigma'' \\
        \vec q & \vec q'
    \end{smallmatrix*}}}
    \av \dots_0 =
    %
    \let\displaystyle\textstyle
    %
    \begin{aligned}[t]
        &- \sum_{\vec k' \sigma'}
        && G^0_{\vec k \sigma}(\tau - \tau'')
        && G^0_{\vec k' \sigma'}(0)
        && G^0_{\vec k \sigma}(\tau'')
        && D^0_0(\tau' - \tau'')
        &  \text{(\ref{Hartree})} \\
        &- \sum_{\vec k'' \sigma''}
        && G^0_{\vec k \sigma}(\tau - \tau')
        && G^0_{\vec k'' \sigma''}(0)
        && G^0_{\vec k \sigma}(\tau')
        && D^0_0(\tau' - \tau'')
        &  \\
        &+ \sum_{\vec q}
        && G^0_{\vec k \sigma}(\tau - \tau')
        && G^0_{\vec k - \vec q \sigma}(\tau' - \tau'')
        && G^0_{\vec k \sigma}(\tau'')
        && D^0_{\vec q}(\tau' - \tau'')
        &  \text{(\ref{Fock})} \\
        &+ \sum_{\vec q}
        && G^0_{\vec k \sigma}(\tau - \tau'')
        && G^0_{\vec k + \vec q \sigma}(\tau'' - \tau')
        && G^0_{\vec k \sigma}(\tau')
        && D^0_{\vec q}(\tau' - \tau'')
        &  \\
        &+ \textstyle \sum_{\vec k' \vec k'' \sigma' \sigma''}
        && G^0_{\vec k' \sigma'}(0)
        && G^0_{\vec k'' \sigma''}(0)
        && G^0_{\vec k \sigma}(\tau)
        && D^0_0(\tau' - \tau'')
        &  \text{(\ref{glasses})} \\
        &- \sum_{\vec k' \sigma' \vec q}
        && G^0_{\vec k' \sigma'}(\tau' - \tau'')
        && G^0_{\vec k' + \vec q \sigma'}(\tau'' - \tau')
        && G^0_{\vec k \sigma}(\tau)
        && D^0_{\vec q}(\tau' - \tau'')
        &  \text{(\ref{porthole})} \\
        &+ \sum_{\vec k' \sigma' \vec q}
        && F^0_{\vec k' \sigma'}(\tau' - \tau'')
        && \smash{\widetilde F^0_{\vec k' + \vec q \sigma'}}(\tau'' - \tau')
        && G^0_{\vec k \sigma}(\tau)
        && D^0_{\vec q}(\tau' - \tau'').
        &  \text{(\ref{porthole, anomalous})}
    \end{aligned}
\end{align*}
%
The corresponding \textsc{Feynman} diagrams are depicted in Fig.~\ref{Feynman
diagrams}.

The last three terms represent \emph{disconnected diagrams}. Calculating the
second order of $\av{S(\beta)}_0$, which has been ignored so far, yields these
very diagrams, except that the in- and outgoing electron lines are missing. It
turns out that every diagram part contained in the denominator $\av{S(\beta)}_0$
of Eq.~\ref{perturbation series} will reappear in the nominator in a way that
%
\begin{equation*}
    \av{\op T \, \op A[\tau] \, \op S(\beta) \, \op B[0]}_0 =
    \av{\op S(\beta)}_0
    \av{\op T \, \op A[\tau] \, \op S(\beta) \, \op B[0]}_0 \super{c.}
\end{equation*}
%
where $\av \dots_0 \super{c.}$ is $\av \dots_0$ less all terms corresponding to
disconnected diagrams. Hence,
%
\begin{equation*}
    G(\tau) = -\av{\op T \, \op A[\tau] \, \op S(\beta) \, \op B[0]}_0
    \super{c.}.
\end{equation*}

In the following, all terms with $D^0_0(\tau' - \tau'')$ are neglected since
phonons with $\vec q = 0$ are no actual phonons but rather translations of the
crystal as a whole or a permanent strain \cite[p.~82]{Mahan00}, which shall not
be considered.

Performing the integrals, the remaining two terms turn out to be equivalent
since $\tau'$ and $\tau''$ occur in exchangeable positions. Transforming to
\textsc{Matsubara} frequencies as per Eq.~\ref{Matsubara-Green function} yields
%
\begin{align*}
    G \super{2nd}_{\vec k \sigma}(\I \omega_n)
    &= \int \from 0 \till \beta \D \tau \,
    \E^{\I \omega_n \tau} G \super{2nd}_{\vec k \sigma}(\tau) \\
    &= -\frac {g^2} N \sum_{\vec q}
    \int \from 0 \till \beta \D \tau
    \int \from 0 \till \beta \D \tau'
    \int \from 0 \till \beta \D \tau'' \,
    \E^{\I \omega_n \tau}
    G^0_{\vec k \sigma}(\tau - \tau') \,
    G^0_{\vec k - \vec q \sigma}(\tau' - \tau'') \,
    G^0_{\vec k \sigma}(\tau'') \,
    D^0_{\vec q}(\tau' - \tau'') \\
    &= -\frac {g^2}{N \beta} \sum_{\vec q} \sum_{i j k m}
    G^0_{\vec k \sigma}(\I \omega_i) \,
    G^0_{\vec k - \vec q \sigma}(\I \omega_m) \,
    G^0_{\vec k \sigma}(\I \omega_j) \,
    D^0_{\vec q}(\I \nu_k) \times \dots \\
    \dots &\times
    \frac 1 \beta \int \from 0 \till \beta \D \tau \,
    \E^{\I (\omega_n - \omega_i) \tau} \,
    \frac 1 \beta \int \from 0 \till \beta \D \tau' \,
    \E^{\I (\omega_i - \omega_m - \nu_k) \tau'} \,
    \frac 1 \beta \int \from 0 \till \beta \D \tau'' \,
    \E^{\I (\omega_m - \omega_j + \nu_k) \tau''}.
\end{align*}
%
Noting that differences of \name{Matsubara} frequencies of mixed and equal type
are fermionic and bosonic, respectively, and that for bosons the
\name{Matsubara-Fourier} series is defined as an ordinary \name{Fourier} series,
one can apply the orthogonality relation given in Eq.~\ref{Fourier series
orthogonality relation}. Thus
%
\begin{equation*}
    G \super{2nd}_{\vec k \sigma}(\I \omega_n) =
    G^0_{\vec k \sigma}(\I \omega_n) \,
    \Sigma^0_{\vec k \sigma}(\I \omega_n) \,
    G^0_{\vec k \sigma}(\I \omega_n),
\end{equation*}
%
where the part in between is a \emph{self-energy contribution} and given by
%
\begin{equation*}
    \Sigma^0_{\vec k \sigma}(\I \omega_n)
    = -\frac {g^2}{N \beta} \sum_{\vec q m}
    G^0_{\vec k - \vec q \sigma}(\I \omega_m) \,
    D^0_{\vec q}(\I \omega_n - \I \omega_m)
    = -\frac {g^2}{N \beta} \sum_{\vec q m}
    G^0_{\vec q \sigma}(\I \omega_m) \,
    D^0_{\vec k - \vec q}(\I \omega_n - \I \omega_m).
\end{equation*}

\subsubsection{\name{Gor'kov-Green} function}

Next, $F_{\vec k \sigma}(\tau)$ is analyzed. The procedure is as above except
that
%
\begin{align*}
    \av{\text{el.}}_0= \av{\op T \,
        \op c_{\vec k \sigma}[\tau] \,
        \op c_{\vec k' + \vec q \sigma'}^+[\tau'] \,
        \op c_{\vec k' \sigma'}[\tau'] \,
        \op c_{\vec k'' + \vec q' \sigma''}^+[\tau''] \,
        \op c_{\vec k'' \sigma''}[\tau''] \,
        \op c_{-\vec k -\sigma}[0]}_0.
\end{align*}
%
has to be considered. With the help of \textsc{Wick}'s theorem this reduces to
%
\begingroup
    \def\minalignsep{0pt}
    \begin{align*}
        &-
        && \av{\op T \, \op c_{\vec k \sigma}[\tau]
        \, \op c_{\vec k' + \vec q \sigma'}^+[\tau']}_0
        && \av{\op T \, \op c_{\vec k' \sigma'}[\tau']
        \, \op c_{\vec k'' \sigma''}[\tau'']}_0
        && \av{\op T \, \op c_{\vec k'' + \vec q' \sigma''}^+[\tau'']
        \, \op c_{-\vec k -\sigma}[0]}_0
        && \delta_{\vec q'}^{-\vec q}
        && \delta_{\sigma'}^\sigma \delta_{\sigma''}^{-\sigma}
        && \delta_{\vec k'}^{\vec k - \vec q}
           \delta_{\vec k''}^{\vec q - \vec k} \\
        &-
        && \av{\op T \, \op c_{\vec k \sigma}[\tau]
        \, \op c_{\vec k'' + \vec q' \sigma''}^+[\tau'']}_0
        && \av{\op T \, \op c_{\vec k'' \sigma''}[\tau'']
        \, \op c_{\vec k' \sigma'}[\tau']}_0
        && \av{\op T \, \op c_{\vec k' + \vec q \sigma'}^+[\tau']
        \, \op c_{-\vec k -\sigma}[0]}_0
        && \delta_{\vec q'}^{-\vec q}
        && \delta_{\sigma'}^{-\sigma} \delta_{\sigma''}^\sigma
        && \delta_{\vec k'}^{-\vec k - \vec q}
           \delta_{\vec k''}^{\vec k + \vec q}.
    \end{align*}
\endgroup
%
Substituting \textsc{Green}'s functions, one finds two contributions which again
will prove equivalent:
%
\begin{align*}
    \sum_{\mathclap{\vec k' \vec k'' \sigma' \sigma'' \vec q \vec q'}}
    \av \dots_0 =
    %
    \let\displaystyle\textstyle
    %
    \begin{aligned}[t]
        &- \sum_{\vec q}
        && G^0_{\vec k \sigma}(\tau - \tau')
        && F^0_{\vec k - \vec q \sigma}(\tau' - \tau'')
        && \smash{\widetilde G^0_{\vec k \sigma}}(\tau'')
        && D^0_{\vec q}(\tau' - \tau'')
        &  \text{(Fig.~\ref{Fock, anomalous})} \\
        &- \sum_{\vec q}
        && G^0_{\vec k \sigma}(\tau - \tau'')
        && F^0_{\vec k + \vec q \sigma}(\tau'' - \tau')
        && \smash{\widetilde G^0_{\vec k \sigma}}(\tau')
        && D^0_{\vec q}(\tau' - \tau'').
        &
    \end{aligned}
\end{align*}
%
Bar the sign, the final result formally resembles the the one for $G
\super{2nd}_{\vec k \sigma}(\I \omega_n)$:
%
\begin{equation*}
    F \super{2nd}_{\vec k \sigma}(\I \omega_n)
    = G^0_{\vec k \sigma}(\I \omega_n) \,
    \Sigma^0_{\vec k \sigma}(\I \omega_n) \,
    \widetilde G^0_{\vec k \sigma}(\I \omega_n)
    \quad \text{with} \quad
    \Sigma^0_{\vec k \sigma}(\I \omega_n)
    = \frac {g^2}{N \beta} \sum_{\vec q m}
    F^0_{\vec q \sigma}(\I \omega_m) \,
    D^0_{\vec k - \vec q}(\I \omega_n - \I \omega_m)
\end{equation*}

\subsubsection{Auxiliary \name{Green} functions}

For completeness, the corresponding results for $\widetilde G_{\vec k \sigma}(\I
\omega_n)$ and $\widetilde F_{\vec k \sigma}(\I \omega_n)$ shall be derived as
well. From the properties given in Eqs.~\ref{Green functions of interest} it
follows that $\widetilde G_{\vec k \sigma}(\I \omega_n) = -G_{-\vec k
-\sigma}(-\I \omega_n)$, $[G_{\vec k \sigma}(\I \omega_n)]^* = G_{\vec k
\sigma}(-\I \omega_n)$ and $[F_{\vec k \sigma}(\I \omega_n)]^* = \widetilde
F_{\vec k \sigma}(-\I \omega_n)$ as well as $D_{\vec q}(\I \omega_n) = D_{-\vec
q}(-\I \omega_n)$. Thus
%
\begin{align*}
    \widetilde G \super{2nd}_{\vec k \sigma}(\I \omega_n)
    &= -G \super{2nd}_{-\vec k -\sigma}(-\I \omega_n) &&
    \\
    &= \widetilde G^0_{\vec k \sigma}(\I \omega_n) \,
    \Sigma^0_{\vec k \sigma}(\I \omega_n) \,
    \widetilde G^0_{\vec k \sigma}(\I \omega_n),
    &
    \Sigma^0_{\vec k \sigma}(\I \omega_n)
    &= -\frac {g^2}{N \beta} \sum_{\vec q m}
    \widetilde G^0_{\vec q \sigma}(\I \omega_m) \,
    D^0_{\vec k - \vec q}(\I \omega_n - \I \omega_m),
    \\
    \widetilde F \super{2nd}_{\vec k \sigma}(\I \omega_n)
    &= [F \super{2nd}_{\vec k \sigma}(-\I \omega_n)]^* &&
    \\
    &= \widetilde G^0_{\vec k \sigma}(\I \omega_n) \,
    \Sigma^0_{\vec k \sigma}(\I \omega_n) \,
    G^0_{\vec k \sigma}(\I \omega_n),
    &
    \Sigma^0_{\vec k \sigma}(\I \omega_n)
    &= +\frac {g^2}{N \beta} \sum_{\vec q m}
    \widetilde F^0_{\vec q \sigma}(\I \omega_m) \,
    D^0_{\vec k - \vec q}(\I \omega_n - \I \omega_m).
\end{align*}

\subsection{Homogeneous electron gas}

The interaction within an \emph{homogenous electron gas} reads
\cite[165]{Czycholl08}
%
\begin{align*}
    \op V = \frac 1 2 \sum_{\vec k \vec k' \sigma \sigma' \vec q} U_{\vec q} \,
    \op c_{\vec k + \vec q \sigma}^+
    \op c_{\vec k' - \vec q \sigma'}^+
    \op c_{\vec k' \sigma'}
    \op c_{\vec k \sigma}
    \quad \text{with} \quad
    U_{\vec q} = \frac{e^2}{V} \frac{4 \pi}{q^2}.
\end{align*}

\subsubsection{Normal \name{Green} function}

The first-order term of the \textsc{Feynman-Dyson} perturbation series for
$G_{\vec k \sigma}(\tau)$ is thus
%
\begin{align*}
    G \super{1st}_{\vec k \sigma}(\tau) &= \frac 1 2
    \sum_{\vec k' \vec k'' \sigma' \sigma'' \vec q} U_{\vec q}
    \int \from 0 \till \beta \D \tau' \, \av \dots_0 \super{c.},
    \\
    \av \dots_0 &= \av{\op T \,
        \op c_{\vec k \sigma}[\tau] \,
        \op c_{\vec k' + \vec q \sigma'}^+[\tau'] \,
        \op c_{\vec k'' - \vec q \sigma''}^+[\tau'] \,
        \op c_{\vec k'' \sigma''}[\tau'] \,
        \op c_{\vec k' \sigma'}[\tau'] \,
        \op c_{\vec k \sigma}^+[0]
    }_0 \\ &= \av{\op T \,
        \op c_{\vec k \sigma}[\tau] \,
        \op c_{\vec k' + \vec q \sigma'}^+[\tau'] \,
        \op c_{\vec k' \sigma'}[\tau'] \,
        \op c_{\vec k'' - \vec q \sigma''}^+[\tau'] \,
        \op c_{\vec k'' \sigma''}[\tau'] \,
        \op c_{\vec k \sigma}^+[0]
    }_0.
\end{align*}
%
A comparison with Eq.~\ref{Holstein 2nd}, the corresponding formula for the
\name{Holstein} model, reveals that
%
\begin{align*}
    \av \dots_0 =
    \sum_{\vec q'} \delta_{\vec q'}^{-\vec q}
    \int \from 0 \till \beta \D \tau'' \,
    \delta(\tau'' - \tau'_+) \, \av{\text{el.}}_0,
\end{align*}
%
where $\tau'_+$ corresponds to a time infinitesimally later than $\tau'$ in
order to ensure a causal time-ordering. The \textsc{Kronecker} delta is actually
superflous for being contained in each term of $\av{\text{el.}}_0$ anyway. Once
the factorized expression has been rewritten in terms of \name{Green} functions,
the subscript plus sign can be dropped again. One finds
%
\begin{align*}
    \sum_{\mathclap{\vec k' \vec k'' \sigma' \sigma'' \vec q}}
    U_{\vec q} \av \dots_0 \super{c.} =
    2 G^0_{\vec k \sigma}(\tau - \tau')
    \hyper{\Big[
        U_0 \sum_{\vec k' \sigma'} G^0_{\vec k' \sigma'}(0) -
        \sum_{\vec q} U_{\vec q} \, G^0_{\vec k - \vec q \sigma}(0)
    \Big]}{\equiv \Lambda_{\vec k \sigma}(0)}
    G^0_{\vec k \sigma}(\tau').
\end{align*}

As a function of \name{Matsubara} frequencies one has
%
\begin{align*}
    G \super{1st}_{\vec k \sigma}(\I \omega_n) &=
    \int \from 0 \till \beta \D \tau \, \E^{\I \omega_n \tau}
    G \super{1st}_{\vec k \sigma}(\tau) =
    \int \from 0 \till \beta \D \tau
    \int \from 0 \till \beta \D \tau' \,
    \E^{\I \omega_n \tau}
    G^0_{\vec k \sigma}(\tau - \tau') \,
    \Lambda_{\vec k \sigma}(0) \,
    G^0_{\vec k \sigma}(\tau')
    \\
    &= \frac 1 \beta \sum_{i j m}
    G^0_{\vec k \sigma}(\I \omega_i) \,
    \Lambda_{\vec k \sigma}(\I \omega_m) \,
    G^0_{\vec k \sigma}(\I \omega_j) \,
    \frac 1 \beta \int \from 0 \till \beta \D \tau \,
    \E^{\I (\omega_n - \omega_i) \tau} \,
    \frac 1 \beta \int \from 0 \till \beta \D \tau' \,
    \E^{\I (\omega_i - \omega_j) \tau'}
    \\
    &= G^0_{\vec k \sigma}(\I \omega_n) \,
    \Sigma^0_{\vec k \sigma} \,
    G^0_{\vec k \sigma}(\I \omega_n)
    \quad \text{with} \quad
    \Sigma^0_{\vec k \sigma}
    = \frac {U_0} \beta \sum_{\vec k' \sigma' m}
    G^0_{\vec k' \sigma'}(\I \omega_m)
    - \frac 1 \beta \sum_{\vec q m} U_{\vec k - \vec q}
    G^0_{\vec q \sigma}(\I \omega_m).
\end{align*}
%
The two terms in $\Sigma^0_{\vec k \sigma}$, which is independent of frequency
because the \name{Coulomb} interaction is unscreened and assumed to be
instantaneous, are the leading \textsc{Hartree} and \textsc{Fock} self-energy
contributions. The former may be compensated exactly by a homogenous positive
background, which is done in the so-called \emph{jellium model}
\cite[182]{Czycholl08}.

\subsubsection{\name{Gor'kov-Green} function}

Analogously, during the calculation of $F^0_{\vec k \sigma}(\tau)$ one obtains
%
\begin{align*}
    \sum_{\mathclap{\vec k' \vec k'' \sigma' \sigma'' \vec q}}
    \av \dots_0 \super{c.}
    = 2 \sum_{\vec q} U_{\vec q} \,
    G^0_{\vec k \sigma}(\tau - \tau') \,
    F^0_{\vec k - \vec q \sigma}(0) \,
    \widetilde G^0_{\vec k \sigma}(\tau').
\end{align*}
%
As in the case of the \name{Holstein} interaction, there is no anomalous
\name{Hartree} contribution and, apart from the sign the anomalous \name{Fock}
contribution formally resembles the normal one:
%
\begin{equation*}
    F \super{1st}_{\vec k \sigma}(\I \omega_n)
    = G^0_{\vec k \sigma}(\I \omega_n) \,
    \Sigma^0_{\vec k \sigma} \,
    \widetilde G^0_{\vec k \sigma}(\I \omega_n)
    \quad \text{with} \quad
    \Sigma^0_{\vec k \sigma}
    = \frac 1 \beta \sum_{\vec q m} U_{\vec k - \vec q}
    F^0_{\vec q \sigma}(\I \omega_m).
\end{equation*}

\subsection{\name{Hubbard} model}
\label{Hubbard model}

The \textsc{Hubbard} model further restricts the \textsc{Coulomb} interaction to
occur only between electrons at the same site, which must consequently have
opposite spins because of the \name{Pauli} principle. With the on-site
\textsc{Coulomb} repulsion $U$, the corresponding operator in real space reads
\cite{vonderLindenBergerValasek95}
%
\begin{equation*}
    \op V = U \sum_{\vec R}
    \op c_{\vec R \uparrow}^+
    \op c_{\vec R \downarrow}^+
    \op c_{\vec R \downarrow}
    \op c_{\vec R \uparrow}.
\end{equation*}

As for the \name{Holstein} model, a discrete \name{Fourier} transform using
Eq.~\ref{DFT orthogonality relation} is applied, yielding
%
\begin{align*}
    \op V &=
    \frac U N \sum_{\vec k \vec k' \vec q \vec q'}
    \frac 1 N \sum_{\vec R}
    \E^{\I (\vec q + \vec{q'} - \vec k - \vec{k'}) \vec R}
    \op c_{\vec q' \uparrow}^+
    \op c_{\vec q \downarrow}^+
    \op c_{\vec k' \downarrow}
    \op c_{\vec k \uparrow}
    \\
    &= \frac U N \sum_{\vec k \vec k' \vec q}
    \op c_{\vec k + \vec k' - \vec q \uparrow}^+
    \op c_{\vec q \downarrow}^+
    \op c_{\vec k' \downarrow}
    \op c_{\vec k \uparrow}
    = \frac U N \sum_{\vec k \vec k' \vec q}
    \op c_{\vec k + \vec q \uparrow}^+
    \op c_{\vec k' - \vec q \downarrow}^+
    \op c_{\vec k' \downarrow} \op c_{\vec k \uparrow}.
\end{align*}

\subsubsection{Normal \name{Green} function}

In terms of $\av{\text{el.}}_0$ from Eq.~\ref{Holstein 2nd}, the first order of
the perturbation series for $G_{\vec k \sigma}(\tau)$ reads
%
\begin{equation*}
    G \super{1st}_{\vec k \sigma}(\tau) =
    \frac U N \sum_{\vec k' \vec k'' \sigma' \sigma'' \vec q \vec q'}
    \delta_{\sigma'}^\up \delta_{\sigma''}^\down \delta_{\vec q'}^{-\vec q}
    \int \from 0 \till \beta \D \tau' \,
    \int \from 0 \till \beta \D \tau'' \,
    \delta(\tau'' - \tau'_+) \,
    \av{\text{el.}}_0 \super{c.}.
\end{equation*}
%
Since interactions between electrons with the same spin are not considered in
this model, an electron cannot interact with itself and thus there is no
\name{Fock} contribution. The \name{Hartree} part is
%
\begin{align*}
    G \super{1st}_{\vec k \sigma}(\I \omega_n)
    &= G^0_{\vec k \sigma}(\I \omega_n) \,
    \Sigma^0_\sigma \,
    G^0_{\vec k \sigma}(\I \omega_n)
    \quad \text{with} \quad
    \Sigma^0_\sigma
    = \frac U {N \beta} \sum_{\vec k' m} G^0_{\vec k' -\sigma}(\I \omega_m).
\end{align*}

\subsubsection{\name{Gor'kov-Green} function}

Nevertheless, the analogous calculation of $F_{\vec k \sigma}(\tau)$ yields the
usual anomalous \name{Fock} contribution
%
\begin{equation*}
    F \super{1st}_{\vec k \sigma}(\I \omega_n)
    = G^0_{\vec k \sigma}(\I \omega_n) \,
    \Sigma^0_\sigma \,
    \widetilde G^0_{\vec k \sigma}(\I \omega_n)
    \quad \text{with} \quad
    \Sigma^0_\sigma
    = \frac U {N \beta} \sum_{\vec q m} F^0_{\vec q \sigma}(\I \omega_m).
\end{equation*}

\section{Self-energy}

The full \name{Green} function is found by adding \emph{all} diagrams, that are
properly connected. But which are valid connections? In Fig.~\ref{Feynman
diagrams}, each diagram features three fermion lines, a single boson line and
two vertices at each of which one fermion is annihilated and another created
under the influence of one boson. This process is simultaneous and conserves
both momentum and spin. Leaving the anomalous lines
%
\tikz [baseline=-0.5ex] \draw [ inward] (0, 0) -- (1, 0); and
\tikz [baseline=-0.5ex] \draw [outward] (0, 0) -- (1, 0);
%
out of account for the moment, a general diagram for this type of interaction
contains
%
\begin{equation*}
    n \times \tikz [baseline=-0.5ex] \draw [phonon] (0, 0) -- (1, 0);,
    \quad
    (2 n + 1) \times \tikz [baseline=-0.5ex] \draw [electron] (0, 0) -- (1, 0);
    \quad \text{and} \quad
    2 n \times
    \begin{tikzpicture}[baseline=-0.5ex]
        \begin{scope}[gray]
            \draw [  phonon] (0, 0) -- +(90:5mm);
            \draw [backward] (0, 0) -- +(225:5mm);
            \draw [ forward] (0, 0) -- +(315:5mm);
        \end{scope}
        \fill (0, 0) circle (1.5pt);
    \end{tikzpicture}
    \quad \text{with} \quad
    n \in \mathds N,
\end{equation*}
%
where the gray lines define the \q{contacts}. As a consequence, there is always
one in- and one outgoing side of a fermion line, which is not connected to a
vertex.

It is convenient to define the \emph{self-energy} $\Sigma = \sum_i \Sigma_i$ as
the sum of all \emph{irreducible} diagram parts $\Sigma_i$, i.e. such which can
not be split into two disconnected parts by \q{cutting} one fermion line,
because a way to generate all diagrams is to take all permutations of all
possible subsets of all irreducible parts and join them with fermion lines:
%
\begin{align*}
    G &= G_0 + G_0 \Sigma G_0 + G_0 \Sigma G_0 \Sigma G_0 + \dots \\
    &= G_0 + G_0 \Sigma (G_0 + G_0 \Sigma G_0 + \dots) \\
    &= G_0 + G_0 \Sigma G.
\end{align*}

\begin{subequations} \label{Hedin equations}
    This is knows as the \emph{\name{Dyson} equation} which can also be
    formulated diagrammatically as
    %
    \begin{equation}
        %!TEX root = ../thesis.tex
%
\begin{tikzpicture}[baseline]
    \pgfmathsetmacro\r{sqrt(sqrt(3)/pi)/2}

    \draw [electron, double] (0, 0) -- (1, 0);

    \node at (1.5, 0) {$=$};

    \draw [electron] (2, 0) -- (3, 0);

    \node at (3.5, 0) {$+$};

    \coordinate (i) at (5-\r, 0);
    \coordinate (o) at (5+\r, 0);

    \draw [electron] (4, 0) -- (i);
    \draw [electron, double] (o) -- (6, 0);

    \draw [fill=black!10!white] (5, 0) circle [radius=\r] node {$\Sigma$};

    \foreach \point in {i, o} \fill (\point) circle (1.5pt);
\end{tikzpicture}
.
    \end{equation}
    %
    Since the \q{dressed} $G$ appears on both sides of the equation, the latter
    defines a self-consistency problem. It can be solved iteratively, i.e. by
    calculating $G_{n + 1} = G_0 + G_0 \Sigma G_n$ with the \q{bare} $G_0$ as
    the initial value. At each step a higher number of diagrams is considered.
    Multiplication with $G_0^{-1}$ and $G^{-1}$ from the left and right,
    respectively, yields
    %
    \begin{equation*}
        G^{-1} = G_0^{-1} - \Sigma.
    \end{equation*}
    %
    The denominators of $G$ and $G_0$ contain the excitation energies of the
    full and the unperturbed system, as can be seen in Eq.~\ref{G(i omega_n)}
    and Eq.~\ref{G_0(i omega_n)}, respectively. Thus $\Sigma$ acts as an energy
    \emph{renormalization}.

    How can the full self-energy be obtained? It turns out that it can be
    derived from the leading self-energy contributions, calculated in the
    previous section and shown in Fig.~\ref{Hartree} and \ref{Fock}, through
    renormalization of selected lines and even one vertex. This yields
    %
    \begin{equation}
        %!TEX root = ../thesis.tex
%
\tikzsetnextfilename{self-energy}
%
\begin{tikzpicture}[baseline]
    \pgfmathsetmacro\r{sqrt(sqrt(3)/pi)/2}

    \coordinate (i) at (1-2*\r, 0);

    \draw [fill=black!10!white] (1-\r, 0) circle [radius=\r] node {$\Sigma$};

    \coordinate (o) at (1, 0);

    \node at (2, 0) {$=$};

    \coordinate (0) at (3, 0.0);
    \coordinate (1) at (3, 0.5);

    \node at (4, 0) {$+$};

    \coordinate (I) at (5.0, 0);
    \coordinate (X) at (5.5, {sqrt(3)/2});
    \coordinate (R) at (6.0, 0);
    \coordinate (O) at (7.0, 0);

    \begin{scope}[gray]
        \foreach \point in {i, 0, I}
            \draw [backward] (\point) -- +(180:5mm);
    
        \foreach \point in {o, 0, O}
            \draw [forward] (\point) -- +(0:5mm);
    \end{scope}

    \draw [phonon] (0) -- (1);
    \draw [electron, double] (1) arc (-90:270:2.5mm);

    \draw [phonon, double] (X) -- (O);
    \draw [electron, double] (R) -- (O);

    \draw [fill=black!10!white] (X) -- (I) -- (R) -- (X);

    \foreach \point in {i, o, 0, 1, I, X, R, O} \fill (\point) circle (1.5pt);

    \node [below=3mm] at (X) {$\Gamma$};

    \useasboundingbox ([yshift=1mm] current bounding box.north east);
\end{tikzpicture}
,
    \end{equation}
    %
    where the \emph{full vertex} $\Gamma$ has been introduced. The
    \name{Hartree} term on the left can be either neglected, as in the phonon
    case, or incorporated into the single-particle dispersion relation.

    The dressed boson \name{Green} function, $D$ say, obeys another \name{Dyson}
    equation,
    %
    \begin{equation}
        %!TEX root = ../thesis.tex
%
\tikzsetnextfilename{phonon}
%
\begin{tikzpicture}[baseline]
    \pgfmathsetmacro\r{sqrt(sqrt(3)/pi)/2}

    \draw [phonon, double] (0, -0.5) -- (0, 0.5);

    \node at (0.5, 0) {$=$};

    \draw [phonon] (1, -0.5) -- (1, 0.5);

    \node at (1.5, 0) {$+$};

    \coordinate (T) at (2+\r, +\r);
    \coordinate (B) at (2+\r, -\r);

    \draw [phonon, double] (T) -- +(+90:1-\r);
    \draw [phonon] (B) -- +(-90:1-\r);

    \draw [fill=black!10!white] (2+\r, 0) circle [radius=\r] node {$\Pi$};

    \foreach \point in {T, B} \fill (\point) circle (1.5pt);
\end{tikzpicture}%

        \qquad \text{with} \qquad
        %!TEX root = ../thesis.tex
%
\begin{tikzpicture}[baseline]
    \pgfmathsetmacro\r{sqrt(sqrt(3)/pi)/2}

    \coordinate (t) at (1-\r, +\r);
    \coordinate (b) at (1-\r, -\r);

    \draw [fill=black!10!white] (1-\r, 0) circle [radius=\r] node {$\Pi$};

    \node at (1.5, 0) {$=$};

    \coordinate (I) at (2.0, 0);
    \coordinate (T) at (2.5, +{sqrt(3)/2});
    \coordinate (B) at (2.5, -{sqrt(3)/2});
    \coordinate (O) at (3.0, 0);

    \foreach \point in {t, T}
        \draw [phonon, gray] (\point) -- +(+90:5mm);

    \foreach \point in {b, B}
        \draw [phonon, gray] (\point) -- +(-90:5mm);

    \draw [electron, double] (B) -- (I);
    \draw [electron, double] (O) -- (B);

    \draw [fill=black!10!white] (T) -- (I) -- (O) -- (T);

    \foreach \point in {t, b, I, T, B, O} \fill (\point) circle (1.5pt);

    \node [below=3mm] at (T) {$\Gamma$};
\end{tikzpicture}%
,
    \end{equation}
    %
    where the corresponding self-energy $\Pi$ is referred to as
    the \emph{polarization}.

    Finally, the full vertex is given by the plain vertex together with vertex
    corrections, i.e. by all diagram parts with can mediate a boson-induced
    fermionic transition:
    %
    \begin{equation*}
        %!TEX root = ../thesis.tex
%
\tikzsetnextfilename{vertex}
%
\begin{tikzpicture}[baseline]
    \coordinate (i) at (0.0, -{sqrt(3)/4});
    \coordinate (x) at (0.5, +{sqrt(3)/4});
    \coordinate (o) at (1.0, -{sqrt(3)/4});

    \node at (1.5, 0) {$=$};

    \coordinate (0) at (2.5, 0);

    \node at (3.5, 0) {$+$};

    \coordinate (I) at (4.0, -{sqrt(3)/4});
    \coordinate (X) at (4.5, +{sqrt(3)/4});
    \coordinate (O) at (5.0, -{sqrt(3)/4});

    \node at (5.5, 0) {$+$};
    \node at (6.5, 0) {$\dots$};

    \begin{scope}[gray]
        \foreach \point in {i, 0, I}
            \draw [electron] (\point) +(225:5mm) -- (\point);
    
        \foreach \point in {x, 0, X}
            \draw [phonon] (\point) -- +(90:5mm);
    
        \foreach \point in {o, 0, O}
            \draw [electron] (\point) -- +(-45:5mm);
    \end{scope}

    \draw [phonon] (I) -- (O);
    \draw [electron] (I) -- (X);
    \draw [electron] (X) -- (O);

    \draw [fill=black!10!white] (x) -- (i) -- (o) -- (x);

    \foreach \point in {i, x, o, 0, I, X, O}
        \fill (\point) circle (1.5pt);

    \node [below=3mm] at (x) {$\Gamma$};
\end{tikzpicture}%

    \end{equation*}
    %
    Unfortunately, there is no integral equation for $\Gamma$ that could be
    formulated diagrammatically with nothing but the building blocks introduced
    so far, since the vertex corrections involve a functional derivative of the
    self-energy with respect to the dressed \name{Green} function. Representing
    this part with a trapezoidal shape, one can write
    %
    \begin{equation}
        %!TEX root = ../thesis.tex
%
\tikzsetnextfilename{vertex-exact}
%
\begin{tikzpicture}[baseline]
    \coordinate (i) at (0.0, -{sqrt(3)/4});
    \coordinate (x) at (0.5, +{sqrt(3)/4});
    \coordinate (o) at (1.0, -{sqrt(3)/4});
    %
    \node at (1.5, 0) {$=$};
    %
    \coordinate (0) at (2.5, 0);
    %
    \node at (3.5, 0) {$+$};
    %
    \coordinate (I) at (4.00, -{sqrt(3)/2});
    \coordinate (l) at (4.25, -{sqrt(3)/4});
    \coordinate (L) at (4.50, 0);
    \coordinate (X) at (5.00, +{sqrt(3)/2});
    \coordinate (R) at (5.50, 0);=
    \coordinate (r) at (5.75, -{sqrt(3)/4});
    \coordinate (O) at (6.00, -{sqrt(3)/2});
    %
    \begin{scope}[gray]
        \foreach \point in {i, 0, I}
            \draw [backward] (\point) -- +(225:5mm);
        %
        \foreach \point in {x, 0, X}
            \draw [phonon] (\point) -- +(90:5mm);
        %
        \foreach \point in {o, 0, O}
            \draw [forward] (\point) -- +(315:5mm);
    \end{scope}
    %
    \draw [electron, double] (l) -- (L);
    \draw [electron, double] (R) -- (r);
    %
    \draw [fill=blockcolor]
        (x) -- (i) -- (o) -- (x)
        (X) -- (L) -- (R) -- (X)
        (I) -- (O) -- (r) -- (l) -- (I);
    %
    \foreach \point in {i, x, o, 0, I, l, L, X, R, r, O}
        \fill (\point) circle (1.5pt);
    %
    \foreach \point in {x, X}
        \node [below=3mm] at (\point) {$\Gamma$};
    %
    \node at (5, -{3*sqrt(3)/8}) {$\delta \Sigma / \delta G$};
\end{tikzpicture}%
.
    \end{equation}
\end{subequations}

Eqs.~\ref{Hedin equations}, known as \name{Hedin}'s equations \cites[Appendix
A]{Hedin65}[Eqs.~13.19]{HedinLundqvist69}, are all coupled among themselves,
which makes their solution a non-trivial task. In practice one often applies
certain approximations, which are presented in the following.

\subsection{GW approximation}

A considable simplification consists in neglecting all vertex corrections, i.e.
replacing $\Gamma$ with the bare vertex \cites[Eq.~A27,
A28]{Hedin65}[Eqs.~13.20, 13.21]{HedinLundqvist69}. The corresponding diagram
reads:
%
\begin{equation} \label{GW approximation}
    %!TEX root = ../thesis.tex
%
\begin{tikzpicture}[baseline]
    \pgfmathsetmacro\r{sqrt(sqrt(3)/pi)/2}
    %
    \coordinate (i) at (1-2*\r, 0);
    %
    \draw [fill=black!10!white] (1-\r, 0) circle [radius=\r] node {$\Sigma$};
    %
    \coordinate (o) at (1, 0);
    %
    \node at (2, 0) {$=$};
    %
    \coordinate (0) at (3, 0.0);
    \coordinate (1) at (3, 0.5);
    %
    \node at (4, 0) {$+$};
    %
    \coordinate (I) at (5, 0);
    \coordinate (O) at (5+2*\r, 0);
    %
    \begin{scope}[gray]
        \foreach \point in {i, 0, I}
            \draw [electron] (\point) +(180:5mm) -- (\point);
        %
        \foreach \point in {o, 0, O}
            \draw [electron] (\point) -- +(0:5mm);
    \end{scope}
    %
    \draw [phonon] (0) -- (1);
    \draw [electron, double] (1) arc (-90:270:2.5mm);
    %
    % -15 in place of 0 to compensate low accuracy:
    \draw [phonon, double] (I) -- (O);
    \draw [electron, double] (I) arc (180:0:\r);
    %
    \foreach \point in {i, o, 0, 1, I, O} \fill (\point) circle (1.5pt);
\end{tikzpicture}

    \qquad \text{with} \qquad
    %!TEX root = ../thesis.tex
%
\tikzsetnextfilename{gw-polarization}
%
\begin{tikzpicture}[baseline]
    \pgfmathsetmacro\r{sqrt(sqrt(3)/pi)/2}

    \coordinate (t) at (1-\r, +\r);
    \coordinate (b) at (1-\r, -\r);

    \draw [fill=blockcolor] (1-\r, 0) circle [radius=\r] node {$\Pi$};

    \node at (1.5, 0) {$=$};

    \coordinate (T) at (2+\r, +\r);
    \coordinate (B) at (2+\r, -\r);

    \begin{scope}[gray]
        \foreach \point in {t, T}
            \draw [phonon] (\point) -- +(+90:5mm);

        \foreach \point in {b, B}
            \draw [phonon] (\point) -- +(-90:5mm);
    \end{scope}

    \draw [electron, double] (B) arc (-90:90:\r);
    \draw [electron, double] (T) arc (90:270:\r);

    \foreach \point in {t, b, T, B} \fill (\point) circle (1.5pt);

    \useasboundingbox ([xshift=1mm] current bounding box.north east);
\end{tikzpicture}%
.
\end{equation}
%
Since the dressed boson line is often denoted as $W$, this is known as the
\emph{GW approximation} with reference to the resulting formula for the
\name{Fock} part of $\Sigma$. With focus on $\Pi$ and for historical reasons it
is also referred to as the \emph{random-phase approximation} (RPA).

\subsection{HF approximation}

The self-energy can be further simplified by neglecting the polarization and
thus replacing all dressed boson lines with simple ones. This yields the
\emph{\name{Hartree-Fock} approximation} (HF):
%
\begin{equation} \label{HF approximation}
    %!TEX root = ../thesis.tex
%
\begin{tikzpicture}[baseline]
    \pgfmathsetmacro\r{sqrt(sqrt(3)/pi)/2}
    %
    \coordinate (i) at (1-2*\r, 0);
    %
    \draw [fill=black!10!white] (1-\r, 0) circle [radius=\r] node {$\Sigma$};
    %
    \coordinate (o) at (1, 0);
    %
    \node at (2, 0) {$=$};
    %
    \coordinate (0) at (3, 0.0);
    \coordinate (1) at (3, 0.5);
    %
    \node at (4, 0) {$+$};
    %
    \coordinate (I) at (5, 0);
    \coordinate (O) at (5+2*\r, 0);
    %
    \begin{scope}[gray]
        \foreach \point in {i, 0, I}
            \draw [electron](\point) +(180:5mm) -- (\point);
        %
        \foreach \point in {o, 0, O}
            \draw [electron](\point) -- +(0:5mm);
    \end{scope}
    %
    \draw [phonon] (0) -- (1);
    \draw [electron, double] (1) arc (-90:270:2.5mm);
    %
    % -15 in place of 0 to compensate low accuracy:
    \draw [phonon] (I) -- (O);
    \draw [electron, double] (I) arc (180:0:\r);
    %
    \foreach \point in {i, o, 0, 1, I, O} \fill (\point) circle (1.5pt);
\end{tikzpicture}
.
\end{equation}
%
Within HF, many-body problems can always be mapped onto effective
single-particle problems.
