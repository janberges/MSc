%!TEX root = ../thesis.tex

\chapter{\name{Eliashberg} theory}

Based on the results presented in the previous chapter, one can formulate a set
of equations, namely the \name{Eliashberg} equations, which determine
self-energies for both normal and anomalous \name{Green} functions, the latter
being suitable order parameters for the superconducting state.

To that end it is convenient to combine the \name{Green} functions of interest
into a $2 \times 2$ matrix, i.e. to use the \name{Nambu} formalism, which is
presented in the first section. With that, the general form of the
\name{Eliashberg} equations on the imaginary frequency axis are derived,
followed by some common simplifications which assume (i) a $\vec k$-independent
self-energy, (ii) a constant density of states and (iii) the suitability of a
\name{Coulomb} pseudo-potential. Next, it is shown how the results, which are
generally just the numerical values corresponding to a finite set of
\name{Matsubara} frequencies rather than analytical expressions, can be
continued to the real axis by means of \name{Padé} approximants. Finally, it is
outlined how \name{McMillan} and \name{Dynes}' formula for the critical
temperature can be obtained from the \name{Eliashberg} theory.

\section{\name{Nambu} formalism}

As found by \name{Nambu} \cite{Nambu60}, the \text{Dyson} equations for all four
electronic \name{Green} functions introduced at the beginning of
Section~\ref{Model interactions} can be compactly formulated as a single matrix equation:
%
\begin{equation*}
    % !TEX root = ../thesis.tex
%
\tikzsetnextfilename{nambu-dyson}
%
\begin{tikzpicture}[yscale=-1]
    \foreach \r in {
        (-0.25, 0.25), (2.25, 0.25), (4.75, 0.25), (7.00, 0.25), (9.25, 0.25),
        ( 2.25, 1.50), (2.25, 2.75)}
        \node at \r {$\Bigg[$};

    \foreach \r in {
        ( 1.75, 0.25), (4.25, 0.25), (6.75, 0.25), (9.00, 0.25), (11.25, 0.25),
        (12.25, 1.50), (8.25, 2.75)}
        \node at \r {$\Bigg]$};

    \foreach \r in {
        (2.5, 0.00), (5.0, 0.00), ( 2.5, 1.25), (3.5, 1.25),
        (5.5, 1.25), (8.5, 1.25), (10.5, 1.25), (2.5, 2.50),
        (3.5, 2.50), (4.5, 2.50), ( 6.5, 2.50), (4.5, 3.00)}
        \draw [backward] \r -- +(0.5, 0);

    \foreach \r in {(3.75, 0), (6.25, 0), (2.75, 0.5), (5.25, 0.5)}
        \node at \r {$0$};

    \foreach \r in {
        (3.5, 0.50), (6.0, 0.50), ( 3.5, 1.75), (5.5, 1.75),
        (7.5, 1.75), (8.5, 1.75), (10.5, 1.75), (7.5, 2.50),
        (3.5, 3.00), (5.5, 3.00), ( 6.5, 3.00), (7.5, 3.00)}
        \draw [forward] \r -- +(0.5, 0);

    \draw [backward] (4, 2.5) arc (180:360:2.5mm);
    \draw [ outward] (7, 2.5) arc (180:360:2.5mm);
    \draw [  inward] (4, 3.0) arc (180:360:2.5mm);
    \draw [ forward] (7, 3.0) arc (180:360:2.5mm);

    \foreach \r in {(4, 2.5), (7, 2.5), (4, 3), (7, 3)}
        \draw [phonon] \r -- +(0.5, 0);

    \foreach \r in {(0, 0.0), ( 9.5, 0.0), ( 4.5, 1.25), ( 4.5, 1.75)}
        \draw [backward, double] \r -- +(0.5, 0);
    \foreach \r in {(1, 0.0), (10.5, 0.0), ( 9.5, 1.25), ( 9.5, 1.75)}
        \draw [ outward, double] \r -- +(0.5, 0);
    \foreach \r in {(0, 0.5), ( 9.5, 0.5), ( 6.5, 1.25), ( 6.5, 1.75)}
        \draw [  inward, double] \r -- +(0.5, 0);
    \foreach \r in {(1, 0.5), (10.5, 0.5), (11.5, 1.25), (11.5, 1.75)}
        \draw [ forward, double] \r -- +(0.5, 0);

    \foreach \r in {(7.25, 0.0), (4, 1.25), ( 9, 1.25)}
        \draw [backward, double] \r arc (180:360:2.5mm);
    \foreach \r in {(8.25, 0.0), (6, 1.25), (11, 1.25)}
        \draw [ outward, double] \r arc (180:360:2.5mm);
    \foreach \r in {(7.25, 0.5), (4, 1.75), ( 9, 1.75)}
        \draw [  inward, double] \r arc (180:360:2.5mm);
    \foreach \r in {(8.25, 0.5), (6, 1.75), (11, 1.75)}
        \draw [ forward, double] \r arc (180:360:2.5mm);

    \foreach \r in {
        (7.25, 0), (8.25, 0), (7.25, 0.5), (8.25, 0.5),
        (4, 1.25), (6, 1.25), (9, 1.25), (11, 1.25),
        (4, 1.75), (6, 1.75), (9, 1.75), (11, 1.75)}
        \draw [phonon, double] \r -- +(0.5, 0);

    \foreach \r in {
        (7.25, 0), (8.25, 0), (7.25, 0.5), (8.25, 0.5),
        (4, 1.25), (6, 1.25), (9, 1.25), (11, 1.25),
        (4, 1.75), (6, 1.75), (9, 1.75), (11, 1.75),
        (4, 2.50), (7, 2.50), (4, 3.00), ( 7, 3.00)}
        \fill \r circle (1.5pt) +(0.5, 0) circle (1.5pt);

    \foreach \r in {(2, 0.25), (2, 1.5)} \node at \r {$=$};

    \node at (2, 2.75) {$\approx$};

    \foreach \r in {
        (4.50, 0.25),
        (3.25, 1.25), (5.25, 1.25), (8.25, 1.75), (10.25, 1.25),
        (3.25, 2.50), (5.25, 1.75), (6.25, 3.00), (10.25, 1.75)}
        \node at \r {$+$};
\end{tikzpicture}%

\end{equation*}
%
or formally, where quantum numbers and frequency arguments are suppressed, as
%
\begin{align*}
	\begin{bmatrix}
		G & F \\
		\widetilde F & \widetilde G
	\end{bmatrix}
	&=
	\begin{bmatrix}
		G_0 & 0 \\
		0 & \widetilde G_0
	\end{bmatrix}
	+
	\begin{bmatrix}
		G_0 & 0 \\
		0 & \widetilde G_0
	\end{bmatrix}
	\begin{bmatrix}
		\Sigma^G & \Sigma^F \\
		\Sigma^{\widetilde F} & \Sigma^{\widetilde G}
	\end{bmatrix}
	\begin{bmatrix}
		G & F \\
		\widetilde F & \widetilde G
	\end{bmatrix}
	\\
	&=
	\begin{bmatrix*}[r]
		G_0 + G_0 \Sigma^G G + G_0 \Sigma^F \widetilde F
        & G_0 \Sigma^G F + G_0 \Sigma^F \widetilde G \\
		\widetilde G_0 \Sigma^{\widetilde F} G
        + \widetilde G_0 \Sigma^{\widetilde G} \widetilde F
        & \widetilde G_0 + \widetilde G_0 \Sigma^{\widetilde F} F
        + \widetilde G_0 \Sigma^{\widetilde G} \widetilde G
	\end{bmatrix*}
	\\
	&\approx
	\begin{bmatrix*}[r]
		G_0 + G_0 \Sigma^G_0 G_0 & G_0 \Sigma^F_0 \widetilde G_0 \\
		\widetilde G_0 \Sigma^{\widetilde F}_0 G_0 & \widetilde G_0
        + \widetilde G_0 \Sigma^{\widetilde G}_0 \widetilde G_0
	\end{bmatrix*}.
\end{align*}

A comparison with the results for the \name{Holstein} model derived in
Section~\ref{Holstein model} reveals that the above approximation is correct up
to the second order of the perturbation series.

Giving names to the involved matrices, the \name{Dyson} equation may be written
as $\vec G = \vec G_0 + \vec G_0 \vec \Sigma \vec G$ or even as $\vec G^{-1} =
\vec G_0^{-1} - \vec \Sigma$ using inverse matrices. For a concise definition of
$\vec G$ and $\vec G_0$ the two-component quantities
%
\begin{equation*}
	\vec \uppsi_{\vec k} =
    \begin{bmatrix}
        \op c_{\vec k \uparrow} \\
        \op c_{-\vec k \downarrow}^+
    \end{bmatrix}
    \quad \text{and} \quad
    \vec \uppsi_{\vec k}^+ =
    \begin{bmatrix}
        \op a_{\vec k \uparrow}^+ &
        \op a_{-\vec k \downarrow}
    \end{bmatrix}
\end{equation*}
%
are introduced, together with the \textsc{Pauli} matrices
%
\begin{equation*}
	\sigma_1 =
    \begin{bmatrix}
        0 & 1 \\
        1 & 0
    \end{bmatrix},
    \quad
    \sigma_2 =
    \begin{bmatrix}
        0 & -\I \\
        \I & 0
    \end{bmatrix},
    \quad
    \sigma_3 =
    \begin{bmatrix}
    1 & 0 \\
    0 & -1
    \end{bmatrix}
    \quad \text{and} \quad
    \sigma_0 =
    \begin{bmatrix}
        1 & 0 \\
        0 & 1
    \end{bmatrix}.
\end{equation*}
%
Therewith, the dressed (bare) \name{Green} function may be written as
%
\begin{equation*}
    \vec G_{\vec k}^{(0)}(\tau) = -\av{\op T \,
    \vec \uppsi_{\vec k}(\tau) \,
    \vec \uppsi_{\vec k}^+(0)}_{(0)}.
\end{equation*}

\section{General equations}

For a self-energy on the imaginary axis, which may be written as
%
\begin{align*}
    \vec \Sigma_{\vec k}(\I \omega_n) &= -\frac 1 \beta \sum_{m \, \vec q}
    \vec \sigma_3 \vec G_{\vec q \sigma}(\I \omega_m) \vec \sigma_3 \,
    V_{\vec k, \vec q}(\omega_n - \omega_m)
    \\
    &= \I \omega_n [1 - Z_{\vec k}(\I \omega_n)] \vec \sigma_0
    + \phi_{\vec k}(\I \omega_n) \vec \sigma_1
    + \phi'_{\vec k}(\I \omega_n) \vec \sigma_2
    + \chi_{\vec k}(\I \omega_n) \vec \sigma_3,
\end{align*}
%
equating coefficients yields the \textsc{Eliashberg} equations
%
\begin{align} \label{Eliashberg equations}
    \begin{split}
        \I \omega_n [1 - Z_{\vec k}(\I \omega_n)] &=
        \frac 1 \beta \sum_{m \, \vec q}
        \frac
            {\I \omega_m Z_{\vec q}(\I \omega_m)}
            {\Theta_{\vec q}(\I \omega_m)}
        V_{\vec k, \vec q}(\omega_n - \omega_m),
        \\
        \phi^{(\prime)}_{\vec k}(\I \omega_n) &=
        -\frac 1 \beta \sum_{m \, \vec q}
        \frac
            {\phi^{(\prime)}_{\vec q}(\I \omega_m)}
            {\Theta_{\vec q}(\I \omega_m)}
        V_{\vec k, \vec q}(\omega_n - \omega_m),
        \\
        \chi_{\vec k}(\I \omega_n) &=
        \frac 1 \beta \sum_{m \, \vec q}
        \frac
            {\epsilon_{\vec q} + \chi_{\vec q}(\I \omega_m)}
            {\Theta_{\vec q}(\I \omega_m)}
        V_{\vec k, \vec q}(\omega_n - \omega_m),
        \\
        \Theta_{\vec k}(\I \omega_n) &=
        [\omega_n Z_{\vec k}(\I \omega_n)]^2
        + [\epsilon_{\vec k} + \chi_{\vec k}(\I \omega_n)]^2
        + \phi_{\vec k}^2(\I \omega_n)
        + \phi'^2_{\vec k}(\I \omega_n).
    \end{split}
\end{align}

\section{Isotropic equations}

\name{Holstein} and/or \name{Hubbard} model

\subsection{Constant-density approximation}

\subsection{Computational implementation}
