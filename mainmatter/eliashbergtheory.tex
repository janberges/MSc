%!TEX root = ../thesis.tex

\chapter{\name{Eliashberg} theory}

\section{\name{Nambu} formalism}

\section{General equations}

For a self-energy on the imaginary axis, which may be written as
%
\begin{align*}
    \vec \Sigma_{\vec k}(\I \omega_n) &= -\frac 1 \beta \sum_{m \, \vec q}
    \vec \sigma_3 \vec G_{\vec q \sigma}(\I \omega_m) \vec \sigma_3 \,
    V_{\vec k, \vec q}(\omega_n - \omega_m)
    \\
    &= \I \omega_n [1 - Z_{\vec k}(\I \omega_n)] \vec \sigma_0
    + \phi_{\vec k}(\I \omega_n) \vec \sigma_1
    + \phi'_{\vec k}(\I \omega_n) \vec \sigma_2
    + \chi_{\vec k}(\I \omega_n) \vec \sigma_3,
\end{align*}
%
equating coefficients yields the \textsc{Eliashberg} equations
%
\begin{align} \label{Eliashberg equations}
    \begin{split}
        \I \omega_n [1 - Z_{\vec k}(\I \omega_n)] &=
        \frac 1 \beta \sum_{m \, \vec q}
        \frac
            {\I \omega_m Z_{\vec q}(\I \omega_m)}
            {\Theta_{\vec q}(\I \omega_m)}
        V_{\vec k, \vec q}(\omega_n - \omega_m),
        \\
        \phi^{(\prime)}_{\vec k}(\I \omega_n) &=
        -\frac 1 \beta \sum_{m \, \vec q}
        \frac
            {\phi^{(\prime)}_{\vec q}(\I \omega_m)}
            {\Theta_{\vec q}(\I \omega_m)}
        V_{\vec k, \vec q}(\omega_n - \omega_m),
        \\
        \chi_{\vec k}(\I \omega_n) &=
        \frac 1 \beta \sum_{m \, \vec q}
        \frac
            {\epsilon_{\vec q} + \chi_{\vec q}(\I \omega_m)}
            {\Theta_{\vec q}(\I \omega_m)}
        V_{\vec k, \vec q}(\omega_n - \omega_m),
        \\
        \Theta_{\vec k}(\I \omega_n) &=
        [\omega_n Z_{\vec k}(\I \omega_n)]^2
        + [\epsilon_{\vec k} + \chi_{\vec k}(\I \omega_n)]^2
        + \phi_{\vec k}^2(\I \omega_n)
        + \phi'^2_{\vec k}(\I \omega_n).
    \end{split}
\end{align}

\section{Isotropic equations}

\name{Holstein} and/or \name{Hubbard} model

\subsection{Constant-density approximation}

\subsection{Computational implementation}
