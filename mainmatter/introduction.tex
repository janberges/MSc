% !TEX root = ../thesis.tex

\chapter{Introduction}

Ever since \name{McMillan}'s formula has been published in 1968
\cite{McMillan68}, it has been widely used%
%
\footnote{Today, the American Physical Society alone lists 3203 references to
the original paper by \name{McMillan}, to be complemented by 1326 citations of a
closely related publication by \name{Allen} and \name{Dynes}
\cite{AllenDynes75}.}
%
to obtain estimates of the critical temperature of superconductors as a function
of three effective material parameters, namely an average phonon frequency $\av
\omega$, the electron-phonon coupling strength $\lambda$ and the \name{Coulomb}
pseudo-potential $\mu^*$, which can be extracted from experiment
\cite{McMillanRowell69} or first-principles calculations. It constitutes an
approximation to the more general \name{Eliashberg} theory of superconductivity
\cite{Eliashberg60} from which it was derived by fitting analytic approximations
of the underlying equations to exact numerical results. Although for the latter
the special phononic density of states of niobium has been assumed, which was
simply at hand at that time \cite{NakagawaWoods63}, the validity of the
resulting formula turned out to be much more general.

The aim of the present work is to trace the steps that lead from the theory of
the fundamental interactions between electrons and phonons to the handy formula
for the critical temperature and to perform further tests on its scope, many of
them, supposedly, have already been carried out somewhere in its past of almost
half a century and fallen into oblivion or, more probably, just overlooked this
time. Special attention is paid to potential discrepancies emerging from
exceptional densities of electronic states and the question if and possibly how
the multi-band case with non-scalar coupling strengths can be brought into
accordance. Notwithstanding that in the course of the investigations no
references to specific materials are made but rather simple models applied, it
is intended that the results be of use for the understanding of novel,
especially two-dimensional materials.

For this purpose, an appropriate software is developed which may be used not
only to obtain electronic self-energies on the imaginary or real frequency axis
as solutions of the multi-band \name{Eliashberg} equations or analytically
continued by means of \name{Padé} approximants \cite{VidbergSerene77},
respectively, but also to solve the linearized critical-state equations for a
parameter of choice, which may be either the critical temperature itself, the
phonon frequency or any element of the matrices defining the coupling strengths,
for the respective other quantities fixed.

To make a start, the following Chapter~\ref{superconductivity} gives a very
brief introduction to the field of superconductivity, including an outline of
its early history and the presentation of the prominent BCS theory \cites
{BardeenCooperSchrieffer57a} {BardeenCooperSchrieffer57b}. Before this subject
can be discussed in more detail, it is necessary to extend the theoretical
framework by introducing the fundamental concepts of many-body physics such as
\name{Green} functions and diagrammatical perturbation theory, which will be
done in Chapter~\ref{many-body physics}. On this basis, the different
formulations and special cases of the \name{Eliashberg} theory of
superconductivity are dealt with in Chapter~\ref{Eliashberg theory}, which
completes the preparative part of the thesis. Subsequently, the actual numerical
results are demonstrated in Chapters~\ref{single-band results} and
\ref{multi-band results} which are dedicated to the results for single- and
muli-band systems, respectively, where \emph{band} stands in place for any
chosen subset of electronic states. Finally, in Chapter.~\ref{conclusion} the
most important results are summarized and the pending questions brought up for
future discussions. A good deal of the work falls within the scope of
Appendix~\ref{source code} where, following a short formulary on \name{Fourier}
analysis in Appendix~\ref{Fourier analysis}, the source code of the employed
programs is exposed and commented, supplemented with a short user manual.
