% !TEX root = ../thesis.tex

\chapter{Superconductivity}

This chapter gives an introduction to the field of superconductivity and is not
intended to be exhaustive. The theory that will actually be applied in this work
is presented the following two chapters.

In the first section the earlier history of superconductivity is briefly
reviewed up to the point when the interaction of electrons and phonons is
identified as causing the phenomenon.%
%
\footnote{The history is retold following H. \name{Fröhlich}
\cite{Froehlich82}.}
%
Next, a canonical transformation is presented which reveals how this interaction
can lead to an effective attraction between electrons and consequently to the
formation of \name{Cooper} pairs. Thereafter, the theory of \name{Bardeen},
\name{Cooper} and \name{Schrieffer} is presented, followed by a description of
the concept of \q{off-diagonal long-range order}.

\section{History of superconductivity}

In 1911 the dutch physicist H. \name{Kamerlingh Onnes} finds that mercury ceases
to resist electric current completely when cooled down to temperatures below
$4\,^\circ \mathrm K$ with the help of liquid helium.%
%
\footnote{According to Ref.~\barecite{vanDelftKes10}, on April 8, 1911
\name{Kamerlingh Onnes} writes \qq{Kwik nagenoeg nul} in his notebook, which
means \qq{[resistance of] mercury near-enough zero}. In Communication No. 122b
from the Physical Laboratory at Leiden he states more precisely: \qq{At
$3\,^\circ \mathrm K$, the resistance was found to have fallen below [\dots] one
ten-millionth of the value which it would have at $0\,^\circ \mathrm C$}
\cite{KamerlinghOnnes11}.}
%
This is the first time a manifestation of superconductivity is observed. In the
years that follow, similar observations are made for other metals and the
fundamental properties of the novel state are exposed.

\section{Canonical transformation}

It is now shown that the electron-phonon interaction can involve an effective
electron-electron interaction which is attractive.%
%
\footnote{The original derivation is due to \name{Fröhlich} \cite{Froehlich52}.
The presentation at hand is rather guided by \name{Czycholl} \cite{Czycholl08}.}
%
To that end the \name{Fröhlich-Hamilton} operator
%
\begin{gather*}
    \op H = \overbrace{
        \sum_{\vec k} \epsilon_{\vec k} \op c_{\vec k}^+ \op c_{\vec k} +
        \sum_{\vec q} \omega_{\vec q} \op b_{\vec q}^+ \op b_{\vec q}
    }^{\op H_0} + \overbrace{
        \sum_{\vec k \vec q} g_{\vec q} \op c_{\vec k + \vec q}^+ \op c_{\vec k}
        [\op b_{\vec q} + \op b_{-\vec q}^+]
    }^{\op V}
\end{gather*}
%
is considered, which describes the interaction between electrons with wave
number $\vec k$ and energy $\epsilon_{\vec k}$, which are annihilated and
created by the \name{Fermi} operator $\op a_{\vec k}$ and its adjoint,
respectively, and longitudinal phonons, where analogous definitions hold for
$\vec q$, $\omega_{\vec q}$ and $\op b_{\vec q}$. The strength of the coupling
is given by $g_{\vec q}$. The spin is of no importance here and thus omitted.

The idea is to apply a canonical transformation to the \name{Hamilton} operator
by means of the unitary operator $\E^{\I \op S}$, where $\op S$ is self-adjoint.
With the help of \name{Hadamard}'s lemma,
%
\begin{align*}
    \op H \sub T
    = \E^{-\I \op S} \, \op H \, \E^{\I \op S}
    &= \op H + \I [\op H, \op S] - \frac 1 2 [[\op H, \op S], \op S] + \dots \\
    &= \op H_0 + \op V + \I [\op H_0, \op S] + \I [\op V, \op S]
    - \frac 1 2 [[\op H_0, \op S], \op S] + \dots \\
    &\approx \op H_0 + \frac \I 2 [\op V, \op S] \equiv \op H_0 + \op V',
\end{align*}
%
choosing $[\op H_0, \op S] = \I \op V$, which implies a linear dependence of
$\op S$ on $\op V$. Thus all terms with are at least quadratic in the
interaction are neglected. Making use of the commutation relations
%
\begin{align*}
    [\op b_{\vec q'}^+ \op b_{\vec q'}, \ \op b_{\vec q \phantom +}]
    &= -\op b_{\vec q \phantom +} \delta_{\vec q'}^{\vec q},
    &
    [ \op c_{\vec k' \phantom{+ \vec q'}}^+ \op c_{\vec k'},
    \ \op c_{\vec k + \vec q}^+ \op c_{\vec k} ]
    &= \op c_{\vec k + \vec q}^+ \op c_{\vec k}
    \big[ \delta_{\vec k'}^{\vec k + \vec q} - \delta_{\vec k'}^{\vec k} \big],
    \\
    [\op b_{\vec q'}^+ \op b_{\vec q'}, \ \op b_{-\vec q}^+]
    &= \phantom + \op b_{-\vec q}^+ \delta_{\vec q'}^{-\vec q},
    &
    [ \op c_{\vec k' + \vec q'}^+ \op c_{\vec k'},
    \  \op c_{\vec k  + \vec q }^+ \op c_{\vec k } ]
    &= \op c_{\vec k + \vec q + \vec q'}^+ \op c_{\vec k}
    \delta_{\vec k'}^{\vec k + \vec q}
    - \op c_{\vec k + \vec q}^+ \op c_{\vec k - \vec q'}
    \delta_{\vec k'}^{\vec k - \vec q'}
\end{align*}
%
one can verify both that $\op S$ has the explicit form
%
\begin{equation*}
    \op S = \I \sum_{\vec k \vec q} g_{\vec q}
    \op c_{\vec k + \vec q}^+ \op c_{\vec k} \left[
        \frac {\op b_{\vec q}}
        {\epsilon_{\vec k + \vec q} - \epsilon_{\vec k} - \omega_{\vec q}}
        + \frac {\op b_{-\vec q}^+}
        {\epsilon_{\vec k + \vec q} - \epsilon_{\vec k} + \omega_{-\vec q}}
    \right]
\end{equation*}
%
and, assuming $g_{\vec q}^* = g_{\vec - q}$ and $\omega_{\vec q} = \omega_{-\vec
q}$, that the renormalized interaction is given by
%
\begin{align*}
    \op V' = &\sum_{\vec k \vec k' \vec q}
    \frac {|g_{\vec q}|^2 \omega_{\vec q}}
    {(\epsilon_{\vec k + \vec q} - \epsilon_{\vec k})^2 - \omega_{\vec q}^2}
    \op c_{\vec k  + \vec q}^+ \op c_{\vec k }
    \op c_{\vec k' - \vec q}^+ \op c_{\vec k'} + \dots
    \\
    \dots - \frac 1 2 &\sum_{\vec k \vec q \vec q'} g_{\vec q} g_{\vec q'}
    \big[ \op c_{\vec k + \vec q + \vec q'}^+ \op c_{\vec k}
    - \op c_{\vec k + \vec q}^+ \op c_{\vec k - \vec q'} \big]
    \big[ \op b_{\vec q'} + \op b_{-\vec q'}^+ \big] \left[
        \frac {\op b_{\vec q}}
        {\epsilon_{\vec k + \vec q} - \epsilon_{\vec k} - \omega_{\vec q}}
        + \frac {\op b_{-\vec q}^+}
        {\epsilon_{\vec k + \vec q} - \epsilon_{\vec k} + \omega_{-\vec q}}
    \right].
\end{align*}
%
The first term describes an effective interaction between electrons which is
attractive, i.e. negative, for $|\epsilon_{\vec k + \vec q} - \epsilon_{\vec k}|
< \omega_{\vec q}$. The seconds term describes processes involving two phonons.

\section{BCS theory}

\name{Cooper} pairs

\section{Off-diagonal long-range order}

\begin{itemize}
    \item \name{Gor'kov-Green} functions (equations-of-motion approach?)
    \item creation/annihilation of indeterminate \name{Cooper} pairs
\end{itemize}
