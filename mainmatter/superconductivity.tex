% !TEX root = ../thesis.tex

\chapter{Superconductivity}
\label{superconductivity}

This chapter gives an introduction to the field of superconductivity and is not
at all intended to be exhaustive. The theory that will actually be applied in
this work is presented in the following two chapters.

In the first section the earlier history of superconductivity is briefly
reviewed, loosely following a presentation by \name{Fröhlich}
\cite{Froehlich82}. Next, a canonical transformation introduced by the latter is
performed, which reveals that the interaction between electrons and phonons can
lead to an effective attraction between electrons and consequently to the
formation of \name{Cooper} pairs. Finally, the corresponding microscopic theory
of superconductivity by \name{Bardeen}, \name{Cooper} and \name{Schrieffer} is
presented.%
%
\footnote{Both derivations in this chapter are guided by \name{Czycholl}
\cite{Czycholl08}.}

Throughout this work units are chosen in which the \name{Boltzmann} and the
reduced \name{Planck} constant are unity, i.e. $k \sub B = \hbar = 1$.
Consequently, the same dimension is attributed to energy and both temperature
and frequency.

\section{Early history of superconductivity}

In 1911 the dutch physicist \name{Kamerlingh Onnes} finds that mercury ceases to
resist electric current completely when cooled down below a critical temperature
of about $4 \, \unit K$ with the help of liquid helium.%
%
\footnote{According to Ref.~\barecite{vanDelftKes10}, on April 8, 1911
\name{Kamerlingh Onnes} writes \qq{Kwik nagenoeg nul} in his notebook, which
means \qq{[resistance of] mercury near enough zero}. In Communication No. 122b
from the Physical Laboratory at Leiden he states more precisely: \qq{At $3 \,
^\circ \unit K$ [sic], the resistance was found to have fallen below [\dots] one
ten-millionth of the value which it would have at $0 \, ^\circ \unit C$}
\cite{KamerlinghOnnes11}.}
%
This is the first time a manifestation of superconductivity is observed. In the
years that follow, similar observations are made for other metals and the
fundamental properties of the novel state exposed.

The misconception of superconductors which obey \name{Ohm}'s law is overcome in
1933, when \name{Meißner} and \name{Ochsenfeld} find them to be perfect
diamagnets \cite{MeissnerOchsenfeld33}: Up to a critical strength an external
magnetic field is expelled from a superconductor -- even if it was already there
\emph{before} the critical temperature has been undercut. On the basis of this
observation the \name{London} brothers formulate a first macroscopic theory of
superconductivity in 1935 \cite{LondonLondon35}.

However, not only have the characteristics of a phase transition been left out
of consideration, but also the underlying physical mechanisms remain unclear.
Some of the further are addressed by the phenomenological theory of
\name{Ginzburg} and \name{Landau} published in 1950 \cite{GinzburgLandau50},
which introduces an order parameter for the superconducting state.

An important hint towards the right direction is the discovery of the isotope
effect by \name{Maxwell} \cite{Maxwell50} and \name{Reynolds} et al.
\cite{ReynoldsSerinWrightNesbitt50} in the same year: The critical temperature
depends on the nuclear mass just as the phononic behavior. At that time,
\name{Fröhlich} starts to successfully use field-theoretical methods to describe
the interaction between electrons and phonons, which is gradually accepted as
causing superconductivity. In 1952, his electron-phonon \name{Hamilton} operator
is established and mapped onto an effective interaction between electrons which
turns out to be attractive \cite{Froehlich52}.

It is not until 1957 that on this basis the first microscopic theory of
superconductivity is formulated by \name{Bardeen}, \name{Cooper} and
\name{Schrieffer} \cites {BardeenCooperSchrieffer57a}
{BardeenCooperSchrieffer57b}. The principal idea is the formation of a
condensate of \name{Cooper} pairs which opens up an energy gap at the
\name{Fermi} level the width of which turns out to be a suitable order
parameter.

Since the modeled interaction in the BCS theory is instantaneous as opposed to
the underlying electron-phonon interaction which is retarded, discrepancies
between theory and experiment emerge, especially when the coupling is strong. In
1960, \name{Eliashberg} establishes a theory which accounts for this retarded
nature of the interaction \cite{Eliashberg60}.

\section{Canonical transformation}

It is now shown that the electron-phonon interaction can involve an effective
electron-electron interaction which is attractive \cite{Froehlich52}. To that
end the \emph{\name{Fröhlich-Hamilton} operator}
%
\begin{gather*}
    \op H = \hyper{
        \sum_{\vec k} \epsilon_{\vec k} \op c_{\vec k}^+ \op c_{\vec k} +
        \sum_{\vec q} \omega_{\vec q} \op b_{\vec q}^+ \op b_{\vec q}
    }{\op H_0} + \hyper{
        \sum_{\vec k \vec q} g_{\vec q} \op c_{\vec k + \vec q}^+ \op c_{\vec k}
        [\op b_{\vec q} + \op b_{-\vec q}^+]
    }{\op V}
\end{gather*}
%
is considered, which describes the interaction between electrons with wave
number $\vec k$ and energy $\epsilon_{\vec k}$, which are annihilated and
created by the \name{Fermi} operator $\op c_{\vec k}$ and its adjoint,
respectively, and longitudinal phonons, where analogous definitions hold for
$\vec q$, $\omega_{\vec q}$ and $\op b_{\vec q}$. The strength of the coupling
is given by $g_{\vec q}$. The spin is of no importance here and thus omitted.

The idea is to apply a canonical transformation to the \name{Hamilton} operator
by means of the unitary operator $\E^{\I \op S}$, where $\op S$ is self-adjoint.
Expanding the exponential functions,
%
\begin{align*}
    \op H \sub T
    = \E^{-\I \op S} \, \op H \, \E^{\I \op S}
    &= \op H + \I [\op H, \op S] - \frac 1 2 [[\op H, \op S], \op S] + \dots \\
    &= \op H_0 + \op V + \I [\op H_0, \op S] + \I [\op V, \op S]
    - \frac 1 2 [[\op H_0, \op S], \op S] + \dots \\
    &\approx \op H_0 + \frac \I 2 [\op V, \op S] \equiv \op H_0 + \op V \sub T,
\end{align*}
%
where $[\op H_0, \op S] = \I \op V$ has been chosen, which implies a linear
dependence of $\op S$ on $\op V$. Thus all terms which are at least quadratic in
the interaction are neglected. Using the commutators
%
\begin{align*}
    \big[ \op b_{\vec q'}^+ \op b_{\vec q'}, \ \op b_{\vec q \phantom +} \big]
    &= -\op b_{\vec q \phantom +} \delta_{\vec q'}^{\vec q},
    &
    \big[
        \op c_{\vec k' \phantom{+ \vec q'}}^+ \op c_{\vec k'}, \
        \op c_{\vec k + \vec q}^+ \op c_{\vec k}
    \big]
    &= \op c_{\vec k + \vec q}^+ \op c_{\vec k}
    \big[ \delta_{\vec k'}^{\vec k + \vec q} - \delta_{\vec k'}^{\vec k} \big],
    \\
    \big[ \op b_{\vec q'}^+ \op b_{\vec q'}, \ \op b_{-\vec q}^+ \big]
    &= \phantom + \op b_{-\vec q}^+ \delta_{\vec q'}^{-\vec q},
    &
    \big[
        \op c_{\vec k' + \vec q'}^+ \op c_{\vec k'}, \
        \op c_{\vec k  + \vec q }^+ \op c_{\vec k }
    \big]
    &= \op c_{\vec k + \vec q + \vec q'}^+ \op c_{\vec k}
    \delta_{\vec k'}^{\vec k + \vec q}
    - \op c_{\vec k + \vec q}^+ \op c_{\vec k - \vec q'}
    \delta_{\vec k'}^{\vec k - \vec q'},
\end{align*}
%
one can verify both that $\op S$ has the explicit form
%
\begin{equation*}
    \op S = \I \sum_{\vec k \vec q} g_{\vec q}
    \op c_{\vec k + \vec q}^+ \op c_{\vec k} \left[
        \frac {\op b_{\vec q}}
        {\epsilon_{\vec k + \vec q} - \epsilon_{\vec k} - \omega_{\vec q}}
        + \frac {\op b_{-\vec q}^+}
        {\epsilon_{\vec k + \vec q} - \epsilon_{\vec k} + \omega_{-\vec q}}
    \right]
\end{equation*}
%
and, assuming $g_{\vec q}^* = g_{\vec - q}$ and $\omega_{\vec q} = \omega_{-\vec
q}$, that the renormalized interaction is given by
%
\begin{align*}
    \op V \sub T = &\sum_{\vec k \vec k' \vec q}
    \frac {|g_{\vec q}|^2 \omega_{\vec q}}
    {(\epsilon_{\vec k + \vec q} - \epsilon_{\vec k})^2 - \omega_{\vec q}^2}
    \op c_{\vec k  + \vec q}^+ \op c_{\vec k }
    \op c_{\vec k' - \vec q}^+ \op c_{\vec k'} + \cdots
    \\
    \cdots - \frac 1 2 &\sum_{\vec k \vec q \vec q'} g_{\vec q} g_{\vec q'}
    \big[ \op c_{\vec k + \vec q + \vec q'}^+ \op c_{\vec k}
    - \op c_{\vec k + \vec q}^+ \op c_{\vec k - \vec q'} \big]
    \big[ \op b_{\vec q'} + \op b_{-\vec q'}^+ \big] \left[
        \frac {\op b_{\vec q}}
        {\epsilon_{\vec k + \vec q} - \epsilon_{\vec k} - \omega_{\vec q}}
        + \frac {\op b_{-\vec q}^+}
        {\epsilon_{\vec k + \vec q} - \epsilon_{\vec k} + \omega_{-\vec q}}
    \right].
\end{align*}
%
The first term describes an effective interaction between electrons which is
attractive, i.e. negative, for $|\epsilon_{\vec k + \vec q} - \epsilon_{\vec k}|
< \omega_{\vec q}$. The seconds term describes processes involving two phonons
and is disregarded in the following.

Since the energy transfer associated with the attractive interaction is small,
only electrons near the \name{Fermi} surface are affected, where both free and
occupied states are available. Taking further the conservation of momentum into
account, the possible momentum transfers are very limited except for the case
when the total momentum vanishes. As a consequence, an electron is most
susceptible for being effectively attracted to its time-revered counterpart,
which has both opposite momentum and spin, whereby the latter prevents local
interactions from vanishing. The crucial idea which led to the understanding of
superconductivity is that such electrons form so-called \emph{\name{Cooper}
pairs}, bound by the attractive interaction.

\section{BCS theory}

\begin{figure}
    \small
    \input{results/bcs.sl}
    \caption[BCS gap]{
        Temperature dependence of the BCS gap for different values of the
        coupling strength $V N(\epsilon)$. The \name{Debye} frequency is assumed
        to be $20 \, \unit{meV}$.}
    \label{BCS gap}
\end{figure}
%
In 1957, \name{Bardeen}, \name{Cooper} and \name{Schrieffer} developed, on the
basis of the results presented in the previous section, a model \name{Hamilton}
operator to describe superconductivity \cites {BardeenCooperSchrieffer57a}
{BardeenCooperSchrieffer57b}. It is reduced to the essential, namely an
attractive interaction of uniform strength $V$ between electrons which form a
\name{Cooper} pair and are no further away from the \name{Fermi} surface than a
typical phonon frequency, typically \name{Debye}'s $\omega \sub D$. Since the
total momentum of a \name{Cooper} pair is zero, the internal momentum transfer
corresponds to a rotation in $\vec k$-space. The model reads
%
\begin{equation*}
    \op H = \sum_{\vec k \sigma} \epsilon_{\vec k} \op n_{\vec k \sigma}
    - \sum_{\vec k \vec k'} V_{\vec k \vec k'} \op C_{\vec k'}^+ \op C_{\vec k}
    \quad \text{with} \quad
    \op n_{\vec k \sigma} = \op c_{\vec k \sigma}^+ \op c_{\vec k \sigma}.
\end{equation*}
%
$\op C_{\vec k} = \op c_{-\vec k \down} \op c_{\vec k \up}$ is a
\name{Cooper}-pair annihilator, which does \emph{not} satisfy \name{Bose}
commutation relations:
%
\begin{equation*}
    [\op C_{\vec k}, \op C_{\vec k'}^+]
    = (1 - \op n_{\vec k \up} - \op n_{-\vec k \down}) \delta_{\vec k \vec k'}.
\end{equation*}
%
Measuring energies relative to the \name{Fermi} level, the interaction strength
is given by
%
\begin{equation*}
    V_{\vec k \vec k'} = \begin{cases}
        V & \text{
             if $|\epsilon_{\vec k }| < \omega \sub D$
            and $|\epsilon_{\vec k'}| < \omega \sub D$,} \\
        0 & \text{otherwise.}
    \end{cases}
\end{equation*}

In order to promote the solution of the \name{Hamilton} operator, the
interaction term is usually factorized into anomalous expectation values. Hence,
in the exact identity \cite[Eq.~4.20]{Nolting15}
%
\begin{equation*}
	\op C_{\vec k'}^+ \op C_{\vec k} =
    (\op C_{\vec k'}^+ - \av{\op C_{\vec k'}^+})
    (\op C_{\vec k} - \av{\op C_{\vec k}})
    + \av{\op C_{\vec k'}^+} \op C_{\vec k}
    + \av{\op C_{\vec k}} \op C_{\vec k'}^+
    - \av{\op C_{\vec k'}^+} \av{\op C_{\vec k}}
\end{equation*}
%
leading and trailing summands, which represent fluctuations of the \name{Cooper}
pair operators and a constant energy shift, respectively, are neglected from now
on. The averages of either two creation or annihilation operator do not vanish,
as one might expect, because the \name{Hamilton} operator with respect to which
they are evaluated no longer conserves the particle number:
%
\begin{equation} \label{BCS order parameter}
    \op H = \sum_{\vec k \sigma} \epsilon_{\vec k} \op n_{\vec k \sigma}
    - \sum_{\vec k} \Delta_{\vec k} [\op C_{\vec k} + \op C_{\vec k}^+]
    \quad \text{with} \quad
    \Delta_{\vec k} = \sum_{\vec k'} V_{\vec k \vec k'} \av{\op C_{\vec k'}}.
\end{equation}
%
A self-consistently problem has emerged. $\Delta_{\vec k}$ will turn out to be a
suitable order parameter for the superconducting state. Being not interested in
its phase, which would be important to describe e.g. tunneling effects, it is
assumed to be real.

At this point, the \name{Hamilton} operator is still not diagonal. This may be
accomplished with the help of the \emph{\name{Bogoliubov} quasi-particles}
\cite[42]{Bogoliubov57},
%
\begin{align*}
    \upalpha_{\vec k} &=
        u_{\vec k} \op c_{\vec k \up} - v_{\vec k} \op c_{-\vec k \down}^+, &
    \op c_{\vec k \up} &=
        u_{\vec k} \upalpha_{\vec k} + v_{\vec k} \upbeta_{\vec k}^+, \\
    \upbeta_{\vec k} &=
        u_{\vec k} \op c_{-\vec k \down} + v_{\vec k} \op c_{\vec k \up}^+, &
    \op c_{-\vec k \down} &=
        u_{\vec k} \upbeta_{\vec k} - v_{\vec k} \upalpha_{\vec k}^+,
\end{align*}
%
where the coefficients $u_{\vec k}$ and $v_{\vec k}$ may also chosen to be real.
It is desirable that the new operators obey \name{Fermi} commutation relations.
Except with their own adjoints they already anti-commute so that the only
further requirement is
%
\begin{equation*}
    \{ \upalpha_{\vec k}, \upalpha_{\vec k}^+ \}
    = \{ \upbeta_{\vec k}, \upbeta_{\vec k}^+ \}
    = u_{\vec k}^2 + v_{\vec k}^2 \equiv 1.
\end{equation*}
%
A choice of $u_{\vec k}$ and $v_{\vec k}$ which both satisfies the above
relation and diagonalizes the \name{Hamilton} operator is given by
\cite[Eq.~7]{Bogoliubov57}
%
\begin{equation*}
    \begin{Bmatrix} u_{\vec k}^2 \\ v_{\vec k}^2 \end{Bmatrix} = \frac 1 2
    \bigg[ 1 \pm \frac{\epsilon_{\vec k}}{E_{\vec k}} \bigg]
    \quad \text{with} \quad
    E_{\vec k} = \sqrt{\epsilon_{\vec k}^2 + \Delta_{\vec k}^2},
\end{equation*}
%
where braces enclose alternatives. Therewith, the free and interacting parts are
found to be
%
\begin{align*}
    \sum_{\vec k \sigma} \epsilon_{\vec k} \op n_{\vec k \sigma}
    &= \sum_{\vec k} \bigg[
        \frac{\epsilon_{\vec k}^2}{E_{\vec k}}
            ( \upalpha_{\vec k}^+ \upalpha_{\vec k}
            + \upbeta_{\vec k}^+ \upbeta_{\vec k} - 1)
        + \frac{\Delta_{\vec k} \epsilon_{\vec k}}{E_{\vec k}}
            ( \upalpha_{\vec k}^+ \upbeta_{\vec k}^+
            + \upbeta_{\vec k} \upalpha_{\vec k} )
        + \epsilon_{\vec k}
    \bigg],
    \\
    -\sum_{\vec k} \Delta_{\vec k} [\op C_{\vec k} + \op C_{\vec k}^+]
    &= \sum_{\vec k} \bigg[
        \frac{\Delta_{\vec k}^2}{E_{\vec k}}
            ( \upalpha_{\vec k}^+ \upalpha_{\vec k}
            + \upbeta_{\vec k}^+ \upbeta_{\vec k} - 1)
        - \frac{\Delta_{\vec k} \epsilon_{\vec k}}{E_{\vec k}}
            ( \upalpha_{\vec k}^+ \upbeta_{\vec k}^+
            + \upbeta_{\vec k} \upalpha_{\vec k} )
    \bigg],
\end{align*}
%
of which constants will again be neglected. Altogether, the diagonal
\name{Hamilton} operator reads
%
\begin{equation*}
    \op H =
    \sum_{\vec k} E_{\vec k}
        ( \upalpha_{\vec k}^+ \upalpha_{\vec k}
        + \upbeta_{\vec k}^+ \upbeta_{\vec k} ).
\end{equation*}
%
This identifies $E_{\vec k}$ as the new single-particle energy and $\Delta_{\vec
k}$ as half of a band gap which has opened up in the spectrum. The latter
remains to be determined self-consistently:
%
\begin{align*}
    \Delta_{\vec k} &= \sum_{\vec k'} V_{\vec k \vec k'} [
        u_{\vec k'}^2 \av{\upbeta_{\vec k'} \upalpha_{\vec k'}}
        - v_{\vec k'}^2 \av{\upalpha_{\vec k'}^+ \upbeta_{\vec k'}^+}
        + u_{\vec k'} v_{\vec k'} (
            1 - \av{\upalpha_{\vec k'}^+ \upalpha_{\vec k'}}
            - \av{\upbeta_{\vec k'}^+ \upbeta_{\vec k'}}
            )
        ]
    \\
    &= \frac 1 2 \sum_{\vec k} V_{\vec k \vec k'}
    \frac{\Delta_{\vec k'}}{E_{\vec k'}} [1 - 2 f_+(E_{\vec k'})]
    = \frac 1 2 \sum_{\vec k} V_{\vec k \vec k'}
    \frac{\Delta_{\vec k'}}{E_{\vec k'}} \tanh \frac{E_{\vec k'}}{2 T}.
\end{align*}
%
Above, averages of operators which do not conserve the number of
\name{Bogoliubov} quasi-particles vanish, while the average occupation numbers
are given by \name{Fermi} functions $f_+(E_{\vec k})$ as will be derived in
Section~\ref{free particles}. The trivial solution $\Delta_{\vec k} = 0$ to the
above equation exists for all temperatures $T$ and represents the normal state.
The presence of a non-zero solution characterizes the superconducting state.

From the definition of $V_{\vec k \vec k'}$ it follows that also $\Delta_{\vec
k}$ is a constant $\Delta$ for $\vec k$ within the \name{Fermi} shell and zero
otherwise. Hence, dividing by $\Delta \neq 0$,
%
\begin{equation*}
    1 = \frac V 2
    \sum_{\vec k}^{|\epsilon_{\vec k} < \omega \sub D|}
    \frac
        {\tanh \frac{\sqrt{\epsilon_{\vec k}^2 + \Delta^2}}{2 T}}
        {\sqrt{\epsilon_{\vec k}^2 + \Delta^2}}
    = \frac V 2 \int \from{-\omega \sub D} \till{\omega \sub D} \D \epsilon \,
    N(\epsilon) \frac
        {\tanh \frac{\sqrt{\epsilon^2 + \Delta^2}}{2 T}}
        {\sqrt{\epsilon^2 + \Delta^2}},
\end{equation*}
%
where $N(\epsilon)$ is the density of states, usually assumed to be constant
over the range of integration. This is the famous \emph{BCS gap equation}. The
resulting temperature dependence of the order parameter for different coupling
strengths is displayed in Fig.~\ref{BCS gap}.
