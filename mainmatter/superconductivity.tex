% !TEX root = ../thesis.tex

\chapter{Superconductivity}

This chapter gives an introduction to the field of superconductivity and is not
intended to be exhaustive. The theory that will actually be applied in this work
is presented the following two chapters.

In the first section the earlier history of superconductivity is briefly
reviewed up to the point when the interaction of electrons and phonons is
identified as causing the phenomenon.%
%
\footnote{The history is retold following H. \name{Fröhlich}
\cite{Froehlich82}.}
%
Next, a canonical transformation is presented which reveals how this interaction
can lead to an effective attraction between electrons and consequently to the
formation of \name{Cooper} pairs. Thereafter, the theory of \name{Bardeen},
\name{Cooper} and \name{Schrieffer} is presented, followed by a description of
the concept of \q{off-diagonal long-range order}.

\section{History of superconductivity}

In 1911 the dutch physicist H. \name{Kamerlingh Onnes} finds that mercury ceases
to resist electric current completely when cooled down to temperatures below
$4\,^\circ \mathrm K$ with the help of liquid helium.%
%
\footnote{According to Ref.~\barecite{vanDelftKes10}, on April 8, 1911
\name{Kamerlingh Onnes} writes \qq{Kwik nagenoeg nul} in his notebook, which
means \qq{[resistance of] mercury near-enough zero}. In Communication No. 122b
from the Physical Laboratory at Leiden he states more precisely: \qq{At
$3\,^\circ \mathrm K$, the resistance was found to have fallen below [\dots] one
ten-millionth of the value which it would have at $0\,^\circ \mathrm C$}
\cite{KamerlinghOnnes11}.}
%
This is the first time a manifestation of superconductivity is observed. In the
years that follow, similar observations are made for other metals and the
fundamental properties of the novel state are exposed.

\section{Canonical transformation}

It is now shown that the electron-phonon interaction can involve an effective
electron-electron interaction which is attractive.%
%
\footnote{The original derivation is due to \name{Fröhlich} \cite{Froehlich52}.
The presentation at hand is rather guided by \name{Czycholl} \cite{Czycholl08}.}
%
To that end the \name{Fröhlich-Hamilton} operator
%
\begin{gather*}
    \op H = \overbrace{
        \sum_{\vec k} \epsilon_{\vec k} \op c_{\vec k}^+ \op c_{\vec k} +
        \sum_{\vec q} \omega_{\vec q} \op b_{\vec q}^+ \op b_{\vec q}
    }^{\op H_0} + \overbrace{
        \sum_{\vec k \vec q} g_{\vec q} \op c_{\vec k + \vec q}^+ \op c_{\vec k}
        [\op b_{\vec q} + \op b_{-\vec q}^+]
    }^{\op V}
\end{gather*}
%
is considered, which describes the interaction between electrons with wave
number $\vec k$ and energy $\epsilon_{\vec k}$, which are annihilated and
created by the \name{Fermi} operator $\op a_{\vec k}$ and its adjoint,
respectively, and longitudinal phonons, where analogous definitions hold for
$\vec q$, $\omega_{\vec q}$ and $\op b_{\vec q}$. The strength of the coupling
is given by $g_{\vec q}$. The spin is of no importance here and thus omitted.

The idea is to apply a canonical transformation to the \name{Hamilton} operator
by means of the unitary operator $\E^{\I \op S}$, where $\op S$ is self-adjoint.
With the help of \name{Hadamard}'s lemma,
%
\begin{align*}
    \op H \sub T
    = \E^{-\I \op S} \, \op H \, \E^{\I \op S}
    &= \op H + \I [\op H, \op S] - \frac 1 2 [[\op H, \op S], \op S] + \dots \\
    &= \op H_0 + \op V + \I [\op H_0, \op S] + \I [\op V, \op S]
    - \frac 1 2 [[\op H_0, \op S], \op S] + \dots \\
    &\approx \op H_0 + \frac \I 2 [\op V, \op S] \equiv \op H_0 + \op V',
\end{align*}
%
choosing $[\op H_0, \op S] = \I \op V$, which implies a linear dependence of
$\op S$ on $\op V$. Thus all terms with are at least quadratic in the
interaction are neglected. Making use of the commutation relations
%
\begin{align*}
    [\op b_{\vec q'}^+ \op b_{\vec q'}, \ \op b_{\vec q \phantom +}]
    &= -\op b_{\vec q \phantom +} \delta_{\vec q'}^{\vec q},
    &
    [ \op c_{\vec k' \phantom{+ \vec q'}}^+ \op c_{\vec k'},
    \ \op c_{\vec k + \vec q}^+ \op c_{\vec k} ]
    &= \op c_{\vec k + \vec q}^+ \op c_{\vec k}
    \big[ \delta_{\vec k'}^{\vec k + \vec q} - \delta_{\vec k'}^{\vec k} \big],
    \\
    [\op b_{\vec q'}^+ \op b_{\vec q'}, \ \op b_{-\vec q}^+]
    &= \phantom + \op b_{-\vec q}^+ \delta_{\vec q'}^{-\vec q},
    &
    [ \op c_{\vec k' + \vec q'}^+ \op c_{\vec k'},
    \  \op c_{\vec k  + \vec q }^+ \op c_{\vec k } ]
    &= \op c_{\vec k + \vec q + \vec q'}^+ \op c_{\vec k}
    \delta_{\vec k'}^{\vec k + \vec q}
    - \op c_{\vec k + \vec q}^+ \op c_{\vec k - \vec q'}
    \delta_{\vec k'}^{\vec k - \vec q'}
\end{align*}
%
one can verify both that $\op S$ has the explicit form
%
\begin{equation*}
    \op S = \I \sum_{\vec k \vec q} g_{\vec q}
    \op c_{\vec k + \vec q}^+ \op c_{\vec k} \left[
        \frac {\op b_{\vec q}}
        {\epsilon_{\vec k + \vec q} - \epsilon_{\vec k} - \omega_{\vec q}}
        + \frac {\op b_{-\vec q}^+}
        {\epsilon_{\vec k + \vec q} - \epsilon_{\vec k} + \omega_{-\vec q}}
    \right]
\end{equation*}
%
and, assuming $g_{\vec q}^* = g_{\vec - q}$ and $\omega_{\vec q} = \omega_{-\vec
q}$, that the renormalized interaction is given by
%
\begin{align*}
    \op V' = &\sum_{\vec k \vec k' \vec q}
    \frac {|g_{\vec q}|^2 \omega_{\vec q}}
    {(\epsilon_{\vec k + \vec q} - \epsilon_{\vec k})^2 - \omega_{\vec q}^2}
    \op c_{\vec k  + \vec q}^+ \op c_{\vec k }
    \op c_{\vec k' - \vec q}^+ \op c_{\vec k'} + \dots
    \\
    \dots - \frac 1 2 &\sum_{\vec k \vec q \vec q'} g_{\vec q} g_{\vec q'}
    \big[ \op c_{\vec k + \vec q + \vec q'}^+ \op c_{\vec k}
    - \op c_{\vec k + \vec q}^+ \op c_{\vec k - \vec q'} \big]
    \big[ \op b_{\vec q'} + \op b_{-\vec q'}^+ \big] \left[
        \frac {\op b_{\vec q}}
        {\epsilon_{\vec k + \vec q} - \epsilon_{\vec k} - \omega_{\vec q}}
        + \frac {\op b_{-\vec q}^+}
        {\epsilon_{\vec k + \vec q} - \epsilon_{\vec k} + \omega_{-\vec q}}
    \right].
\end{align*}
%
The first term describes an effective interaction between electrons which is
attractive, i.e. negative, for $|\epsilon_{\vec k + \vec q} - \epsilon_{\vec k}|
< \omega_{\vec q}$. The seconds term describes processes involving two phonons.

\section{BCS theory}

In 1957, \name{Bardeen}, \name{Cooper} and \name{Schrieffer} developed, on the
basis of the results presented in the previous chapter, a model \name{Hamilton}
operator to describe superconductivity. It is reduced to the essentials, namley
a local coupling constant $V$ with a negative sign prepended to it in order to
indicate the attractive nature of the interaction, which is limited to electrons
which form a \name{Cooper} pair. Their energy difference is restricted to a
\name{Fermi} shell with the same width as a typical phonon energy, in this case
the \name{Debye} frequency $\omega \sub D$. The model reads
%
\begin{gather*}
    \op H = \sum_{\vec k \sigma}
    \epsilon_{\vec k} \op c_{\vec k \sigma}^+ \op c_{\vec k \sigma}
    -V \sum_{\vec k \vec k'}^{
        \mathclap{|\epsilon_{\vec k'} - \epsilon_{\vec k}| < \omega \sub D}
        }
    \op C_{\vec k'}^+ \op C_{\vec k},
    \quad \text{where} \quad
    \op C_{\vec k} = \op c_{-\vec k \down} \op c_{\vec k \up}
\end{gather*}
%
represents a \name{Cooper}-pair annihilator, which does \emph{not} satisfy
\name{Bose} commutation relations:
%
\begin{gather*}
    [\op C_{\vec k}, \op C_{\vec k}^+]
    = 1 - \op n_{\vec k \up} - \op n_{\vec k \down}
    \quad \text{with} \quad
    \op n_{\vec k \sigma} = \op c_{\vec k \sigma}^+ \op c_{\vec k \sigma}
\end{gather*}
%
In order to promote the solution of the \name{Hamilton} operator, the
interaction term it is usually factorized into anomalous expectation values:
%
\begin{align*}
	\op C_{\vec k'}^+ \op C_{\vec k}
    = \hypo{
        (\op C_{\vec k'}^+ - \av{\op C_{\vec k'}^+})
        (\op C_{\vec k} - \av{\op C_{\vec k}})
        }{\Cov[C_{\vec k'}^+, C_{\vec k}]}
    + \av{\op C_{\vec k'}^+} \op C_{\vec k}
    + \av{\op C_{\vec k}} \op C_{\vec k'}^+
    - \av{\op C_{\vec k'}^+} \av{\op C_{\vec k}},
\end{align*}
%
whereby the covariance in neglected from now on. The averages of either two
creation or annihilation operator do not vanish, as one might expect, because
the factorized \name{Hamilton} operator which respect to which they are
evaluated no longer conserves the particle number. Thus a self-consistently
problem has emerged. The factorized \name{Hamilton} may be written as
%
\begin{gather*}
    \op H = \sum_{\vec k \sigma}
    \epsilon_{\vec k} \op c_{\vec k \sigma}^+ \op c_{\vec k \sigma} -
    \sum_{\vec k} [\Delta^* \op C_{\vec k} + \Delta \op C_{\vec k}^+]
    + \frac {|\Delta|^2} V,
    \quad \text{where} \quad
    \Delta = V \sum_{\vec k} \av{\op C_{\vec k}}
\end{gather*}
%
will turn out to be a suitable order parameter for the superconducting phase.
The \name{Hamilton} operator is still not diagonal. This may be accomplished
with the help of the \name{Bogoliubov} quasi-particles,
%
\begin{align*}
    \upalpha_{\vec k} &=
        u_{\vec k} \op c_{\vec k \up} - v_{\sub k} \op c_{-\vec k \down}^+, \\
    \upbeta_{\vec k} &=
        u_{\vec k} \op c_{-\vec k \down} + v_{\sub k} \op c_{\vec k \up}^+,
\end{align*}
%
to which no picturesque meaning can be ascribed. It is desirable that the new
operators obey \name{Fermi} commutation relations. Except with their own
adjoints they anti-commute so that the only further requirement is
%
\begin{gather*}
    \Bigg \{
        \begin{Bmatrix} \upalpha_{\vec k}   \\ \upbeta_{\vec k}   \end{Bmatrix},
        \begin{Bmatrix} \upalpha_{\vec k}^+ \\ \upbeta_{\vec k}^+ \end{Bmatrix}
    \Bigg \}
    = |u_{\vec k}|^2 + |v_{\vec k}|^2 \equiv 1,
\end{gather*}
%
where the inner braces enclose alternatives. A choice of $u_{\vec k}$ and
$v_{\vec k}$ which both satisfies the above relation and diagonalized the
\name{Hamilton} operator is given by
%
\begin{gather*}
    \begin{Bmatrix} u_{\vec k}^2 \\ v_{\vec k}^2 \end{Bmatrix} = \frac 1 2
    \Big[ 1 \pm \frac{\epsilon_{\vec k}}{E_{\vec k}} \Big]
    \quad \text{where} \quad
    E_{\vec k} = \sqrt{\epsilon_{\vec k}^2 + |\Delta|^2}.
\end{gather*}
%
Interpreting $E_{\vec k}$ as the new, renormalized single-particle energy,
$|\Delta|$ is identified as half of a band gap which has opened in the spectrum.
The diagonal \name{Hamilton} operator reads
%
\begin{gather*}
    \op H =
    \sum_{\vec k} \{ E_{\vec k} [
        \upalpha_{\vec k}^+ \upalpha_{\vec k}
        + \upbeta_{\vec k}^+ \upbeta_{\vec k} - 1
        ]
    + \epsilon_{\vec k} \} + \frac {|\Delta|^2} V.
\end{gather*}
%
The order parameter or energy gap remains to be determined self-consistently:
%
\begin{gather*}
    \Delta = \dots
    = V \sum_{\vec k} [
        u_{\vec k} v_{\vec k} - v_{\vec k} u_{\vec k}
        \av{\upalpha_{\vec k}^+ \upalpha_{\vec k}} - u_{\vec k} v_{\vec k}
        \av{\upbeta_{\vec k}^+ \upbeta_{\vec k}}
        ]
    = V \sum_{\vec k} \frac \Delta {E_{\vec k}} [1 - f(E_{\vec k})].
\end{gather*}
%
In the last step it has been used that the \name{Bogoliubov} quasi-particles are
the fermions with respect to which the \name{Hamiton} operator is diagonal. In
Section~\ref{free particles} it will be derived that in this case the occurring
averages are simply \name{Fermi} functions $f(\epsilon)$.

The trivial solution $\Delta = 0$ to the above equations always exists and
represents the normal state. Division by $\Delta \neq 0$ and expression of the
\name{Fermi} function in terms of the hyperbolic tangent yields the famous
\emph{BCS self-consistency equation}
%
\begin{gather*}
    1 = \frac V 2 \sum_{\vec k}
    \frac{\tanh(\frac{E_{\vec k}}{2 k T})}{E_{\vec k}}
\end{gather*}
%
for the \emph{order parameter of the superconducting state}, a designation which
is justified by having a look at its temperature dependence, displayed in Fig.~0

In order to solve the equation, the $\vec k$-summation has been replaced by an
integral over the energy range $[-\omega \sub D, \omega \sub D]$, where the
emerging density of states is assumed to a constant $N_0$:
%
\begin{gather*}
    1 = \frac {V N_0} 2
    \int \from{-\omega \sub D} \till{\omega \sub D} \D \epsilon \,
    \frac{\tanh(\frac{E_{\vec k}}{2 k T})}{E_{\vec k}}.
\end{gather*}
%
For $T = 0$ one can further approximate
%
\begin{gather*}
    |\Delta(T = 0)| = \Delta_0 \approx 2 \omega \sub D \E^{-\frac 1 {V \rho_0}}.
\end{gather*}
%
The \emph{critical temperature} $T \sub c$ is determined by setting $\Delta =
0$:
%
\begin{gather*}
    k T \sub c = \frac {\E^\gamma} \pi \Delta_0
    \quad \text{with} \quad \gamma \approx 0.57,
\end{gather*}
%
\name{Euler}'s constant. An indication of the underlying electron-phonon
interaction is given by the isotope effect:
%
\begin{gather*}
    T \sub c \propto \omega \sub D \propto \frac 1 {\sqrt M}.
\end{gather*}

\section{Off-diagonal long-range order}

\begin{itemize}
    \item \name{Gor'kov-Green} functions (equations-of-motion approach?)
    \item creation/annihilation of indeterminate \name{Cooper} pairs
\end{itemize}
