% !TEX root = ../thesis.tex

\chapter{Conclusion}
\label{conclusion}

Principally, the results of the present work are consistent with the critical
temperatures obtained from \name{McMillan}'s formula. In the only case where a
direct comparison is possible, i.e. for scalar coupling strengths and within the
CDOS approximation, the agreement is satisfactory except for large
electron-phonon coupling strengths, a limitation which is well-known though. To
some degree, the conformity may even be enhanced by adjusting the fit parameters
to the considered phonon spectrum. Beyond this, there no one-to-one
correspondence between either the \name{Coulomb} pseudo-potentials entering
\name{McMillan}'s formula and the \name{Eliashberg} equations considering a real
density of states or the coupling matrices and effective scalars in the
multi-band case.

Regarding the \name{Coulomb} pseudo-potential, several observations have been
made: First, the CODS \name{Eliashberg} equations require a cutoff of the
\name{Matsubara} sums. If the original strength of the \name{Coulomb}
interaction is used without any rescaling, the critical temperature does not
converge with an increasing cutoff as it would be the case without the CDOS
approximation. Instead, it is known that the correct results are obtained at a
cutoff frequency similar to the electronic bandwidth. There is, however, a
fairly well justified relation between the parameter for \name{McMillan}'s
equation and the rescaled one to be used at a certain cutoff. Unfortunately this
fails for real densities of states, where a heuristic mapping has to be applied.
As derived and tested in this work, the actual rescaling may then be performed
in consideration of the specific density of states, which yields an improved the
rate of convergence.

The \name{Coulomb} interaction apart, the predictions of \name{McMillan}'s
formula are also quite accurate beyond the CDOS approach, as long as the density
of states near the chemical potential varies slowly. Both at \name{van Hove}
singularities and band edges is has been observed that the effective
contribution of the density of states is less than the one with respect to which
the coupling strengths are defined, which leads to overestimated critical
temperatures.

Taking multi-band interactions into account, testing the validity of
\name{McMillan}'s formula is equivalent to searching appropriate mappings from
coupling matrices onto effective scalars which yield the same critical
temperature. Hereof, no exact solution may be reported but two approximations
which yield acceptable results. One of them neglects the energy renormalization
and tends to overestimate the coupling strength, the other has been derived at
the lowest cutoff frequency possible but astonishingly maintains most of its
accuracy up to reasonable cutoffs.

Furthermore, for a two-band system it has been found that either the diagonal or
off-diagonal elements of the electron-phonon coupling matrix may be varied along
specific hyperbolas without altering the corresponding critical temperature. An
exact mapping would have been found if the position of the asymptotes as a
function of the remaining matrix elements were known. This has not been
accomplished and is thus a possible object of further research. The two
approximate mappings only give some qualitative results on that score.

Besides questions concerning the critical temperature, the thesis endeavors to
give a useful overview of the \name{Eliashberg} theory of superconductivity,
including its derivation from the fundamental interactions as well as
discussions of the variety of special cases and approximations such as the CDOS
approach. Moreover, software has been developed which is yet to be applied in a
more physical context such as the description of realistic materials.
