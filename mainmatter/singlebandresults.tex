% !TEX root = ../thesis.tex

\chapter{Single-band results}

Having presented most of the analytic framework, the following chapters will be
dedicated to the presentation of more specific, mostly numerical results. For
now, only a single electronic band is taken into account.

To make a start, the self-energy components which constitute the solution of the
\name{Eliashberg} equations will be presented as functions of both
\name{Matsubara} and real frequencies. Before that, however, the analytic
continuation by means of \name{Padé} approximants shall be validated and an
exemplary density of states to work with introduced. Next, several convergence
tests are performed which guarantee the accuracy of the following results:
\name{McMillan}'s equation is adapted to the special case of \name{Einstein}
phonon spectra and subsequently tested as part of a series of
critical-temperature benchmarks. Finally, the influence of density of states and
particle number is discussed in detail.

\section{Preliminary considerations}

\subsection{Validation of \name{Padé} approximant}

\begin{figure}
    \input{results/renormalization/imaginary-axis.sl}
    \input{results/renormalization/real-axis.sl}
    \caption{Exact normal-state cDOS renormalization together with selected
             \name{Padé} approximants}
    \label{validation Pade}
\end{figure}
%
In this section the suitability of \name{Padé} approximants to perform an
analytic continuation of numerical data is tested using the example of the only
self-energy component of interest for which an analytic expression is available,
namely the renormalization function in the normal state within the cDOS
approximation.

With the help of Eq.~A.7 of Ref.~\barecite{AllenMitrovic82} one can easily
extend the domain of Eq.~\ref{normal-state renormalization} from the
\name{Matsubara} frequencies on the imaginary axis to the whole complex plane.
For a single band,
%
\begin{equation*}
    Z(\omega) = 1 + \frac {\pi \I T} \omega \lambda \bigg \{
        1 + \frac{\omega \sub E}{2 \pi \I T} \Big[
              \psi(\tfrac 1 2 + \tfrac{\omega \sub E + \omega}{2 \pi \I T})
            - \psi(\tfrac 1 2 - \tfrac{\omega \sub E - \omega}{2 \pi \I T})
            + \psi(1 - \tfrac{\omega \sub E}{2 \pi \I T})
            - \psi(1 + \tfrac{\omega \sub E}{2 \pi \I T})
        \Big]
    \bigg \}
\end{equation*}
%
In Fig.~\ref{validation Pade}, $Z(\omega)$ is plotted on both the real and the
imaginary axis. In the former case it is complemented with some \name{Padé}
approximants which interpolate the imaginary-axis result $Z(\I \omega_n)$ at all
\name{Matsubara} frequencies $\omega_n \in (0, \omega \sub{max.})$. All beyond
the respective $\omega \sub{max.}$ is discarded.

On the imaginary axis the renormalization is real and bell-shaped with the
center at the origin. On the real axis it is more complicated: There is a peak
with an imaginary discontinuity at the phonon frequency $\pm \omega \sub E$.
Below that, the imaginary part vanishes and the real part increases with
frequency starting at $1 + \lambda$. Beyond that, real and imaginary parts decay
towards unity and zero, respectively.

It turns out that the quality of the \name{Padé} approximant increases with
$\omega \sub{max.}$, as expected. Already for small multiples of $\omega \sub E$
the exact and approximate curves coincide to a high degree.

\subsection{Density of states of the square lattice}

\section{Self-energy on real and imaginary axis}

\begin{figure}
    \small
	\input{results/self-energy/Delta(iomega).sl}
	\input{results/self-energy/Delta(omega).sl}
	\input{results/self-energy/chi(iomega).sl}
	\input{results/self-energy/chi(omega).sl}
	\input{results/self-energy/Z(iomega).sl}
	\input{results/self-energy/Z(omega).sl}
	\caption{Imaginary- and real-axis self-energy components}
\end{figure}

\begin{itemize}
    \item Magnitude of order parameter changes with temperature as opposed to
          invariant shape (\name{Vidberg/Serene}).
    \begin{itemize}
        \item[$\rightarrow$] common position of zero-crossing
    \end{itemize}
\end{itemize}

\subsection{Temperature dependence of the order parameter}

\section{Convergence tests}

\subsection{Cutoff-induced errors of the self-energy}

\begin{itemize}
    \item numeric: $Z(\I \omega_n)$, $\Delta(\I \omega_n)$, $\chi(\I \omega_n)$
    \item analytic: $Z(\I \omega_n)$
\end{itemize}

\subsection{Convergence of critical temperature}

\section{\name{McMillan}'s equation for \name{Einstein} spectra}

\section{Critical-temperature benchmarks}

\section{Dependence on density of states and occupancy}

\begin{itemize}
    \item energy shift and chemical potential as function of the initial value
          of the latter
    \item non-well-defined coupling constants because of:
    %
    \begin{itemize}
        \item steep slope of the density of states
        \item non-representative density of states at \name{Fermi} level
    \end{itemize}
\end{itemize}
